% Options for packages loaded elsewhere
\PassOptionsToPackage{unicode}{hyperref}
\PassOptionsToPackage{hyphens}{url}
%
\documentclass[
]{article}
\usepackage{amsmath,amssymb}
\usepackage{lmodern}
\usepackage{iftex}
\ifPDFTeX
  \usepackage[T1]{fontenc}
  \usepackage[utf8]{inputenc}
  \usepackage{textcomp} % provide euro and other symbols
\else % if luatex or xetex
  \usepackage{unicode-math}
  \defaultfontfeatures{Scale=MatchLowercase}
  \defaultfontfeatures[\rmfamily]{Ligatures=TeX,Scale=1}
\fi
% Use upquote if available, for straight quotes in verbatim environments
\IfFileExists{upquote.sty}{\usepackage{upquote}}{}
\IfFileExists{microtype.sty}{% use microtype if available
  \usepackage[]{microtype}
  \UseMicrotypeSet[protrusion]{basicmath} % disable protrusion for tt fonts
}{}
\makeatletter
\@ifundefined{KOMAClassName}{% if non-KOMA class
  \IfFileExists{parskip.sty}{%
    \usepackage{parskip}
  }{% else
    \setlength{\parindent}{0pt}
    \setlength{\parskip}{6pt plus 2pt minus 1pt}}
}{% if KOMA class
  \KOMAoptions{parskip=half}}
\makeatother
\usepackage{xcolor}
\usepackage{longtable,booktabs,array}
\usepackage{calc} % for calculating minipage widths
% Correct order of tables after \paragraph or \subparagraph
\usepackage{etoolbox}
\makeatletter
\patchcmd\longtable{\par}{\if@noskipsec\mbox{}\fi\par}{}{}
\makeatother
% Allow footnotes in longtable head/foot
\IfFileExists{footnotehyper.sty}{\usepackage{footnotehyper}}{\usepackage{footnote}}
\makesavenoteenv{longtable}
\usepackage{graphicx}
\makeatletter
\def\maxwidth{\ifdim\Gin@nat@width>\linewidth\linewidth\else\Gin@nat@width\fi}
\def\maxheight{\ifdim\Gin@nat@height>\textheight\textheight\else\Gin@nat@height\fi}
\makeatother
% Scale images if necessary, so that they will not overflow the page
% margins by default, and it is still possible to overwrite the defaults
% using explicit options in \includegraphics[width, height, ...]{}
\setkeys{Gin}{width=\maxwidth,height=\maxheight,keepaspectratio}
% Set default figure placement to htbp
\makeatletter
\def\fps@figure{htbp}
\makeatother
\setlength{\emergencystretch}{3em} % prevent overfull lines
\providecommand{\tightlist}{%
  \setlength{\itemsep}{0pt}\setlength{\parskip}{0pt}}
\setcounter{secnumdepth}{-\maxdimen} % remove section numbering
\ifLuaTeX
\usepackage[bidi=basic]{babel}
\else
\usepackage[bidi=default]{babel}
\fi
\babelprovide[main,import]{english}
% get rid of language-specific shorthands (see #6817):
\let\LanguageShortHands\languageshorthands
\def\languageshorthands#1{}
\ifLuaTeX
  \usepackage{selnolig}  % disable illegal ligatures
\fi
\IfFileExists{bookmark.sty}{\usepackage{bookmark}}{\usepackage{hyperref}}
\IfFileExists{xurl.sty}{\usepackage{xurl}}{} % add URL line breaks if available
\urlstyle{same} % disable monospaced font for URLs
\hypersetup{
  pdftitle={The Project Gutenberg eBook of Bookbinding, and the Care of Books, by Douglas Cockerell},
  pdflang={en},
  hidelinks,
  pdfcreator={LaTeX via pandoc}}

\title{The Project Gutenberg eBook of Bookbinding, and the Care of
Books, by Douglas Cockerell}
\author{}
\date{}

\begin{document}
\maketitle

\begin{verbatim}

Project Gutenberg's Bookbinding, and the Care of Books, by Douglas Cockerell

This eBook is for the use of anyone anywhere at no cost and with
almost no restrictions whatsoever.  You may copy it, give it away or
re-use it under the terms of the Project Gutenberg License included
with this eBook or online at www.gutenberg.org


Title: Bookbinding, and the Care of Books
       A handbook for Amateurs, Bookbinders & Librarians

Author: Douglas Cockerell

Editor: W. R. Lethaby

Illustrator: Noel Rooke

Release Date: September 19, 2008 [EBook #26672]

Language: English

Character set encoding: ISO-8859-1

*** START OF THIS PROJECT GUTENBERG EBOOK BOOKBINDING, AND THE CARE OF BOOKS ***




Produced by Suzanne Shell, Irma Spehar and the Online
Distributed Proofreading Team at http://www.pgdp.net





\end{verbatim}

THE ARTISTIC CRAFTS SERIES\\
OF TECHNICAL HANDBOOKS\\
EDITED BY W.~R. LETHABY

BOOKBINDING

\hypertarget{bookbinding-and-the-care-of-books}{%
\section{\texorpdfstring{BOOKBINDING, AND\\
THE CARE OF
BOOKS}{BOOKBINDING, AND THE CARE OF BOOKS}}\label{bookbinding-and-the-care-of-books}}

A HANDBOOK FOR AMATEURS BOOKBINDERS \& LIBRARIANS BY DOUGLAS COCKERELL\\
WITH\\
DRAWINGS BY NOEL ROOKE AND OTHER ILLUSTRATIONS

\includegraphics[width=0.78125in,height=0.90625in]{images/tp01.jpg}

NEW YORK\\
D. APPLETON AND COMPANY\\
1910

{Copyright, 1901,\\
By D. Appleton and Company}\\
\strut \\
\emph{All rights reserved}

\href{images/gs005.jpg}{\includegraphics{images/gs005_th.jpg}}

{White Pigskin.}---\emph{Basle}, 1512.

\hypertarget{editors-preface7}{%
\subsection[EDITOR'S
PREFACE]{\texorpdfstring{\protect\hypertarget{EDITORS_PREFACE}{}{}EDITOR'S
PREFACE{\protect\hypertarget{Page_7}{}{{[}7{]}}}}{EDITOR'S PREFACE{[}7{]}}}\label{editors-preface7}}

{In} issuing this volume of a series of Handbooks on the Artistic
Crafts, it will be well to state what are our general aims.

In the first place, we wish to provide trustworthy text-books of
workshop practice, from the points of view of experts who have
critically examined the methods current in the shops, and putting aside
vain survivals, are prepared to say what is good workmanship, and to set
up a standard of quality in the crafts which are more especially
associated with design. Secondly, in doing this, we hope to treat design
itself as an essential part of good workmanship. During the last century
most of the arts, save painting{\protect\hypertarget{Page_8}{}{{[}8{]}}}
and sculpture of an academic kind, were little considered, and there was
a tendency to look on ``design'' as a mere matter of \emph{appearance}.
Such ``ornamentation'' as there was was usually obtained by following in
a mechanical way a drawing provided by an artist who often knew little
of the technical processes involved in production. With the critical
attention given to the crafts by Ruskin and Morris, it came to be seen
that it was impossible to detach design from craft in this way, and
that, in the widest sense, true design is an inseparable element of good
quality, involving as it does the selection of good and suitable
material, contrivance for special purpose, expert workmanship, proper
finish and so on, far more than mere ornament, and indeed, that
ornamentation itself was rather an exuberance of fine workmanship than a
matter of merely abstract lines. Workmanship when separated by too wide
a gulf from fresh thought---that is, from design---inevitably decays,
and, on the other hand,{\protect\hypertarget{Page_9}{}{{[}9{]}}}
ornamentation, divorced from workmanship, is necessarily unreal, and
quickly falls into affectation. Proper ornamentation may be defined as a
language addressed to the eye; it is pleasant thought expressed in the
speech of the tool.

In the third place, we would have this series put artistic craftsmanship
before people as furnishing reasonable occupation for those who would
gain a livelihood. Although within the bounds of academic art, the
competition, of its kind, is so acute that only a very few per cent. can
fairly hope to succeed as painters and sculptors; yet, as artistic
craftsmen, there is every probability that nearly every one who would
pass through a sufficient period of apprenticeship to workmanship and
design would reach a measure of success.

In the blending of handwork and thought in such arts as we propose to
deal with, happy careers may be found as far removed from the dreary
routine of hack labour, as from the terrible
uncertainty{\protect\hypertarget{Page_10}{}{{[}10{]}}} of academic art.
It is desirable in every way that men of good education should be
brought back into the productive crafts: there are more than enough of
us ``in the city,'' and it is probable that more consideration will be
given in this century than in the last to Design and Workmanship.

W.~R. LETHABY.

\hypertarget{authors-note11}{%
\subsection[AUTHOR'S
NOTE]{\texorpdfstring{\protect\hypertarget{AUTHORS_NOTE}{}{}AUTHOR'S
NOTE{\protect\hypertarget{Page_11}{}{{[}11{]}}}}{AUTHOR'S NOTE{[}11{]}}}\label{authors-note11}}

{It} is hoped that this book will help bookbinders and librarians to
select sound methods of binding books.

It is intended to supplement and not to supplant workshop training for
bookbinders. No one can become a skilled workman by reading text-books,
but to a man who has acquired skill and practical experience, a
text-book, giving perhaps different methods from those to which he has
been accustomed, may be helpful.

My thanks are due to many friends, including the workmen in my workshop,
for useful suggestions and other help, and to the Society of Arts for
permission to quote from the report of their Special Committee on
leather for bookbinding.{\protect\hypertarget{Page_12}{}{{[}12{]}}}

I should also like to express my indebtedness to my master, Mr. T.~J.
Cobden-Sanderson, for it was in his workshop that I learned my craft,
and anything that may be of value in this book is due to his influence.

D.~C.

\emph{November} 1901.

\hypertarget{contents13}{%
\subsection[CONTENTS]{\texorpdfstring{\protect\hypertarget{CONTENTS}{}{}CONTENTS{\protect\hypertarget{Page_13}{}{{[}13{]}}}}{CONTENTS{[}13{]}}}\label{contents13}}

\begin{longtable}[]{@{}ll@{}}
\toprule()
\endhead
\multicolumn{2}{@{}>{\raggedright\arraybackslash}p{(\columnwidth - 2\tabcolsep) * \real{0.0000} + 2\tabcolsep}@{}}{%
PART I} \\
\multicolumn{2}{@{}>{\raggedright\arraybackslash}p{(\columnwidth - 2\tabcolsep) * \real{0.0000} + 2\tabcolsep}@{}}{%
{\emph{BINDING}}} \\
~ & {PAGE} \\
Editor's Preface & \protect\hyperlink{Page_7}{7} \\
Author's Note & \protect\hyperlink{Page_11}{11} \\
\multicolumn{2}{@{}>{\raggedright\arraybackslash}p{(\columnwidth - 2\tabcolsep) * \real{0.0000} + 2\tabcolsep}@{}}{%
CHAPTER I} \\
Introduction & \protect\hyperlink{Page_17}{17} \\
\multicolumn{2}{@{}>{\raggedright\arraybackslash}p{(\columnwidth - 2\tabcolsep) * \real{0.0000} + 2\tabcolsep}@{}}{%
CHAPTER II} \\
Entering---Books in Sheets---Folding---Collating---Pulling to
Pieces---Refolding---Knocking out Joints &
\protect\hyperlink{Page_33}{33} \\
\multicolumn{2}{@{}>{\raggedright\arraybackslash}p{(\columnwidth - 2\tabcolsep) * \real{0.0000} + 2\tabcolsep}@{}}{%
CHAPTER III} \\
Guarding---Throwing Out---Paring Paper---Soaking off India
Proofs---Mounting very Thin Paper---Splitting
Paper---Inlaying---Flattening Vellum &
\protect\hyperlink{Page_53}{53} \\
\multicolumn{2}{@{}>{\raggedright\arraybackslash}p{(\columnwidth - 2\tabcolsep) * \real{0.0000} + 2\tabcolsep}@{}}{%
CHAPTER IV} \\
{\protect\hypertarget{Page_14}{}{{[}14{]}}}Sizing---Washing---Mending &
\protect\hyperlink{Page_67}{67} \\
\multicolumn{2}{@{}>{\raggedright\arraybackslash}p{(\columnwidth - 2\tabcolsep) * \real{0.0000} + 2\tabcolsep}@{}}{%
CHAPTER V} \\
End Papers---Leather Joints---Pressing &
\protect\hyperlink{Page_80}{80} \\
\multicolumn{2}{@{}>{\raggedright\arraybackslash}p{(\columnwidth - 2\tabcolsep) * \real{0.0000} + 2\tabcolsep}@{}}{%
CHAPTER VI} \\
Trimming Edges before Sewing---Edge Gilding &
\protect\hyperlink{Page_92}{92} \\
\multicolumn{2}{@{}>{\raggedright\arraybackslash}p{(\columnwidth - 2\tabcolsep) * \real{0.0000} + 2\tabcolsep}@{}}{%
CHAPTER VII} \\
Marking up---Sewing---Materials for Sewing &
\protect\hyperlink{Page_98}{98} \\
\multicolumn{2}{@{}>{\raggedright\arraybackslash}p{(\columnwidth - 2\tabcolsep) * \real{0.0000} + 2\tabcolsep}@{}}{%
CHAPTER VIII} \\
Fraying out Slips---Glueing up---Rounding and Backing &
\protect\hyperlink{Page_114}{114} \\
\multicolumn{2}{@{}>{\raggedright\arraybackslash}p{(\columnwidth - 2\tabcolsep) * \real{0.0000} + 2\tabcolsep}@{}}{%
CHAPTER IX} \\
Cutting and Attaching Boards---Cleaning off Back---Pressing &
\protect\hyperlink{Page_124}{124} \\
\multicolumn{2}{@{}>{\raggedright\arraybackslash}p{(\columnwidth - 2\tabcolsep) * \real{0.0000} + 2\tabcolsep}@{}}{%
CHAPTER X} \\
Cutting in Boards---Gilding and Colouring Edges &
\protect\hyperlink{Page_139}{139} \\
\multicolumn{2}{@{}>{\raggedright\arraybackslash}p{(\columnwidth - 2\tabcolsep) * \real{0.0000} + 2\tabcolsep}@{}}{%
CHAPTER XI} \\
Headbanding & \protect\hyperlink{Page_147}{147} \\
\multicolumn{2}{@{}>{\raggedright\arraybackslash}p{(\columnwidth - 2\tabcolsep) * \real{0.0000} + 2\tabcolsep}@{}}{%
CHAPTER XII} \\
Preparing for Covering---Paring Leather---Covering---Mitring
{\protect\hypertarget{Page_15}{}{{[}15{]}}}Corners---Filling-in Boards &
\protect\hyperlink{Page_152}{152} \\
\multicolumn{2}{@{}>{\raggedright\arraybackslash}p{(\columnwidth - 2\tabcolsep) * \real{0.0000} + 2\tabcolsep}@{}}{%
CHAPTER XIII} \\
Library Binding---Binding very Thin Books---Scrap-Books---Binding in
Vellum---Books covered with Embroidery &
\protect\hyperlink{Page_173}{173} \\
\multicolumn{2}{@{}>{\raggedright\arraybackslash}p{(\columnwidth - 2\tabcolsep) * \real{0.0000} + 2\tabcolsep}@{}}{%
CHAPTER XIV} \\
Decoration---Tools---Finishing---Tooling on Vellum---Inlaying on Leather
& \protect\hyperlink{Page_188}{188} \\
\multicolumn{2}{@{}>{\raggedright\arraybackslash}p{(\columnwidth - 2\tabcolsep) * \real{0.0000} + 2\tabcolsep}@{}}{%
CHAPTER XV} \\
Lettering---Blind Tooling---Heraldic Ornament &
\protect\hyperlink{Page_215}{215} \\
\multicolumn{2}{@{}>{\raggedright\arraybackslash}p{(\columnwidth - 2\tabcolsep) * \real{0.0000} + 2\tabcolsep}@{}}{%
CHAPTER XVI} \\
Designing for Gold-Tooled Decoration &
\protect\hyperlink{Page_230}{230} \\
\multicolumn{2}{@{}>{\raggedright\arraybackslash}p{(\columnwidth - 2\tabcolsep) * \real{0.0000} + 2\tabcolsep}@{}}{%
CHAPTER XVII} \\
Pasting down End Papers---Opening Books &
\protect\hyperlink{Page_254}{254} \\
\multicolumn{2}{@{}>{\raggedright\arraybackslash}p{(\columnwidth - 2\tabcolsep) * \real{0.0000} + 2\tabcolsep}@{}}{%
CHAPTER XVIII} \\
Clasps and Ties---Metal on Bindings &
\protect\hyperlink{Page_259}{259} \\
\multicolumn{2}{@{}>{\raggedright\arraybackslash}p{(\columnwidth - 2\tabcolsep) * \real{0.0000} + 2\tabcolsep}@{}}{%
CHAPTER XIX} \\
Leather & \protect\hyperlink{Page_263}{263} \\
\multicolumn{2}{@{}>{\raggedright\arraybackslash}p{(\columnwidth - 2\tabcolsep) * \real{0.0000} + 2\tabcolsep}@{}}{%
CHAPTER XX} \\
{\protect\hypertarget{Page_16}{}{{[}16{]}}}Paper---Pastes---Glue &
\protect\hyperlink{Page_280}{280} \\
\multicolumn{2}{@{}>{\raggedright\arraybackslash}p{(\columnwidth - 2\tabcolsep) * \real{0.0000} + 2\tabcolsep}@{}}{%
PART II} \\
\multicolumn{2}{@{}>{\raggedright\arraybackslash}p{(\columnwidth - 2\tabcolsep) * \real{0.0000} + 2\tabcolsep}@{}}{%
{\emph{CARE OF BOOKS WHEN BOUND}}} \\
\multicolumn{2}{@{}>{\raggedright\arraybackslash}p{(\columnwidth - 2\tabcolsep) * \real{0.0000} + 2\tabcolsep}@{}}{%
CHAPTER XXI} \\
Injurious Influences to which Books are Subjected &
\protect\hyperlink{Page_291}{291} \\
\multicolumn{2}{@{}>{\raggedright\arraybackslash}p{(\columnwidth - 2\tabcolsep) * \real{0.0000} + 2\tabcolsep}@{}}{%
CHAPTER XXII} \\
To Preserve Old Bindings---Re-backing &
\protect\hyperlink{Page_302}{302} \\
{Specifications} & \protect\hyperlink{Page_307}{307} \\
{Glossary} & \protect\hyperlink{Page_313}{313} \\
{Reproductions of Bindings} (Eight Collotypes) &
\protect\hyperlink{Page_319}{319} \\
{Index} & \protect\hyperlink{Page_337}{337} \\
\bottomrule()
\end{longtable}

\hypertarget{part-i-binding}{%
\subsection[PART I\\
BINDING]{\texorpdfstring{\protect\hypertarget{PART_I}{}{}PART I\\
BINDING}{PART I BINDING}}\label{part-i-binding}}

\hypertarget{chapter-i17}{%
\subsection[CHAPTER
I]{\texorpdfstring{\protect\hypertarget{CHAPTER_I}{}{}CHAPTER
I{\protect\hypertarget{Page_17}{}{{[}17{]}}}}{CHAPTER I{[}17{]}}}\label{chapter-i17}}

\hypertarget{introduction}{%
\subparagraph{INTRODUCTION}\label{introduction}}

{The} reasons for binding the leaves of a book are to keep them together
in their proper order, and to protect them. That bindings can be made,
that will adequately protect books, can be seen from the large number of
fifteenth and sixteenth century bindings now existing on books still in
excellent condition. That bindings are made, that fail to protect books,
may be seen by visiting any large library, when it will be found that
many bindings have their boards loose and the leather crumbling to dust.
Nearly all librarians complain, that they have to be
continually{\protect\hypertarget{Page_18}{}{{[}18{]}}} rebinding books,
and this not after four hundred, but after only five or ten years.

It is no exaggeration to say that ninety per cent. of the books bound in
leather during the last thirty years will need rebinding during the next
thirty. The immense expense involved must be a very serious drag on the
usefulness of libraries; and as rebinding is always to some extent
damaging to the leaves of a book, it is not only on account of the
expense that the necessity for it is to be regretted.

The reasons that have led to the production in modern times of bindings
that fail to last for a reasonable time, are twofold. The materials are
badly selected or prepared, and the method of binding is faulty. Another
factor in the decay of bindings, both old and new, is the bad conditions
under which they are often kept.

The object of this text-book is to describe the best methods of
bookbinding, and of keeping books when bound, taking into account the
present-day conditions. No attempt has been made to describe all
possible methods, but only such as appear to have answered best on old
books. The methods described are for binding
that{\protect\hypertarget{Page_19}{}{{[}19{]}}} can be done by hand with
the aid of simple appliances. Large editions of books are now bound, or
rather cased, at an almost incredible speed by the aid of machinery, but
all work that needs personal care and thought on each book, is still
done, and probably always will be done, by hand. Elaborate machinery can
only be economically employed when very large numbers of books have to
be turned out exactly alike.

The ordinary cloth ``binding'' of the trade, is better described as
casing. The methods being different, it is convenient to distinguish
between casing and binding. In binding, the slips are firmly attached to
the boards before covering; in casing, the boards are covered
separately, and afterwards glued on to the book. Very great efforts have
been made in the decoration of cloth covers, and it is a pity that the
methods of construction have not been equally considered. If cloth cases
are to be looked upon as a temporary binding, then it seems a pity to
waste so much trouble on their decoration; and if they are to be looked
upon as permanent binding, it is a pity the construction is not
better.{\protect\hypertarget{Page_20}{}{{[}20{]}}}

For books of only temporary interest, the usual cloth cases answer well
enough; but for books expected to have permanent value, some change is
desirable.

Valuable books should either be issued in bindings that are obviously
temporary, or else in bindings that are strong enough to be considered
permanent. The usual cloth case fails as a temporary binding, because
the methods employed result in serious damage to the sections of the
book, often unfitting them for rebinding, and it fails as a permanent
binding on account of the absence of sound construction.

In a temporary publisher's binding, nothing should be done to the
sections of a book that would injure them. Plates should be guarded, the
sewing should be on tapes, without splitting the head and tail, or
``sawing in'' the backs, of the sections; the backs should be glued up
square without backing. The case may be attached, as is now usual. For a
permanent publisher's binding, something like that recommended for
libraries (page \protect\hyperlink{Page_173}{173}) is suggested, with
either leather or cloth on the back.

At the end of the book four specifications are given (page
\protect\hyperlink{Page_307}{307}). The first
is{\protect\hypertarget{Page_21}{}{{[}21{]}}} suggested for binding
books of special interest or value, where no restriction as to price is
made. A binding under this specification may be decorated to any extent
that the nature of the book justifies. The second is for good binding,
for books of reference and other heavy books that may have a great deal
of wear. All the features of the first that make for the strength of the
binding are retained, while those less essential, that only add to the
appearance, are omitted. Although the binding under this specification
would be much cheaper than that carried out under the first, it would
still be too expensive for the majority of books in most libraries; and
as it would seem to be impossible to further modify this form of
binding, without materially reducing its strength, for cheaper work, a
somewhat different system is recommended. The third specification is
recommended for the binding of the general run of small books in most
libraries. The fourth is a modification of this for pamphlets and other
books of little value, that need to be kept together tidily for
occasional reference.

Thanks, in a great measure, to the
work{\protect\hypertarget{Page_22}{}{{[}22{]}}} of Mr. Cobden-Sanderson,
there is in England the germ of a sound tradition for the best binding.
The Report of the Committee appointed by the Society of Arts to
investigate the cause of the decay of modern leather bindings, should
tend to establish a sound tradition for cheaper work. The third
specification at the end of this book is practically the same as that
given in their Report, and was arrived at by selection, after many
libraries had been examined, and many forms of binding compared.

Up to the end of the eighteenth century the traditional methods of
binding books had altered very little during three hundred years. Books
were generally sewn round five cords, the ends of all of these laced
into the boards, and the leather attached directly to the back. At the
end of the eighteenth century it became customary to pare down leather
until it was as thin as paper, and soon afterwards the use of hollow
backs and false bands became general, and these two things together mark
the beginning of the modern degradation of binding, so far as its
utility as a protection is concerned.

The Society of Arts Committee
report{\protect\hypertarget{Page_23}{}{{[}23{]}}} that the bookbinders
must share with the leather manufacturers and librarians the blame for
the premature decay of modern bindings, because---

``1. Books are sewn on too few, and too thin cords, and the slips are
pared down unduly (for the sake of neatness), and are not in all cases
firmly laced into the boards. This renders the attachment of the boards
to the book almost entirely dependent on the strength of the leather.

``2. The use of hollow backs throws all the strain of opening and
shutting on the joints, and renders the back liable to come right off if
the book is much used.

``3. The leather of the back is apt to become torn through the use of
insufficiently strong headbands, which are unable to stand the strain of
the book being taken from the shelf.

``4. It is a common practice to use far too thin leather; especially to
use large thick skins very much pared down for small books.

``5. The leather is often made very wet and stretched a great deal in
covering, with the result that on drying it is further strained, almost
to breaking point, by
contraction,{\protect\hypertarget{Page_24}{}{{[}24{]}}} leaving a very
small margin of strength to meet the accidents of use.''

The history of the general introduction of hollow backs is probably
somewhat as follows: Leather was doubtless first chosen for covering the
backs of books because of its toughness and flexibility; because, while
protecting the back, it would bend when the book was opened and allow
the back to ``throw up'' (see \protect\hyperlink{Fig_1}{fig. 1}, A).
When gold tooling became common, and the backs of books were elaborately
decorated, it was found that the creasing of the leather injured the
brightness or the gold and caused it to crack. To avoid this the binders
lined up the back until it was as stiff as a block of wood. The back
would then not ``throw up'' as the book was opened, the leather would
not be creased, and the gold would remain uninjured (see
\protect\hyperlink{Fig_1}{fig. 1}, B). This was all very well for the
gold, but a book so treated does not open fully, and indeed, if the
paper is stiff, can hardly be got to open at all. To overcome both
difficulties the hollow back was introduced, and as projecting bands
would have been in the way, the sewing cord was sunk in saw cuts made
across the back of the book.{\protect\hypertarget{Page_25}{}{{[}25{]}}}

\protect\hypertarget{Fig_1}{}{}
\includegraphics[width=3.125in,height=3.80208in]{images/gs026.jpg}

Fig. 1.

The use of hollow backs was a very ingenious way out of the difficulty,
as with them the backs could be made to ``throw up,'' and at the same
time the leather was not disturbed (see \protect\hyperlink{Fig_1}{fig.
1}, C). The method of ``sawing in'' bands was known for a long time
before the general{\protect\hypertarget{Page_26}{}{{[}26{]}}} use of
hollow backs. It has been used to avoid the raised bands on books
covered with embroidered material.

If a book is sewn on tapes, and the back lined with leather, there is no
serious objection to a carefully-made hollow back without bands. The
vellum binders use hollow backs made in this way for great account books
that stand an immense amount of wear. They make the ``hollow'' very
stiff, so that it acts as a spring to throw the back up.

But although, if carefully done, satisfactory bindings may be made with
hollow backs, their use has resulted in the production of worthless
bindings with little strength, and yet with the appearance of better
work.

The public having been accustomed to raised bands on the backs of books,
and the real bands being sunk in the back, the binders put false ones
over the ``hollow.'' To save money or trouble, the bands being out of
sight, the book would be sewn on only three or sometimes only two cords,
the usual five false ones still showing at the back. Often only two out
of the three bands would be laced into the board, and sometimes the
slips would not be laced{\protect\hypertarget{Page_27}{}{{[}27{]}}} in
at all. Again, false headbands worked by the yard by machinery would be
stuck on at the head and tail, and a ``hollow'' made with brown paper.
Then leather so thin as to have but little strength, but used because it
is easy to work and needs no paring, would be stuck on. The back would
often be full gilt and lettered, and the sides sprinkled or marbled,
thus further damaging the leather.

In every large library hundreds of books bound somewhat on these lines
may be seen. When they are received from the binder they have the
appearance of being well bound, they look smart on the shelf, but in a
few years, whether they are used or not, the leather will have perished
and the boards become detached, and they will have to be rebound.

As long as librarians expect the appearance of a guinea binding for two
or three shillings, such shams will be produced. The librarian generally
gets his money's worth, for it would be impossible for the binder to do
better work at the price usually paid without materially altering the
appearance of the binding. The polished calf and imitation crushed
morocco must go, and in its place a
rougher,{\protect\hypertarget{Page_28}{}{{[}28{]}}} thicker leather must
be employed. The full-gilt backs must go, the coloured lettering panel
must go, the hollow backs must go, but in the place of these we may have
the books sewn on tapes with the ends securely fastened into split
boards, and the thick leather attached directly to the backs of the
sections. (See specification III. page
\protect\hyperlink{Page_307}{307}.)

Such a binding would look well and not be more expensive than the usual
library binding. It should allow the book to open flat, and if the
materials are well selected, be very durable, and specially strong in
the joints, the weak place in most bindings. The lettering on the back
may be damaged in time if the book is much used, but if so it can easily
be renewed at a fraction of the cost of rebinding, and without injury to
the book.

While the majority of books in most libraries must be bound at a small
cost, at most not exceeding a few shillings a volume, there is a large
demand for good plain bindings, and a limited, but growing, demand for
more or less decorated bindings for special books.

Any decoration but the simplest should be restricted to books bound as
well as{\protect\hypertarget{Page_29}{}{{[}29{]}}} the binder can do
them. The presence of decoration should be evidence that the binder,
after doing his best with the ``forwarding,'' has had time in which to
try to make his work a beautiful, as well as a serviceable, production.

Many books, although well bound, are better left plain, or with only a
little decoration. But occasionally there are books that the binder can
decorate as lavishly as he is able. As an instance of bindings that
cannot be over-decorated, those books which are used in important
ceremonies, such as Altar Books, may be mentioned. Such books may be
decorated with gold and colour until they seem to be covered in a golden
material. They will be but spots of gorgeousness in a great church or
cathedral, and they cannot be said to be over-decorated as long as the
decoration is good.

So, occasionally some one may have a book to which he is for some reason
greatly attached, and wishing to enshrine it, give the binder a free
hand to do his best with it. The binder may wish to make a delicate
pattern with nicely-balanced spots of ornament, leaving the leather for
the most part bare, or he may{\protect\hypertarget{Page_30}{}{{[}30{]}}}
wish to cover the outside with some close gold-tooled pattern, giving a
richness of texture hardly to be got by other means. If he decides on
the latter, many people will say that the cover is over-decorated. But
as a book cover can never be seen absolutely alone, it should not be
judged as an isolated thing covered with ornament without relief, but as
a spot of brightness and interest among its surroundings. If a room and
everything in it is covered with elaborate pattern, then anything with a
plain surface would be welcome as a relief; but in a room which is
reasonably free from ornament, a spot of rich decoration should be
welcome.

It is not contended that the only, or necessarily the best, method of
decorating book covers is by elaborate all-over gold-tooled pattern; but
it is contended that this is a legitimate method of decoration for
exceptional books, and that by its use it is possible to get a beautiful
effect well worth the trouble and expense involved.

Good leather has a beautiful surface, and may sometimes be got of a fine
colour. The binder may often wish to show this surface and colour, and
to restrict his decoration to small portions of the
cover,{\protect\hypertarget{Page_31}{}{{[}31{]}}} and this quite
rightly, he aiming at, and getting, a totally different effect than that
got by all-over patterns. Both methods are right if well done, and both
methods can equally be vulgarised if badly done.

A much debated question is, how far the decoration of a binding should
be influenced by the contents of the book? A certain appropriateness
there should be, but as a general thing, if the binder aims at making
the cover beautiful, that is the best he can do. The hints given for
designing are not intended to stop the development of the student's own
ideas, but only to encourage their development on right lines.

There should be a certain similarity of treatment between the general
get-up of a book and its binding. It is a great pity that printers and
binders have drifted so far apart; they are, or should be, working for
one end, the production of a book, and some unity of aim should be
evident in the work of the two.

The binding of manuscripts and early printed books should be strong and
simple. It should be as strong and durable as the original old bindings,
and, like them, last with reasonable care for four hundred
years{\protect\hypertarget{Page_32}{}{{[}32{]}}} or more. To this end
the old bindings, with their stout sewing cord, wooden boards, and
clasps, may be taken as models.

The question is constantly asked, especially by women, if a living can
be made by setting up as bookbinders. Cheap binding can most
economically be done in large workshops, but probably the best bindings
can be done more satisfactorily by binders working alone, or in very
small workshops.

If any one intends to set up as a bookbinder, doing all the work without
help, it is necessary to charge very high prices to get any adequate
return after the working expenses have been paid. In order to get high
prices, the standard of work must be very high; and in order to attain a
high enough standard of work, a very thorough training is necessary. It
is desirable that any one hoping to make money at the craft should have
at least a year's training in a workshop where good work is done, and
after that, some time will be spent before quite satisfactory work can
be turned out rapidly enough to pay, supposing that orders can be
obtained or the books bound can be sold.

There are some successful binders
who{\protect\hypertarget{Page_33}{}{{[}33{]}}} have had less than a
year's training, but they are exceptional. Those who have not been
accustomed to manual work have usually, in addition to the necessary
skill, to acquire the habit of continuous work. Bookbinding seems to
offer an opening for well-educated youths who are willing to serve an
apprenticeship in a good shop, and who have some small amount of capital
at their command.

In addition to the production of decorated bindings, there is much to be
done by specialising in certain kinds of work requiring special
knowledge. Repairing and binding early printed books and manuscripts, or
the restoration of Parish Registers and Accounts, may be suggested.

\hypertarget{chapter-ii}{%
\subsection[CHAPTER
II]{\texorpdfstring{\protect\hypertarget{CHAPTER_II}{}{}CHAPTER
II}{CHAPTER II}}\label{chapter-ii}}

Entering---Books in Sheets---Folding---Collating---Pulling to
Pieces---Refolding---Knocking out Joints

\hypertarget{entering}{%
\subparagraph{ENTERING}\label{entering}}

{On} receiving a book for binding, its title should be entered in a book
kept for that purpose, with the date of entry,
and{\protect\hypertarget{Page_34}{}{{[}34{]}}} customer's name and
address, and any instructions he may have given, written out in full
underneath, leaving room below to enter the time taken on the various
operations and cost of the materials used. It is well to number the
entry, and to give a corresponding number to the book. It should be at
once collated, and any special features noted, such as pages that need
washing or mending. If the book should prove to be imperfect, or to have
any serious defect, the owner should be communicated with, before it is
pulled to pieces. This is very important, as imperfect books that have
been ``pulled'' are not returnable to the bookseller. Should defects
only be discovered after the book has been taken to pieces, the
bookbinder is liable to be blamed for the loss of any missing leaves.

\hypertarget{books-in-sheets}{%
\subparagraph{BOOKS IN SHEETS}\label{books-in-sheets}}

The sheets of a newly printed book are arranged in piles in the
printer's warehouse, each pile being made up of repetitions of the same
sheet or ``signature.'' Plates or maps are in piles by
themselves{\protect\hypertarget{Page_35}{}{{[}35{]}}} To make a complete
book one sheet is gathered from each pile, beginning at the last sheet
and working backwards to signature A. When a book is ordered from a
publisher in sheets, it is such a ``gathered'' copy that the binder
receives. Some books are printed ``double,'' that is, the type is set up
twice, two copies are printed at once at different ends of a sheet of
paper, and the sheets have to be divided down the middle before the
copies can be separated. Sometimes the title and introduction, or
perhaps only the last sheet, will be printed in this way. Publishers
usually decline to supply in sheets fewer than two copies of such
double-printed books.

If a book is received unfolded, it is generally advisable at once to
fold up the sheets and put them in their proper order, with half-title,
title, introduction, \&c., and, if there are plates, to compare them
with the printed list.

Should there be in a recently published book defects of any kind, such
as soiled sheets, the publisher will usually replace them on
application, although they sometimes take a long time to do so. Such
sheets are called ``imperfections,'' and the printers usually keep a
number of ``overs{\protect\hypertarget{Page_36}{}{{[}36{]}}}'' in order
to make good such imperfections as may occur.

\hypertarget{folding}{%
\subparagraph{FOLDING}\label{folding}}

Books received in sheets must be folded. Folding requires care, or the
margins of different leaves will be unequal, and the lines of printing
not at right angles to the back.

Books of various sizes are known as ``folio,'' ``quarto,'' ``octavo,''
``duodecimo,'' \&c. These names signify the number of folds, and
consequently the number of leaves the paper has been folded into. Thus,
a folio is made up of sheets of paper folded once down the centre,
forming two leaves and four pages. The sheets of a quarto have a second
fold, making four leaves and eight pages, and in an octavo the sheet has
a third fold, forming eight leaves and sixteen pages (see
\protect\hyperlink{Fig_2}{fig. 2}), and so on. Each sheet of paper when
folded constitutes a section, except in the case of folios, where it is
usual to make up the sections by inserting two or more sheets, one
within the other.

Paper is made in several named sizes, such as ``imperial,'' ``royal,''
``demy,{\protect\hypertarget{Page_37}{}{{[}37{]}}}'' ``crown,''
``foolscap,'' \&c. (see p. \protect\hyperlink{Page_283}{283}), so that
the terms ``imperial folio'' or ``crown octavo'' imply that a sheet of a
definite size has been folded a definite number of times.

\protect\hypertarget{Fig_2}{}{}
\includegraphics[width=4.16667in,height=3.5in]{images/gs038.jpg}

Fig. 2.

Besides the traditional sizes, paper is now made of almost any length
and width, resulting in books of odd shape, and the names folio, quarto,
\&c., are rather losing their true meaning, and are often used loosely
to signify pages of certain sizes, irrespective of the number that go to
a sheet.{\protect\hypertarget{Page_38}{}{{[}38{]}}}

On receipt, for instance, of an octavo book for folding, the pile of
sheets is laid flat on the table, and collated by the letter or
signature of each sheet. The first sheet of the book proper will
probably be signature B, as signature A usually consists of the
half-title, title, introduction, \&c., and often has to be folded up
rather differently.

The ``outer'' sides, known by the signature letters B, C, D, \&c.,
should be downwards, and the inner sides facing upwards with the second
signatures, if there are any, B2, C2, D2, \&c., at the right-hand bottom
corner.

The pages of an octave book, commencing at page 1, are shown at
\protect\hyperlink{Fig_3}{fig. 3}. A folder is taken in the right hand,
and held at the bottom of the sheet at about the centre, and the sheet
taken by the left hand at the top right-hand corner and bent over until
pages 3 and 6 come exactly over pages 2 and 7; and when it is seen that
the headlines and figures exactly match,
the{\protect\hypertarget{Page_39}{}{{[}39{]}}} paper, while being held
in that position, is creased down the centre with the folder, and the
fold cut up a little more than half-way. Pages 4, 13, 5, 12 will now be
uppermost; pages 12 and 5 are now folded over to exactly match pages 13
and 4, and{\protect\hypertarget{Page_40}{}{{[}40{]}}} the fold creased
and cut up a little more than half-way, as before. Pages 8 and 9 will
now be uppermost, and will merely require folding together to make the
pages of the section follow in their proper order. If the folding has
been done carefully, and the ``register'' of the printing is good, the
headlines should be exactly even throughout.

\href{images/gs040.jpg}{\includegraphics{images/gs040_th.jpg}}\protect\hypertarget{Fig_3}{}{}

{Fig.} 3.

The object of cutting past the centre at each fold is to avoid the
unsightly creasing that results from folding two or more thicknesses of
paper when joined at the top edge.

A ``duodecimo'' sheet has the pages arranged as at
\protect\hyperlink{Fig_4}{fig. 4}.

The ``inset'' pages, 10, 15, 14, 11, must be cut off, and the rest of
the section folded as for an octavo sheet. The inset is folded
separately and inserted into the centre of the octavo portion.

Other sizes are folded in much the
same{\protect\hypertarget{Page_41}{}{{[}41{]}}} way, and the principle
of folding one sheet having been mastered, no difficulty will be found
in folding any other.

Plates often require trimming, and this must be done with judgment. The
plates should be trimmed to correspond as far as possible with the
printing on the opposite{\protect\hypertarget{Page_42}{}{{[}42{]}}}
page, but if this cannot be done, it is desirable that something
approaching the proportion of margin shown at
\protect\hyperlink{Fig_2}{fig. 2} (folio) should be aimed at. That is to
say, the back margin should be the smallest, the head margin the next,
the fore-edge a little wider, and the tail widest of all. When a plate
consists of a small portrait or diagram in the centre of the page, it
looks better if it is put a little higher and a little nearer the back
than the actual centre.

\href{images/gs042.jpg}{\includegraphics{images/gs042_th.jpg}}\protect\hypertarget{Fig_4}{}{}

{Fig.} 4.

Plates that have no numbers on them must be put in order by the list of
printed plates, or ``instructions to the binder.'' The half-title,
title, dedication, \&c., will often be found to be printed on odd sheets
that have to be made up into section A. This preliminary matter is
usually placed in the following order: Half-title, title, dedication,
preface, contents, list of illustrations or other lists. If there is an
index, it should be put at the end of the book.

All plates should be ``guarded,'' and any ``quarter sections,'' that is,
sections consisting of two leaves, should have their backs strengthened
by a ``guard,'' or they may very easily be torn in the sewing. Odd,
single leaves may be guarded round sections in the same way as
plates.{\protect\hypertarget{Page_43}{}{{[}43{]}}}

When a book has been folded, it should be pressed (see p.
\protect\hyperlink{Page_87}{87}).

There will sometimes be pages marked by the printer with a star. These
have some error in them, and are intended to be cut out. The printer
should supply corrected pages to replace them.

\hypertarget{collating}{%
\subparagraph{COLLATING}\label{collating}}

In addition to the pagination each sheet or section of a printed book is
lettered or numbered. Each letter or number is called the ``sheet's
signature.'' Printers usually leave out J W and V in lettering sheets.
If there are more sections than there are letters in the alphabet, the
printer doubles the letters, signing the sections A A, B B, and so on,
after the single letters are exhausted. Some printers use an Arabic
numeral before the section number to denote the second alphabet, as 2A,
2B, \&c., and others change the character of the letters, perhaps using
capitals for the first alphabet and italics for the second. If the
sheets are numbered, the numbers will of course follow consecutively. In
books of more than one volume, the number of the volume is sometimes
added in{\protect\hypertarget{Page_44}{}{{[}44{]}}} Roman numerals
before the signature, as II A, II B.

The main pagination of the book usually commences with Chapter I., and
all before that is independently paged in Roman numerals. It is unusual
to have actual numbers on the title or half-title, but if the pages are
counted back from where the first numeral occurs, they should come
right.

There will sometimes be one or more blank leaves completing sections at
the beginning or end. Such blank leaves must be retained, as without
them the volume would be ``imperfect.''

To collate a modern book the paging must be examined to see that the
leaves are in order, and that nothing is defective or missing.

The method of doing this is to insert the first finger of the right hand
at the bottom of about the fiftieth page, crook the finger, and turn up
the corners of the pages with it. When this is done the thumb is placed
on page 1, and the hand twisted, so as to fan out the top of the pages.
They can then be readily turned over by the thumb and first finger of
the left hand (see \protect\hyperlink{Fig_5}{fig. 5}). This is repeated
throughout the book,{\protect\hypertarget{Page_45}{}{{[}45{]}}} taking
about fifty pages at a time. It will of course only be necessary to
check the odd numbers, as if they are right, the even ones on the other
side of the leaf must be so. If the pages are numbered at the foot, the
leaves must be fanned out from the head.

\protect\hypertarget{Fig_5}{}{}
\includegraphics[width=3.125in,height=3.17708in]{images/gs046.jpg}

Fig. 5.

Plates or maps that are not paged can only be checked from the printed
list. When checked it will save time if
the{\protect\hypertarget{Page_46}{}{{[}46{]}}} number of the page which
each faces is marked on the back in small pencil figures.

In the case of early printed books or manuscripts, which are often not
paged, special knowledge is needed for their collation. It may roughly
be said, that if the sections are all complete, that is, if there are
the same number of leaves at each side of the sewing in all the
sections, the book may be taken to be perfect, unless of course whole
sections are missing. All unpaged books should be paged through in
pencil before they are taken apart; this is best done with a very fine
pencil, at the bottom left-hand corner; it will only be necessary to
number the front of each leaf.

\hypertarget{pulling-to-pieces}{%
\subparagraph{PULLING TO PIECES}\label{pulling-to-pieces}}

After the volume has been collated it must be ``pulled,'' that is to
say, the sections must be separated, and all plates or maps detached.

If in a bound book there are slips laced in the front cover, they must
be cut and the back torn off. It will sometimes happen that in tearing
off the leather nearly all the glue will come too,
leaving{\protect\hypertarget{Page_47}{}{{[}47{]}}} the backs of the
pages detached except for the sewing. More usually the back will be left
covered with a mass of glue and linen, or paper, which it is very
difficult to remove without injury to the backs of the sections. By
drawing a sharp knife along the bands, the sewing may be cut and the
bands removed, leaving the sections only connected by the glue. Then the
sections of the book can usually be separated with a fine folder, after
the thread from the centre of each has been removed; the point of
division being ascertained by finding the first signature of each
section. In cases where the glue and leather form too hard a back to
yield to this method, it is advisable to soak the glue with paste, and
when soft to scrape it off with a folder. As this method is apt to
injure the backs of the sections, it should not be resorted to unless
necessary; and when it is, care must be taken not to let the damp
penetrate into the book, or it will cause very ugly stains. The book
must be pulled while damp, or else the glue will dry up harder than
before. The separated sections must be piled up carefully to prevent
pages being soiled by the damp
glue.{\protect\hypertarget{Page_48}{}{{[}48{]}}}

All plates or single leaves ``pasted on'' must be removed. These can
usually be detached by carefully tearing apart, but if too securely
pasted they must be soaked off in water, unless of course the plates
have been painted with water-colour. If the plates must be soaked off,
the leaf and attached plate should be put into a pan of slightly warm
water and left to soak until they float apart, then with a soft brush
any remaining glue or paste can be easily removed while in the water.
Care must be taken not to soak modern books printed on what is called
``Art Paper,'' as this paper will hardly stand ordinary handling, and is
absolutely ruined if wetted. The growing use of this paper in important
books is one of the greatest troubles the bookbinder has to face. The
highly loaded and glazed surface of some of the heavy plate papers
easily flakes off, so that any guard pasted on these plates is apt to
come away, taking with it the surface of the paper. Moreover, should the
plates chance to be fingered or in any way soiled, nothing can remove
the marks; and should a corner get turned down, the paper breaks and the
corner will fall off. It is the opinion of experts that
this{\protect\hypertarget{Page_49}{}{{[}49{]}}} heavily loaded Art Paper
will not last a reasonable time, and, apart from other considerations,
this should be ample reason for not using it in books that are expected
to have a permanent value. Printers like this paper, because it enables
them to obtain brilliant impressions from blocks produced by cheap
processes.

In ``cased'' books, sewn by machinery, the head and tail of the sheets
will often be found to be split up as far as the ``kettle'' stitches. If
such a book is to be expensively bound, it will require mending
throughout in these places, or the glue may soak into the torn ends, and
make the book open stiffly.

Some books are put together with staples of tinned iron wire, which
rapidly rust and disfigure the book by circular brown marks. Such marks
will usually have to be cut out and the places carefully mended. This
process is lengthy, and consequently so costly, that it is generally
cheaper, when possible, to obtain an unbound copy of the book from the
publishers, than to waste time repairing the damage done by the cloth
binder.

Generally speaking, the sections of a book cased in cloth by modern
methods are so injured as to make it unfit for
more{\protect\hypertarget{Page_50}{}{{[}50{]}}} permanent binding unless
an unreasonable amount of time is spent on it. It is a great pity that
publishers do not, in the case of books expected to have a permanent
literary value, issue a certain number of copies printed on good paper,
and unbound, for the use of those who require permanent bindings; and in
such copies it would be a great help if sufficient margin were left at
the back of the plates for the binder to turn it up to form a guard. If
the plates were very numerous, guards made of the substance of the
plates themselves would make the book too thick; but in the case of
books with not more than a dozen plates, printed on comparatively thin
paper, it would be a great advantage.

Some books in which there are a large number of plates are cut into
single leaves, which are held together at the back by a coating of an
indiarubber solution. For a short time such a volume is pleasant enough
to handle, and opens freely, but before long the indiarubber perishes,
and the leaves and plates fall apart. When a book of this kind comes to
have a permanent binding, all the leaves and plates have to be pared at
the back and made up into sections with guards---a troublesome and
expensive{\protect\hypertarget{Page_51}{}{{[}51{]}}} business. The
custom with binders is to overcast the backs of the leaves in sections,
and to sew through the overcasting thread, but this, though an easy and
quick process, makes a hopelessly stiff back, and no book so treated can
open freely.

\hypertarget{refolding}{%
\subparagraph{REFOLDING}\label{refolding}}

\protect\hypertarget{Fig_6}{}{}
\includegraphics[width=1.67708in,height=3.125in]{images/gs052.jpg}

Fig. 6.---Dividers

When the sheets of books that have to be rebound have been carelessly
folded, a certain amount of readjustment is often advisable, especially
in cases where the book has not been previously cut. The title-page and
the half-title, when found to be out of square, should nearly always be
put straight. The folding of the whole book may be corrected by taking
each pair of leaves and holding them up to the light and adjusting the
fold so that the print on one leaf comes exactly over the print on
the{\protect\hypertarget{Page_52}{}{{[}52{]}}} other, and creasing the
fold to make them stay in that position. With a pair of dividers
(\protect\hyperlink{Fig_6}{fig. 6}) set to the height of the shortest
top margin, points the same distance above the headline of the other
leaves can be made. Then against a carpenter's square, adjusted to the
back of the fold, the head of one pair of leaves at a time can be cut
square (see \protect\hyperlink{Fig_7}{fig. 7}). If the book has been
previously cut this process is apt to throw the leaves so far out of
their original position as to make them unduly uneven.

Accurate folding is impossible if the ``register'' of the printing is
bad, that is to say, if the print on the back of a leaf does not lie
exactly over that on the front.

Crooked plates should usually be made straight by judicious trimming of
the margins. It is better to leave a plate short
at{\protect\hypertarget{Page_53}{}{{[}53{]}}} tail or fore-edge than to
leave it out of square.

\protect\hypertarget{Fig_7}{}{}
\includegraphics[width=3.125in,height=1.47917in]{images/gs053.jpg}

Fig. 7.

\hypertarget{knocking-out-joints}{%
\subparagraph{KNOCKING OUT JOINTS}\label{knocking-out-joints}}

The old ``joints'' must be knocked out of the sections of books that
have been previously backed. To do this, one or two sections at a time
are held firmly in the left hand, and well hammered on the knocking-down
iron fixed into the lying press. It is important that the hammer face
should fall exactly squarely upon the paper, or it may cut pieces out.
The knocking-down iron should be covered with a piece of paper, and the
hammer face must be perfectly clean, or the sheets may be soiled.

\hypertarget{chapter-iii}{%
\subsection[CHAPTER
III]{\texorpdfstring{\protect\hypertarget{CHAPTER_III}{}{}CHAPTER
III}{CHAPTER III}}\label{chapter-iii}}

Guarding---Throwing Out---Paring Paper---Soaking off India
Proofs---Mounting very Thin Paper---Splitting
Paper---Inlaying---Flattening Vellum

\hypertarget{guarding}{%
\subparagraph{GUARDING}\label{guarding}}

{Guards} are slips of thin paper or linen used for strengthening the
fold of leaves that are damaged, or for attaching plates or single
leaves.

\protect\hypertarget{Fig_8}{}{}
\includegraphics[width=2.08333in,height=1.14583in]{images/gs055.jpg}

Fig. 8.

Guards should be of good thin paper. That known as Whatman's Banknote
paper{\protect\hypertarget{Page_54}{}{{[}54{]}}} answers very well. An
easy way to cut guards is shown in \protect\hyperlink{Fig_8}{fig. 8}.
Two or three pieces of paper of the height of the required guards are
folded and pinned to the board by the right-hand corners. A series of
points are marked at the head and tail with dividers set to the width
desired for the guards, and with a knife guided by a straight-edge, cuts
joining the points are made right through the paper, but not extending
quite to either end. On a transverse cut being made near the bottom, the
guards are left attached by one end only (see
\protect\hyperlink{Fig_9}{fig. 9}), and can be torn off as wanted. This
method prevents the paper from slipping while it is being cut.

\protect\hypertarget{Fig_9}{}{}
\includegraphics[width=2.08333in,height=1.15625in]{images/gs055a.jpg}

Fig. 9.

A mount cutter's knife (\protect\hyperlink{Fig_10}{fig. 10}) will be
found to be a convenient form of knife to use for cutting
guards.{\protect\hypertarget{Page_55}{}{{[}55{]}}}

In using the knife and straight-edge a good deal of pressure should be
put on the straight-edge, and comparatively little on the knife.

\protect\hypertarget{Fig_10}{}{}
\includegraphics[width=3.125in,height=0.36458in]{images/gs056.jpg}

Fig. 10.---Mount Cutter's Knife

To mend the torn back of a pair of leaves, a guard should be selected a
little longer than the height of the pages and well pasted with white
paste (see page \protect\hyperlink{Page_288}{288}). If the pair of
leaves are not quite separated, the pasted guard held by its extremities
may be simply laid along the weak place and rubbed down through
blotting-paper. If the leaves are quite apart, it is better to lay the
pasted guard on a piece of glass and put the edges of first one and then
the other leaf on to it and rub down.

On an outside pair of leaves the guard should be inside, so that the
glue may catch any ragged edges; while on the inside pair the guard
should be outside, or it will be found to be troublesome in sewing. In
handling the pasted guards care is needed not to stretch them, or they
may cause the sheet to crinkle as they
dry.{\protect\hypertarget{Page_56}{}{{[}56{]}}}

\protect\hypertarget{Fig_11}{}{}
\includegraphics[width=3.125in,height=1.26042in]{images/gs057.jpg}

Fig. 11.

Plates must be guarded round the sections next them. When there are a
great many plates the back margin of each, to which a guard will be
attached, must be pared (see \protect\hyperlink{Fig_11}{fig. 11}, A), or
the additional thickness caused by the guards will make the back swell
unduly. In guarding plates a number can be pasted at once if they are
laid one on another, with about an eighth of an inch of the back of each
exposed, the top of the pile being protected by a folded piece of waste
paper (see \protect\hyperlink{Fig_12}{fig. 12}). To paste, the brush is
brought from the top to the bottom of the pile only, and not the other
way, or paste will get between the plates and soil them. Guards should
usually be attached to the backs of plates, and should be wide enough to
turn up round the adjoining section, so that
they{\protect\hypertarget{Page_57}{}{{[}57{]}}} may be sewn through.
Should a plate come in the middle of a section, the guard is best turned
back and slightly pasted to the inside of the sheet and then sewn
through in the ordinary way.

\protect\hypertarget{Fig_12}{}{}
\includegraphics[width=3.125in,height=1.58333in]{images/gs058.jpg}

Fig. 12.

If plates are very thick, they must be hinged, as shown at
\protect\hyperlink{Fig_11}{fig. 11}, B. This is done by cutting a strip
of about a quarter of an inch off the back of the plate, and guarding
with a wide guard of linen, leaving a small space between the plate and
the piece cut off to form a hinge. It will save some swelling if the
plate is pared and a piece of thinner paper substituted for the piece
cut off (see \protect\hyperlink{Fig_11}{fig. 11}, C). If the plates are
of cardboard, they should be guarded on both sides with linen, and may
even need a second joint.

A book that consists entirely of
plates{\protect\hypertarget{Page_58}{}{{[}58{]}}} or single leaves must
be made up into sections with guards, and sewn as usual. In books in
which there are a great many plates, it is often found that two plates
either come together in the centre of a section, or come at opposite
sides of the same pair of leaves. Such plates should be guarded together
and treated as folded sheets (see \protect\hyperlink{Fig_13}{fig. 13}).

\protect\hypertarget{Fig_13}{}{}
\includegraphics[width=3.125in,height=0.57292in]{images/gs059.jpg}

Fig. 13.

In order to be sure that the pages of a book to be guarded throughout
will come in their proper order, it is well to make a plan of the
sections as follows, and to check each pair of leaves by it, as they are
guarded:---

Thus, if the book is to be made up into sections of eight leaves, the
pairs of leaves to be guarded together can be seen at once if the number
of the pages are written out---

\begin{itemize}
\tightlist
\item
  1, 3, 5, 7,---9, 11, 13, 15.
\end{itemize}

First the inside pair, 7 and 9, are guarded together with the guard
outside, then the next pair, 5 and 11, then 3 and 13,
and{\protect\hypertarget{Page_59}{}{{[}59{]}}} then the outside pair, 1
and 15, which should have the guard outside. A plan for the whole book
would be more conveniently written thus---

\begin{itemize}
\tightlist
\item
  1-15{17-31}{33-47}
\item
  3-13{19-29}{35-45}
\item
  5-11{21-27}{37-43}
\item
  7-9{23-25}{39-41, and so on.}
\end{itemize}

To arrange a book of single leaves for guarding, it is convenient to
take as many leaves as you intend to go to a section, and opening them
in the centre, take a pair at a time as they come.

The number of leaves it is advisable to put into a section will depend
on the thickness of the paper and the size and thickness of the book. If
the paper is thick, and the backs of the leaves have been pared, four
leaves to a section will be found to answer. But if the paper is thin,
and does not allow of much paring, it is better to have a larger
section, in order to have as little thread in the back as possible.

The sheets of any guarded book should be pressed before sewing, in order
to reduce the swelling of the back caused by the
guards.{\protect\hypertarget{Page_60}{}{{[}60{]}}}

\hypertarget{throwing-out}{%
\subparagraph{THROWING OUT}\label{throwing-out}}

\protect\hypertarget{Fig_14}{}{}
\includegraphics[width=1.4375in,height=3.64583in]{images/gs061.jpg}

Fig. 14.

Maps or diagrams that are frequently referred to in the text of a book,
should be ``thrown out'' on a guard as wide as the sheet of the book.
Such maps, \&c., should be placed at the end, so that they may lie open
for reference while the book is being read (see
\protect\hyperlink{Fig_14}{fig. 14}). Large folded maps or diagrams
should be mounted on linen. To do this take a piece of jaconet and pin
it out flat on the board,{\protect\hypertarget{Page_61}{}{{[}61{]}}}
then evenly paste the back of the map with thin paste in which there are
no lumps, and lay it on the linen, rub down through blotting-paper, and
leave to dry. Unless the pasting is done evenly the marks of the
paste-brush will show through the linen. If a folded map is printed on
very thick paper each fold must be cut up, and the separate pieces
mounted on the linen, with a slight space between them to form a
flexible joint.

\protect\hypertarget{Fig_15}{}{}
\includegraphics[width=3.125in,height=1.35417in]{images/gs062.jpg}

Fig. 15.

A folded map must have in the back of the book sufficient guards to
equal it in thickness at its thickest part when folded, or the book will
not shut properly (see \protect\hyperlink{Fig_15}{fig. 15}).

\hypertarget{paring-paper}{%
\subparagraph{PARING PAPER}\label{paring-paper}}

For paring the edge of paper for mending or guarding, take a very sharp
knife,{\protect\hypertarget{Page_62}{}{{[}62{]}}} and holding the blade
at right angles to the covering-board, draw the edge once or twice along
it from left to right. This should turn up enough of the edge to form a
``burr,'' which causes the knife to cut while being held almost flat on
the paper. The plate or paper should be laid face downwards on the glass
with the edge to be pared away from the workman, the knife held in the
right hand, with the burr downwards. The angle at which to hold the
knife will depend on its shape and on the thickness and character of the
paper to be pared, and can only be learned by practice. If the knife is
in order, and is held at the proper angle, the shaving removed from a
straight edge of paper should come off in a long spiral. If the knife is
not in proper order, the paper may be badly jagged or creased.

\hypertarget{soaking-off-india-proofs}{%
\subparagraph{SOAKING OFF INDIA PROOFS}\label{soaking-off-india-proofs}}

Place a piece of well-sized paper in a pan of warm water, then lay the
mounted India proof, face downwards, upon it and leave it to soak until
the proof floats off. Then carefully take out the old mount, and the
India proof can be readily
removed{\protect\hypertarget{Page_63}{}{{[}63{]}}} from the water on the
under paper, and dried between sheets of blotting-paper.

\hypertarget{mounting-very-thin-paper}{%
\subparagraph{MOUNTING VERY THIN PAPER}\label{mounting-very-thin-paper}}

Very thin paper, such as that of some ``India'' proofs, may be safely
mounted as follows:---The mount, ready for use, is laid on a pad of
blotting-paper. The thin paper to be mounted is laid face downwards on a
piece of glass and very carefully pasted with thin, white paste. Any
paste on the glass beyond the edges of the paper is carefully wiped off
with a clean cloth. The glass may then be turned over, and the pasted
plate laid on the mount, its exact position being seen through the
glass.

\hypertarget{splitting-paper}{%
\subparagraph{SPLITTING PAPER}\label{splitting-paper}}

It is sometimes desirable to split pieces of paper when the matter on
one side only is needed, or when the matter printed on each side is to
be used in different places. The paper to be split should be well pasted
on both sides with a thickish paste, and fine linen or jaconet placed on
each side. It is then nipped in the press
to{\protect\hypertarget{Page_64}{}{{[}64{]}}} make the linen stick all
over, and left to dry.

If the two pieces of jaconet are carefully pulled apart when dry, half
the paper should be attached to each, unless at any point the paste has
failed to stick, when the paper will tear. The jaconet and paper
attached must be put into warm water until the split paper floats off.

\hypertarget{inlaying-leaves-or-plates}{%
\subparagraph{INLAYING LEAVES OR
PLATES}\label{inlaying-leaves-or-plates}}

\protect\hypertarget{Fig_16}{}{}
\includegraphics[width=1.04167in,height=1.34375in]{images/gs065.jpg}

Fig. 16.

When a small plate or leaf has to be inserted into a larger book, it is
best to ``inlay it''; that is to say, the plate or leaf is let into a
sheet of paper the size of the page of the book. To do this, a piece of
paper as thick as the plate to be inlaid, or a little thicker, is
selected, and on this is laid the plate, which should have been
previously squared, and the positions of the corners marked with a
folder. A point is made about an eighth of an inch inside each corner
mark, and the paper within these points is cut out (see
\protect\hyperlink{Fig_16}{fig. 16}). This leaves a frame of paper,
the{\protect\hypertarget{Page_65}{}{{[}65{]}}} inner edges of which will
slightly overlap the edges of the plate. The under edge of the plate,
and the upper edge of the mount, should then be pared and pasted, and
the plate laid in its place (with the corners corresponding to the
folder marks). If the edges have been properly pared, the thickness
where they overlap should not exceed the thickness of the frame paper.
If an irregular fragment is to be inlaid, it is done in the same way,
except that the entire outline is traced on the new paper with a folder,
and the paper cut away, allowing one eighth of an inch inside the
indented line.

\hypertarget{flattening-vellum}{%
\subparagraph{FLATTENING VELLUM}\label{flattening-vellum}}

The leaves of a vellum book that have become cockled from damp or other
causes may be flattened by damping them, pulling them out straight, and
allowing them to dry under pressure. To do this take the book to pieces,
clean out any dirt there may be in the folds of the leaves, and spread
out each pair of leaves as flatly as possible.

Damp some white blotting-paper by interleaving it with common white
paper{\protect\hypertarget{Page_66}{}{{[}66{]}}} that has been wetted
with a sponge. One sheet of wet paper to two of blotting-paper will be
enough. The pile of blotting-paper and wet paper is put in the press and
left for an hour or two under pressure, then taken out and the common
paper removed.

The blotting-paper should now be slightly and evenly damp. To flatten
the vellum the open pairs of leaves are interleaved with the slightly
damp blotting-paper, and are left for an hour under the weight of a
pressing-board. After this time the vellum will have become quite soft,
and can with care be flattened out and lightly pressed between the
blotting-paper, and left for a night. The next day the vellum leaves
should be looked at to see that they lie quite flat, and the
blotting-paper changed for some that is dry. The vellum must remain
under pressure until it is quite dry, or it will cockle up worse than
ever when exposed to the air. The blotting-paper should be changed every
day or two. The length of time that vellum leaves take to dry will vary
with the state of the atmosphere, and the thickness of the vellum, from
one to six weeks.{\protect\hypertarget{Page_67}{}{{[}67{]}}}

Almost any manuscript or printed book on vellum can be successfully
flattened in this way; miniatures should have pieces of waxed paper laid
over them to prevent the chance of any of the fibres of the
blotting-paper sticking. The pressure must not be great; only enough is
needed to keep the vellum flat as it dries.

This process of flattening, although so simple, requires the utmost
care. If the blotting-paper is used too damp, a manuscript may be
ruined; and if not damp enough, the pressing will have no effect.

\hypertarget{chapter-iv}{%
\subsection[CHAPTER
IV]{\texorpdfstring{\protect\hypertarget{CHAPTER_IV}{}{}CHAPTER
IV}{CHAPTER IV}}\label{chapter-iv}}

Sizing---Washing---Mending

\hypertarget{sizing}{%
\subparagraph{SIZING}\label{sizing}}

{The} paper in old books is sometimes soft and woolly. This is generally
because the size has perished, and such paper can often be made
perfectly sound by resizing.

\protect\hypertarget{Fig_17}{}{}
\includegraphics[width=3.125in,height=1.34375in]{images/gs069.jpg}

Fig. 17.

For size, an ounce of isinglass or good gelatine is dissolved in a quart
of water. This should make a clear solution
when{\protect\hypertarget{Page_68}{}{{[}68{]}}} gently warmed, and
should be used at about a temperature of 120° F. Care must be taken not
to heat too quickly, or the solution may burn and turn brown. If the
size is not quite clear, it should be strained through fine muslin or
linen before being used. When it is ready it should be poured into an
open pan (\protect\hyperlink{Fig_17}{fig. 17}), so arranged that it can
be kept warm by a gas flame or spirit lamp underneath. When this is
ready the sheets to be sized can be put in one after another and taken
out at once. The hot size will be found to take out a great many stains,
and especially those deep brown stains that come from water. If there
are only a few sheets, they can be placed between blotting-paper as they
are removed from the size; but if there is a whole book, it is best to
lay them in a pile one on the other,
and{\protect\hypertarget{Page_69}{}{{[}69{]}}} when all have been sized
to squeeze them in the ``lying press'' between pressing-boards, a pan
being put underneath to catch the liquid squeezed out. When the sheets
have been squeezed they can be readily handled, and should be spread out
to dry on a table upon clean paper. When they are getting dry and firm
they can be hung on strings stretched across the room, slightly
overlapping one another. The strings must first be covered with slips of
clean paper, and the sized sheets should have more paper over them to
keep them clean.

Before sizing it will be necessary to go through a book and take out any
pencil or dust marks that can be removed with indiarubber or bread
crumbs, or the size will fix them, and it will be found exceedingly
difficult to remove them afterwards.

When the sheets are dry they should be carefully mended in any places
that may be torn, and folded up into sections and pressed. A long,
comparatively light pressure will be found to flatten them better and
with less injury to the surface of the paper than a short, very heavy
pressure, such as that of the rolling-machine.

In some cases it will be found
that{\protect\hypertarget{Page_70}{}{{[}70{]}}} sheets of old books are
so far damaged as to be hardly strong enough to handle. Such sheets must
be sized in rather a stronger size in the following way:---Take a sheet
of heavily-sized paper, such as notepaper, and carefully lay your
damaged sheet on that. Then put another sheet of strong paper on the
top, and put all three sheets into the size. It will be found that the
top sheet can then be easily lifted off, and the size be made to flow
over the face of the damaged sheet. Then, if the top sheet be put on
again, the three sheets, if handled as one, can be turned over and the
operation repeated, and size induced to cover the back of the damaged
leaf. The three sheets must then be taken out and laid between
blotting-paper to take up the surplus moisture. The top sheet must then
be carefully peeled off, and the damaged page laid face downwards on
clean blotting-paper. Then the back sheet can be peeled off as well,
leaving the damaged sheet to dry.

The following is quoted from ``Chambers' Encyclopædia'' on Gelatine:---

``Gelatine should never be judged by the eye
alone.{\protect\hypertarget{Page_71}{}{{[}71{]}}}

``Its purity may be very easily tested thus: Soak it in cold water, then
pour upon it a small quantity of boiling water. If pure, it will form a
thickish, clear straw-coloured solution, free from smell; but if made of
impure materials, it will give off a very offensive odour, and have a
yellow, gluey consistency.''

\hypertarget{washing}{%
\subparagraph{WASHING}\label{washing}}

When there are stains or ink marks on books that cannot be removed by
the use of hot size or hot water, stronger measures may sometimes have
to be taken. Many stains will be found to yield readily to hot water
with a little alum in it, and others can be got out by a judicious
application of curd soap with a very soft brush and plenty of warm
water. But some, and especially ink stains, require further treatment.
There are many ways of washing paper, and most of those in common use
are extremely dangerous, and have in many cases resulted in the absolute
destruction of fine books. If it is thought to be absolutely necessary
that the sheets of a book should be washed, the safest method is as
follows:---Take an ounce of permanganate of potash
dissolved{\protect\hypertarget{Page_72}{}{{[}72{]}}} in a quart of
water, and warmed slightly. In this put the sheets to be washed, and
leave them until they turn a dark brown. This will usually take about an
hour, but may take longer for some papers. Then turn the sheets out and
wash them in running water until all trace of purple stain disappears
from the water as it comes away. Then transfer them to a bath of
sulphurous (not sulphuric) acid and water in the proportion of one ounce
of acid to one pint of water. The sheets in this solution will rapidly
turn white, and if left for some time nearly all stains will be removed.
In case any stains refuse to come out, the sheets should be put in clear
water for a short time, and then placed in the permanganate of potash
solution again, and left there for a longer time than before; then after
washing in clear water, again transferred to the sulphurous acid. When
sheets are removed from the sulphurous acid they should be well washed
for an hour or two in running water, and then may be blotted or squeezed
off and hung up on lines to dry. Any sheets treated in this way will
require sizing afterwards. And if, as is often the case, only a few
sheets at the beginning or
end{\protect\hypertarget{Page_73}{}{{[}73{]}}} of the book have to be
washed, it will be necessary to tone down the washed sheets to match the
rest of the book by putting some stain in the size. For staining there
are many things used. A weak solution of permanganate of potash gives a
yellowish stain that will be found to match many papers. Other stains
are used, such as coffee, chicory, tea, liquorice, \&c. Whatever is used
should be put in the size. To ascertain that the right depth of colour
has been obtained, a piece of unsized paper, such as white
blotting-paper, is dipped in the stained size and blotted off and dried
before the fire. It is impossible to judge of the depth of colour in a
stain unless the test piece is thoroughly dried. If the stain is not
right, add more water or more stain as is needed. Experience will tell
what stain to use to match the paper of any given book.

To remove grease or oil stains, ether may be used. Pour it freely in a
circle round the spot, narrowing the circle gradually until the stain is
covered. Then apply a warm iron through a piece of blotting-paper.

Ether should only be used in a draught in a well-ventilated room on
account of{\protect\hypertarget{Page_74}{}{{[}74{]}}} its well-known
inflammable and anæsthetical properties.

A very dilute (about one per cent.) solution of pure hydrochloric acid
in cold water will be found to take out some stains if the paper is left
in it for some hours. When the paper is removed from the solution, it
must be thoroughly washed in running water. It is important that the
hydrochloric acid used should be pure, as the commercial quality
(spirits of salts) often contains sulphuric acid.

The following recipes are quoted from \emph{De l'organisation et de
l'administration des Bibliothèques, par Jules Cusin}:---

To remove stains from paper:---``\emph{Mud Stains.}---To take away these
kinds of stains, spread some soap jelly very evenly over the stained
places, and leave it there for thirty or forty minutes, according to the
depth of the stain. Then dip the sheet in clean water, and then having
spread it on a perfectly clean table, remove the soap lightly with a
hog's hair brush or a fine sponge; all the mud will disappear at the
same time. Put the sheet into the clear water again, to get rid of the
last trace of soap. Let it drain a little, press it lightly between two
sheets of blotting-paper,{\protect\hypertarget{Page_75}{}{{[}75{]}}} and
finish by letting it dry slowly in a dry place in the shade.

``\emph{Stains of Tallow, Stearine, or Fat.}---To take away these stains
cover them with blotting-paper and pass over them a warm flat-iron. When
the paper has soaked up the grease, change it and repeat the operation
until the stains have been sufficiently removed. After that, touch both
sides of the sheets where they have been stained with a brush dipped in
essence of turpentine heated to boiling-point. Then to restore the
whiteness of the paper, touch the places which were stained with a piece
of fine linen soaked in purified spirits of wine warmed in the
water-bath. This method may also be employed to get rid of sealing-wax
stains.

``\emph{Oil Stains.}---Make a mixture of 500 gr. of soap, 300 gr. of
clay, 60 gr. of quicklime, and sufficient water to make it of the right
consistency, spread a thin layer of this on the stain, and leave it
there about a quarter of an hour. Then dip the sheet in a bath of hot
water; take it out, and let it dry slowly.

``You can also use the following method, generally employed for
finger-marks:---

``\emph{Finger-marks.}---These stains are
sometimes{\protect\hypertarget{Page_76}{}{{[}76{]}}} very obstinate.
Still they can generally be mastered by the following method:---Spread
over them a layer of white soap jelly (\emph{savon blanc en gelée}), and
leave it there for some hours. Then remove this with a fine sponge
dipped in hot water, and more often than not all the dirt disappears at
the same time. If this treatment is not sufficient, you might replace
the soap jelly by soft soap (\emph{savon noir}), but you must be careful
not to leave it long on the printing, which might decompose and run, and
that would do more harm than good.''

Sheets of very old books are best left with the stains of age upon them,
excepting, perhaps, such as can be removed with hot water or size.
Nearly all stains \emph{can} be removed, but in the process old paper is
apt to lose more in character than it gains in appearance.

\hypertarget{mending}{%
\subparagraph{MENDING}\label{mending}}

For mending torn sheets of an old book, some paper that matches as
nearly as possible must be found. For this purpose it is the custom for
bookbinders to collect quantities of old paper. If a
piece{\protect\hypertarget{Page_77}{}{{[}77{]}}} of the same tone cannot
be found, paper of similar texture and substance may be stained to
match.

Supposing a corner to be missing, and a piece of paper to have been
found that matches it, the torn page is laid over the new paper in such
a way that the wire marks on both papers correspond. Then the point of a
folder should be drawn along the edge of the torn sheet, leaving an
indented line on the new paper. The new paper should then be cut off
about an eighth of an inch beyond the indented line, and the edge
carefully pared up to the line. The edge of the old paper must be
similarly pared, so that the two edges when laid together will not
exceed the thickness of the rest of the page. It is well to leave a
little greater overlap at the edges of the page. Both cut edges must
then be well pasted with white paste and rubbed down between
blotting-paper. To ensure a perfectly clean joint the pasted edge should
not be touched with the hand, and pasting-paper, brushes, and paste must
be perfectly clean.

In the case of a tear across the page, if there are any overlapping
edges, they may merely be pasted together and
the{\protect\hypertarget{Page_78}{}{{[}78{]}}} end of the tear at the
edge of the paper strengthened by a small piece of pared paper. If the
tear crosses print, and there are no overlapping edges, either tiny
pieces of pared paper may be cut and laid across the tear between the
lines of print, or else a piece of the thinnest Japanese paper, which is
nearly transparent, may be pasted right along the tear over the print;
in either case the mend should be strengthened at the edge of the page
by an additional thickness of paper. In cases where the backs of the
sections have been much damaged, it will be necessary to put a guard the
entire length, or in the case of small holes, to fill them in with
pieces of torn paper. The edges of any mend may, with great care, be
scraped with a sharp knife having a slight burr on the under side, and
then rubbed lightly with a piece of worn fine sand-paper, or a fragment
of cuttle-fish bone. Care must be taken not to pare away too much, and
especially not to weaken the mend at the edges of the sheet. As a
general rule, the new mending paper should go on the back of a sheet.

Sometimes it is thought necessary to fill up worm-holes in the paper.
This{\protect\hypertarget{Page_79}{}{{[}79{]}}} may be done by boiling
down some paper in size until it is of a pulpy consistency, and a little
of this filled into the worm-holes will re-make the paper in those
places. It is a very tedious operation, and seldom worth doing.

\protect\hypertarget{Fig_18}{}{}
\includegraphics[width=1.5625in,height=0.94792in]{images/gs080.jpg}

Fig. 18.

Mending vellum is done in much the same way as mending paper, excepting
that a little greater overlap must be left. It is well to put a stitch
of silk at each end of a vellum patch, as you cannot depend on paste
alone holding vellum securely. The overlapping edges must be well
roughed up with a knife to make sure that the paste will stick. A cut in
a vellum page is best mended with fine silk with a lacing stitch (see
\protect\hyperlink{Fig_18}{fig. 18}).

Mending is most easily done on a sheet of plate-glass, of which the
edges and corners have been rubbed down.

\hypertarget{chapter-v80}{%
\subsection[CHAPTER
V]{\texorpdfstring{\protect\hypertarget{CHAPTER_V}{}{}CHAPTER
V{\protect\hypertarget{Page_80}{}{{[}80{]}}}}{CHAPTER V{[}80{]}}}\label{chapter-v80}}

End Papers---Leather Joints---Pressing

\hypertarget{end-papers}{%
\subparagraph{END PAPERS}\label{end-papers}}

{If} an old book that has had much wear is examined, it will generally
be found that the leaves at the beginning and the end have suffered more
than the rest of the book. On this ground, and also to enable people who
must write notes in books to do so with the least injury to the book, it
is advisable to put a good number of blank papers at each end. As these
papers are part of the binding, and have an important protective
function to perform, they should be of good quality. At all times
difficulty has been found in preventing the first and last section of
the book, whether end papers or not, from dragging away when the cover
is opened, and various devices have been tried to overcome this defect.
In the fifteenth century strips of vellum (usually cut from manuscripts)
were pasted on to the back of the book and on the
inside{\protect\hypertarget{Page_81}{}{{[}81{]}}} of the boards, or in
some cases were merely folded round the first and last section and
pasted on to the covers. The modern, and far less efficient, practice is
to ``overcast'' the first and last sections. This is objectionable,
because it prevents the leaves from opening right to the back, and it
fails in the object aimed at, by merely transferring the strain to the
back of the overcast section.

In order to make provision for any strain there may be in opening the
cover, it is better to adopt some such arrangement as shown in
\protect\hyperlink{Fig_19}{fig. 19}. In this end paper the zigzag opens
slightly in response to any strain.

The way to make this end paper is to take a folded sheet of paper a
little larger than the book. Then with dividers mark two points an
eighth of an inch from the back for the fold, and paste your paste-down
paper, B B, up to these points (see \protect\hyperlink{Fig_19}{fig. 19},
II). When the paste is dry, fold back the sheet (A1) over the paste-down
paper, and A2 the reverse way, leaving the form seen in
\protect\hyperlink{Fig_19}{fig. 19}, III. A folded sheet of paper
similar to A is inserted at C (\protect\hyperlink{Fig_19}{fig. 19}, V,
H), and the sewing passes through this. When
the{\protect\hypertarget{Page_82}{}{{[}82{]}}} book is pasted down the
leaf A1 is torn off, and B1 pasted down on the
board.{\protect\hypertarget{Page_83}{}{{[}83{]}}} If marbled paper is
desired, the marble should be ``made,'' that is, pasted on to B1.

\href{images/gs083.jpg}{\includegraphics{images/gs083_th.jpg}}\protect\hypertarget{Fig_19}{}{}

{Fig.} 19.

There are considerable disadvantages in using marbled papers, as if they
are of thick enough paper to help the strength of the binding, the
``made'' sheet is very stiff, and in a small book is troublesome. On no
account should any marble paper be used, unless it is tough and durable.
The quality of the paper of which most marbled papers are made is so
poor, that it is unsuitable for use as end papers. For most books a
self-coloured paper of good quality answers well for the paste-down
sheets.

It is a mistake to leave end papers to be pasted on after the book has
been forwarded, as in that case they have little constructive value.
Every leaf of such an end paper as is described above will open right to
the back, and the zigzag allows play for the drag of the board.

Paper with a conventional pattern painted or printed on it may be used
for end papers. If such a design is simple, such as a sprig repeated all
over, or an arrangement of stars or dots, it may look very well; but
over elaborate end papers,{\protect\hypertarget{Page_84}{}{{[}84{]}}}
and especially those that aim at pictorial effect, are seldom
successful.

Ends may be made of thin vellum. If so, unless the board is very heavy,
it is best to have leather joints.

A single leaf of vellum (in the place of B1 and 2, II,
\protect\hyperlink{Fig_19}{fig. 19}) should have an edge turned up into
the zigzag with the leather joint, and sewn through. Vellum ends must
always be sewn, as it is not safe to rely upon paste to hold them. They
look well, and may be enriched by tooling. The disadvantage of vellum
is, that it has a tendency to curl up if subjected to heat, and when it
contracts it unduly draws the boards of the book. For large manuscripts,
or printed books on vellum, which are bound in wooden or other thick
boards and are clasped, thicker vellum may be used for the ends; that
with a slightly brown surface looks best. The part that will come into
the joint should be scraped thin with a knife, and a zigzag made of
Japanese paper.

Silk or other fine woven material may be used for ends. It is best used
with a leather joint, and may be stuck on to the first paper of the end
papers (B1, No. 2, \protect\hyperlink{Fig_19}{fig. 19}), and cut with
the book. The{\protect\hypertarget{Page_85}{}{{[}85{]}}} glaire of the
edge gilding will help to stop the edges fraying out. In attaching silk
to paper, thin glue is the best thing to use; the paper, not the silk,
being glued. Some little practice is needed to get sufficient glue on
the paper to make the silk stick all over, and yet not to soil it. When
the silk has been glued to the paper, it should be left under a light
weight to dry. If put in the press, the glue may be squeezed through and
the silk soiled.

If the silk is very thin, or delicate in colour, or if it seems likely
that it will fray out at the edges, it is better to turn the edges in
over a piece of paper cut a little smaller than the page of the book and
stick them down. This forms a pad, which may be attached to the first
leaf of the end papers; a similar pad may be made for filling in the
board.

Before using, the silk should be damped and ironed flat on the wrong
side.

Silk ends give a book a rich finish, but seldom look altogether
satisfactory. If the silk is merely stuck on to the first end paper, the
edges will generally fray out if the book is much used. If the edges are
turned in, an unpleasantly thick end is
made.{\protect\hypertarget{Page_86}{}{{[}86{]}}}

\hypertarget{leather-joints}{%
\subparagraph{LEATHER JOINTS}\label{leather-joints}}

Leather joints are pieces of thin leather that are used to cover the
joints on the inside (for paring, see page
\protect\hyperlink{Page_154}{154}). They add very little strength to the
book, but give a pleasant finish to the inside of the board.

If there are to be leather joints, the end papers are made up without A
1, and the edge of the leather pasted and inserted at D, with a piece of
common paper as a protection (see \protect\hyperlink{Fig_19}{fig. 19},
IV). When the paste is dry, the leather is folded over at E.

A piece of blotting-paper may be pasted on to the inside of the waste
leaf, leaving enough of it loose to go between the leather joint and the
first sheet of the end paper. This will avoid any chance of the leather
joint staining or marking the ends while the book is being bound. The
blotting-paper, of course, is taken out with the waste sheet before the
joint is pasted down.

Joints may also be made of linen or cloth inserted in the same way. A
cloth joint has greater strength than a leather one, as the latter has
to be very thin{\protect\hypertarget{Page_87}{}{{[}87{]}}} in order that
the board may shut properly.

With leather or cloth joints, the sewing should go through both E and F.

\hypertarget{pressing}{%
\subparagraph{PRESSING}\label{pressing}}

\protect\hypertarget{Fig_20}{}{}
\includegraphics[width=3.125in,height=1.4375in]{images/gs088.jpg}

Fig. 20.

\protect\hypertarget{Fig_21}{}{}
\includegraphics[width=2.3125in,height=4.6875in]{images/gs089.jpg}

Fig. 21.---Standing Press

While the end papers are being
made,{\protect\hypertarget{Page_88}{}{{[}88{]}}} the sections of the
book should be pressed. To do this a pressing-board is taken which is a
little larger than the book, and a tin, covered with common paper,
placed on that, then a few sections of the
book,{\protect\hypertarget{Page_89}{}{{[}89{]}}} then another tin
covered with paper, and then more sections, and so on, taking care that
the sections are exactly over one another (see
\protect\hyperlink{Fig_20}{fig. 20}). A second pressing-board having
been placed on the last tin,{\protect\hypertarget{Page_90}{}{{[}90{]}}}
the pile of sections, tins, and pressing-boards can be put into the
standing-press and left under pressure till next day. Newly printed
plates should be protected by thin tissue paper while being pressed. Any
folded plates or maps, \&c., or inserted letters, must either not be
pressed, or have tins placed on each side of them to prevent them from
indenting the adjoining leaves.

\protect\hypertarget{Fig_22}{}{}
\includegraphics[width=2.08333in,height=4.6875in]{images/gs090.jpg}

Fig. 22.---French Standing Press

Hand-printed books, such as the publications of the Kelmscott Press,
should have very little pressure, or the ``impression'' of the print and
the surface of the paper may be injured. Books newly printed on vellum
or heavily coloured illustrations should not be pressed at all, or the
print may ``set off.''

The protecting tissues on the plates of a book that has been printed for
more than a year can generally be left out, unless the titles of the
plates are printed on them, as they are a nuisance to readers and often
get crumpled up and mark the book.

In order to make books solid, that is, to make the leaves lie evenly and
closely to one another, it was formerly the custom to beat books on a
``stone'' with a heavy{\protect\hypertarget{Page_91}{}{{[}91{]}}}
hammer. This process has been superseded by the rolling-press; but with
the admirable presses that are now to be had, simple pressing will be
found to be sufficient for the ``extra'' binder.

At \protect\hyperlink{Fig_21}{fig. 21} is shown an iron standing-press.
This is screwed down first with a short bar, and finally with a long
bar. This form of press is effective and simple, but needs a good deal
of room for the long bar, and must have very firm supports, or it may be
pulled over.

At \protect\hyperlink{Fig_22}{fig. 22} is shown a French standing-press,
in which the pressure is applied by a weighted wheel, which will, in the
first place, by being spun round, turn the screw until it is tight, and
give additional pressure by a hammering action. This press I have found
to answer for all ordinary purposes, and to give as great pressure as
can be got by the iron standing-press, without any undue strain on
supports or workmen.

There are many other forms of press by which great pressure can be
applied, some working by various arrangements of cog-wheels, screws, and
levers, others by hydraulic pressure.

\hypertarget{chapter-vi92}{%
\subsection[CHAPTER
VI]{\texorpdfstring{\protect\hypertarget{CHAPTER_VI}{}{}CHAPTER
VI{\protect\hypertarget{Page_92}{}{{[}92{]}}}}{CHAPTER VI{[}92{]}}}\label{chapter-vi92}}

Trimming Edges before Sewing---Edge Gilding

\hypertarget{trimming-before-sewing}{%
\subparagraph{TRIMMING BEFORE SEWING}\label{trimming-before-sewing}}

{When} the sheets come from the press the treatment of the edges must be
decided upon, that is, whether they are to be entirely uncut, trimmed
before sewing, or cut in boards.

Early printed books and manuscripts should on no account have their
edges cut at all, and any modern books of value are better only slightly
trimmed and gilt before sewing. But for books of reference that need
good bindings, on account of the wear they have to withstand, cutting in
boards is best, as the smooth edge so obtained makes the leaves easier
to turn over. Gilt tops and rough edges give a book a look of unequal
finish.

If the edges are to remain uncut, or be cut ``in boards'' with the
plough, the book will be ready for ``marking up'' as soon as it comes
from the press; but if it is to be gilt before sewing, it must be first
trimmed.{\protect\hypertarget{Page_93}{}{{[}93{]}}}

\protect\hypertarget{Fig_23}{}{}
\includegraphics[width=1.48958in,height=2.08333in]{images/gs094.jpg}

Fig. 23.

The sheets for trimming with end papers and all plates inserted must
first be cut square at the head against a carpenter's square (see
\protect\hyperlink{Fig_7}{fig. 7}). Then a piece of mill-board may be
cut to the size, it is desired to leave the leaves, and the sections
trimmed to it. To do this three nails should be put into the covering
board through a piece of straw-board, and the back of the section slid
along nails 1 and 2 until it touches No. 3 (see
\protect\hyperlink{Fig_23}{fig. 23}). The board is slid in the same way,
and anything projecting beyond it cut off. When the under straw-board
has become inconveniently scored in the first position, by shifting the
lower nail (1) a fresh surface will receive the cuts. Fig. 24 is a
representation of a simple machine that I use in my workshop for
trimming. The slides A A are adjustable to any width required, and are
fixed by the screws B B.{\protect\hypertarget{Page_94}{}{{[}94{]}}} The
brass-bound straight edge C fits on to slots in A A, and as this, by the
adjustment of the slides, can be fixed at any distance from B B, all
sizes of books can be trimmed. As by this machine several sections can
be cut at once, the time taken is not very much greater than if the book
were cut in the plough.

\includegraphics[width=4.16667in,height=2.44792in]{images/gs095.jpg}

Fig. 24.

Considerable judgment is required in trimming. The edges of the larger
pages only, on a previously uncut book, should be cut, leaving the
smaller pages untouched. Such uncut pages are called ``proof,'' and the
existence of proof in a bound book is evidence that it has not been
unduly cut.

Before gilding the edges of the
trimmed{\protect\hypertarget{Page_95}{}{{[}95{]}}} sections, any uncut
folds that may remain should be opened with a folder, as if opened after
gilding, they will show a ragged white edge.

\protect\hypertarget{Fig_25}{}{}
\includegraphics[width=2.08333in,height=1.03125in]{images/gs096.jpg}

Fig. 25.

\hypertarget{edge-gilding}{%
\subparagraph{EDGE GILDING}\label{edge-gilding}}

To gild the edges of trimmed sections, the book must be ``knocked up''
to the fore-edge, getting as many of the short leaves as possible to the
front. It is then put into the ``lying press,'' with gilding boards on
each side (see \protect\hyperlink{Fig_25}{fig. No. 25}), and screwed up
tightly. Very little scraping will be necessary, and usually if well
rubbed with fine sand-paper, to remove any chance finger-marks or loose
fragments of paper, the edge will be smooth enough to gild. If the paper
is very absorbent, the edges must be washed over with vellum size and
left to dry.

The next process is an application of red chalk. For this a piece of
gilder's red chalk is rubbed down on a stone
with{\protect\hypertarget{Page_96}{}{{[}96{]}}} water, making a thickish
paste, and the edges are well brushed with a hard brush dipped in this
mixture, care being taken not to have it wet enough to run between the
leaves. Some gilders prefer to use blacklead or a mixture of chalk and
blacklead. A further brushing with a dry brush will to some extent
polish the leaves. It will then be ready for an application of glaire.
Before glairing, the gold must be cut on the cushion to the width
required (see p. \protect\hyperlink{Page_200}{200}), and may be either
taken up on very slightly greased paper, a gilder's tip, or with a piece
of net stretched on a little frame (see \protect\hyperlink{Fig_26}{fig.
26}). The gold leaf will adhere sufficiently to the net, and can be
readily released by a light breath when it is exactly over the proper
place on the edge.

When the gold is ready, the glaire should be floated on to the edge with
a soft brush, and the gold spread evenly over it and left until dry;
that is, in a workshop of ordinary temperature, for about an hour. The
edge is then lightly rubbed with a piece of leather that has been
previously rubbed on beeswax, and is ready for burnishing. It is best to
commence burnishing through a piece of
thin{\protect\hypertarget{Page_97}{}{{[}97{]}}} slightly waxed paper to
set the gold, and afterwards the burnisher can be used directly on the
edge. A piece of bloodstone ground so as to have no sharp edges (see
\protect\hyperlink{Fig_27}{fig. 27}) makes a good burnisher.

\protect\hypertarget{Fig_26}{}{}
\includegraphics[width=2.08333in,height=2.02083in]{images/gs098.jpg}

Fig. 26.

There are several different preparations used for gilding edges. One
part of beaten up white of egg with four parts of water left to stand
for a day and strained will be found to answer
well.{\protect\hypertarget{Page_98}{}{{[}98{]}}}

\protect\hypertarget{Fig_27}{}{}
\includegraphics[width=2.08333in,height=0.29167in]{images/gs098a.jpg}

Fig. 27.

After the fore-edge is gilt the same operation is repeated at the head
and tail. As it is desirable to have the gilding at the head as solid as
possible, rather more scraping is advisable here, or the head may be
left to be cut with a plough and gilt in boards.

\hypertarget{chapter-vii}{%
\subsection[CHAPTER
VII]{\texorpdfstring{\protect\hypertarget{CHAPTER_VII}{}{}CHAPTER
VII}{CHAPTER VII}}\label{chapter-vii}}

Marking up---Sewing---Materials for Sewing

\hypertarget{marking-up}{%
\subparagraph{MARKING UP}\label{marking-up}}

{This} is drawing lines across the back of the sections to show the
sewer the position of the sewing cords.

Marking up for flexible sewing needs care and judgment, as on it depends
the position of the bands on the back of the bound book. Nearly all
books look best with five bands, but very large, thinnish folios may
have six, and a very small, thick book may look better with four.
Generally speaking, five is the best number. In marking up trimmed
sheets for flexible sewing, the length of the back should be divided
from the head into six
portions,{\protect\hypertarget{Page_99}{}{{[}99{]}}} five equal, and one
at the tail slightly longer. From the points so arrived at, strong
pencil lines should be made across the back with a carpenter's square as
guide, the book having been previously knocked up between
pressing-boards, and placed in the lying press. It is important that the
head should be knocked up exactly square, as otherwise the bands will be
found to slope when the book is bound. In the case of a book which is to
be cut and gilt in boards, before marking up it will be necessary to
decide how much is to be cut off, and allowance made, or the head and
tail division of the back will, when cut, be too small. It must also be
remembered that to the height of the pages the amount of the ``squares''
will be added.

About a quarter of an inch from either end of the back of a trimmed
book, and a little more in the case of one that is to be cut in boards,
a mark should be made for the ``kettle'' or ``catch'' stitch. This may
be slightly sawn in, but before using the saw, the end papers are
removed. If these were sawn, the holes would show in the joint when the
ends are pasted down.

If the book is to be sewn on
double{\protect\hypertarget{Page_100}{}{{[}100{]}}} cords, or on slips
of vellum or tape, two lines will be necessary for each band.

It has become the custom to saw in the backs of books, and to sink the
bands into the saw cuts, using ``hollow backs,'' and putting false bands
to appear when bound. This is a degenerate form, to which is due much of
the want of durability of modern bindings. If the bands are not to show
on the back, it is better to sew on tapes or strips of vellum than to
use sawn-in string bands.

\hypertarget{sewing}{%
\subparagraph{SEWING}\label{sewing}}

The sewing-frame need by bookbinders is practically the same now as is
shown in prints of the early sixteenth century, and probably dates from
still earlier times. It consists of a bed with two uprights and a
crossbar, which can be heightened or lowered by the turning of wooden
nuts working on a screw thread cut in the uprights (see
\protect\hyperlink{Fig_29}{fig. 29}).

To set up for sewing, as many loops of cord, called ``lay cords,'' as
there are to be bands, are threaded on to the cross piece, and to these,
by a simple knot, shown at \protect\hyperlink{Fig_28}{fig. 28}, cords
are fastened to form the bands. The ``lay
cords{\protect\hypertarget{Page_101}{}{{[}101{]}}}'' can be used again
and again until worn out.

\protect\hypertarget{Fig_28}{}{}
\includegraphics[width=2.13542in,height=3.125in]{images/gs102.jpg}

Fig. 28.

To fasten the cord below, a key
is{\protect\hypertarget{Page_102}{}{{[}102{]}}} taken (see
\protect\hyperlink{Fig_28}{fig. 28}) and held below the press by the
right hand; the cord is then pulled up round it by the left, and held in
position on the key by the first finger of the right hand. The key is
then turned over, winding up a little of the string, and the prongs
slipped over the main cord. It is then put through the slit in the bed
of the sewing-press, with the prongs away from the front. The cord is
then cut off, and the same operation repeated for each band. When all
the bands have been set up, the book is laid against them, and they are
moved to correspond with the marks previously made on the back of the
book, care being taken that they are quite perpendicular. If they are of
the same length and evenly set up, on screwing up the crossbar they
should all tighten equally.

It will be found to be convenient to set up the cords as far to the
right hand of the press as possible, as then there will be room for the
sewer's left arm on the inner side of the left hand upright.

A roll of paper that will exactly fill the slot in the sewing-frame is
pushed in in front of the upright cords to
steady{\protect\hypertarget{Page_103}{}{{[}103{]}}} them and ensure that
they are all in the same plane.

When the sewing-frame is ready, with the cords set up and adjusted, the
book must be collated to make sure that neither sheets nor plates have
been lost or misplaced during the previous operations. Plates need
special care to see that the guards go properly round the sheets next
them.

\protect\hypertarget{Fig_29}{}{}
\includegraphics[width=3.125in,height=4.35417in]{images/gs105.jpg}

Fig. 29.

The top back corner, on front and back waste end paper, should be
marked. When this has been done, and all is found to be in order, the
book is laid on a pressing-board behind the sewing-frame, the fore-edge
towards the sewer, and the front end paper uppermost. As it is difficult
to insert the needle into a section placed on the bed of the
sewing-frame, it will be found convenient to sew upon a largish
pressing-board, which will lie on the bed of the frame, and may have
small catches to prevent it from shifting. When the board is in place,
the first section (end paper) is taken in the left hand and turned over,
so that the marks on the back come in the proper places against the
strings. The left hand is inserted into the place where the sewing is to
be,{\protect\hypertarget{Page_104}{}{{[}104{]}}} and with the right hand
a needle and thread is passed through the kettle
stitch{\protect\hypertarget{Page_105}{}{{[}105{]}}} mark (see
\protect\hyperlink{Fig_29}{fig. 29}). It is grasped by the fingers of
the left hand, is passed out through the back at the first mark on the
left-hand side of the first upright cord, and pulled tight, leaving a
loose end of thread at the kettle stitch. Then with the right hand it is
inserted again in the same place, but from the other side of the cord,
and so on round all five bands, and out again at the kettle stitch mark
at the tail, using right and left hands alternately. The centre of the
next section is then found, and it is sewn in the same way from tail to
head, the thread being tied to the loose end hanging from the first
kettle stitch. Another section is laid on and sewn, but when the kettle
stitch is reached, the under thread is caught up in the way shown in
\protect\hyperlink{Fig_30}{fig. 30}. These operations are repeated
throughout the whole book. If the back seems likely to swell too much,
the sections can{\protect\hypertarget{Page_106}{}{{[}106{]}}} be lightly
tapped down with a loaded stick made for the purpose, care being taken
not to drive the sections inwards, as it is difficult to get such
sections out again. When all the sheets and the last end paper have been
sewn on, a double catch stitch is made, and the end cut off. This method
is known as flexible sewing ``all along.''

\protect\hypertarget{Fig_30}{}{}
\includegraphics[width=2.60417in,height=0.77083in]{images/gs106.jpg}

Fig. 30.

\protect\hypertarget{Fig_31}{}{}
\includegraphics[width=2.08333in,height=1.79167in]{images/gs107.jpg}

Fig. 31.

When one needle full of thread is exhausted, another is tied on, making
practically a continuous length of thread going all along each section
and round every band. The weaver's knot is the best for joining the
lengths of thread. A simple way of tying it is shown at
\protect\hyperlink{Fig_31}{fig. 31}. A simple slip knot is made in the
end of the new thread and put over the end of the old, and, on being
pulled tight, the old thread should slip through, as shewn at B. The
convenience of this knot is, that by its use a firm attachment can be
made quite close up to the back of
the{\protect\hypertarget{Page_107}{}{{[}107{]}}} book. This is a great
advantage, as if the knot is made at some distance from the back, it
will have to be dragged through the section two or three times, instead
of only once. The knot, after having been made, must be pulled inside
the section, and remain there. Considerable judgment is required in
sewing. If a book is sewn too loosely, it is almost impossible to bind
it firmly; and if too tightly, especially if the kettle stitches have
been drawn too tight, the thread may break in ``backing,'' and the book
have to be resewn.

One way to avoid having too much swelling in the back of a book
consisting of a great many very thin sections is to sew ``two sheets
on.'' In this form of sewing two sections at a time are laid on the
sewing-frame. The thread is inserted at the ``kettle stitch'' of the
lower section, and brought out as usual at the first cord, but instead
of being reinserted into the lower section, it is passed into the upper
one, and so on, alternately passing into the upper and lower sections.
This will give, if there are five bands, three stitches in each section
instead of six, as there would be if the sewing were
``all{\protect\hypertarget{Page_108}{}{{[}108{]}}} along,'' lessening
the thread, consequently the swelling by half. It is usual to sew the
first and last few sections ``all along.''

The common method of sewing is to make saw cuts in the back, in which
thin cords can be sunk, and the thread merely passes behind them and not
round them, as in flexible sewing. This method, although very quick and
cheap, is not to be recommended, on account of the injury done to the
backs of the sections by the saw, and because the glue running into the
saw cuts is apt to make the back stiff, and to prevent the book from
opening right to the back. Indeed, were a sawn-in book to open right to
the back, as it is expected a flexibly-sewn book will do, showing the
sewing along the centre of each section, the saw marks with the band
inserted would show, and be a serious disfigurement.

\protect\hypertarget{Fig_32}{}{}
\includegraphics[width=2.04167in,height=4.16667in]{images/gs111.jpg}

Fig. 32.

\protect\hypertarget{Fig_33}{}{}
\includegraphics[width=1.29167in,height=2.60417in]{images/gs112.jpg}

Fig. 33.

Mediæval books were usually sewn on double cords or strips of leather,
and the headband was often sewn at the same time, as shown at
\protect\hyperlink{Fig_32}{fig. 32}, A. This is an excellent method for
very large books with heavy sections, and is specially suitable for
large vellum manuscripts, in many of which the sections are very thick.
An{\protect\hypertarget{Page_109}{}{{[}109{]}}} advantage of this method
is, that the twist round the double cord virtually makes a knot at every
band, and should a thread at any place break, there is no danger of the
rest of the thread coming loose. This is the only mode of sewing by
which a thread runs absolutely from end to end of the sections. The
headband sewn at the same time, and so tied down in every section, is
firmer and stronger than if worked on in the way now usual. In the
fifteenth century it was the custom to lace the ends of the headbands
into the boards in the same way as the other bands. This method, while
giving additional strength at the head and tail, and avoiding the
somewhat unfinished look of the cut-off ends of the modern headband, is,
on the whole, of doubtful advantage, as it is necessary to cut the
``turn in'' at the point where strength in the leather is much wanted.

At \protect\hyperlink{Fig_32}{fig. 32} is shown in section the three
methods of sewing mentioned. A is the old sewing round double bands;
with the{\protect\hypertarget{Page_110}{}{{[}110{]}}} headbands worked
at the same time with the same thread; B is the modern flexible sewing,
and C the common sawn-in method.

Books that are very thin or are to
be{\protect\hypertarget{Page_111}{}{{[}111{]}}} bound in vellum, are
best sewn on tapes or vellum slips. The easiest way to set up the
sewing-frame for such sewing is to sling a piece of wood through two of
the lay cords, and to pin one end of the vellum or tape band round this,
pull the other end tight, and secure it with a drawing-pin underneath
the frame. The sewing, in the case of such flat bands, would not go
round, but only across them. To avoid undue looseness, every three or
four threads may be caught up at the back of the band, as shown in
\protect\hyperlink{Fig_33}{fig. 33}.

\hypertarget{materials-for-sewing}{%
\subparagraph{MATERIALS FOR SEWING}\label{materials-for-sewing}}

The cord used should be of the best hemp, specially made with only two
strands of very long fibres to facilitate fraying out. For very large
books where a double cord is to be used, the best water line will
be{\protect\hypertarget{Page_112}{}{{[}112{]}}} found to answer, care
being taken to select that which can be frayed out. If tape is used it
should be unbleached, such as the sailmakers use. Thread should also be
unbleached, as the unnecessary bleaching of most bookbinder's
sewing-thread seems to cause it to rot in a comparatively short time.
Silk of the best quality is better than any thread. The ligature silk,
undyed, as used by surgeons, is perhaps the strongest material, and can
be had in various thicknesses. It is impossible to pay too great
attention to the selection of sewing materials, as the permanency of the
binding depends on their durability. The rebinding of valuable books is
at best a necessary evil, and anything that makes frequent rebinding
necessary, is not only objectionable on account of the cost involved,
but because it seriously shortens the life of the book.

Experience is required to judge what thickness of thread to use for any
given book. If the sections are very thin, a thin thread must be used,
or the ``swelling'' of the back caused by the additional thickness of
the thread in that part will be excessive, and make the book
unmanageable in ``backing.'' On the other
hand,{\protect\hypertarget{Page_113}{}{{[}113{]}}} if the sections are
large, and a too thin thread is used, there will not be enough swelling
to make a firm ``joint.'' Broadly speaking, when there are a great many
very thin sections, the thinnest thread may be used; and coarser thread
may be used when the sections are thicker, or fewer in number. In the
case of large manuscripts on vellum it is best to use very thick silk,
or even catgut. Vellum is so tough and durable, that any binding of a
vellum book should be made as if it were expected to last for hundreds
of years.

In selecting the thickness of cord for a book, some judgment is
required. On an old book the bands are best made rather prominent by the
use of thick cord, but the exact thickness to be used is a matter for
taste and experience to decide.

A very thick band on a small book is clumsy, while a very thin band on
the back of a heavy book suggests weakness, and is therefore unsightly.

In bindings of early printed books and manuscripts an appearance of
great strength is better than extreme neatness.

When the sewing is completed, the cords are cut off close to the lay
cords, and then{\protect\hypertarget{Page_114}{}{{[}114{]}}} the keys
will be loose enough to be easily removed. The knots remaining on the
lay bands are removed, and the keys slung through one of them.

\hypertarget{chapter-viii}{%
\subsection[CHAPTER
VIII]{\texorpdfstring{\protect\hypertarget{CHAPTER_VIII}{}{}CHAPTER
VIII}{CHAPTER VIII}}\label{chapter-viii}}

Fraying out Slips---Glueing up---Rounding and Backing

\hypertarget{fraying-out-slips-and-glueing-up}{%
\subparagraph{FRAYING OUT SLIPS AND GLUEING
UP}\label{fraying-out-slips-and-glueing-up}}

{After} sewing, the book should be looked through to see that all sheets
and plates have been caught by the thread, and special attention should
be given to end papers to see that the sewing lies evenly.

The ends of the cords should next be cut off to within about two inches
of the book on each side, and the free portions frayed out. If proper
sewing cord is used, this will be found to be very easily done, if a
binder's bodkin is first inserted between the two strands, separating
them, and then again in the centre of each separated strand to still
further straighten the fibres (see \protect\hyperlink{Fig_34}{fig. 34}).

The fraying out of the thick cord
recommended{\protect\hypertarget{Page_115}{}{{[}115{]}}} for heavy books
is a more difficult operation, but with a little trouble the fibres of
any good cord can be frayed out. Vellum or tape bands will only require
cutting off, leaving about two inches free on each side. The free parts
of the bands are called slips.

\protect\hypertarget{Fig_34}{}{}
\includegraphics[width=1.15625in,height=1.5625in]{images/gs116.jpg}

Fig. 34.

The book is now ready for glueing up. A piece of waste mill-board or an
old cloth cover is put on each side over the slips, and the book knocked
up squarely at the back and head. Then it is lowered into the lying
press and screwed up, leaving the back with the protecting boards
projecting about three-quarters of an inch. If the back has too much
swelling in it or is spongy, it is better to leave the slips on one side
free and to pull them as tight as possible while the book is held in the
press, or a knocking-down iron may be placed on one side of the
projecting back and the other side tapped with the backing hammer to
make the sections lie close to one another, and then the slips pulled
straight (\protect\hyperlink{Fig_35}{fig. 35}). The back
must{\protect\hypertarget{Page_116}{}{{[}116{]}}} now be glued. The glue
for this operation must be hot, and not too thick. It is very important
that it should be worked well between the sections with the brush, and
it is well after it has been applied to rub the back with a finger or
folder to make quite sure that the glue goes between every section for
its entire length. If the book is too tightly screwed up in the press,
the glue is apt to remain too much on the surface; and if not tightly
enough, it may penetrate too deeply between the sections. If the glue is
thick, or stringy, it may be diluted with hot water and the glue-brush
rapidly spun{\protect\hypertarget{Page_117}{}{{[}117{]}}} round in the
glue-pot to break it up and to make it work freely.

\protect\hypertarget{Fig_35}{}{}
\includegraphics[width=3.125in,height=2.67708in]{images/gs117.jpg}

Fig. 35.

Very great care is needed to see that the head of a previously trimmed
book is knocked up exactly square before the back is glued, for if it is
not, it will be very difficult to get it even afterwards.

\hypertarget{rounding-and-backing}{%
\subparagraph{ROUNDING AND BACKING}\label{rounding-and-backing}}

The amount of rounding on the back of a book should be determined by the
necessities of the case; that is to say, a back that has, through
guarding, or excess of sewing, a tendency to be round, is best not
forced to be flat, and a back that would naturally be flat, is best not
forced to be unduly round. A very round back is objectionable where it
can be avoided, because it takes up so much of the back margins of the
sheets, and is apt to make the book stiff in opening. On the other hand,
a back that is quite flat has to be lined up stiffly, or it may become
concave with use.

\protect\hypertarget{Fig_36}{}{}
\includegraphics[width=3.125in,height=3.19792in]{images/gs119.jpg}

Fig. 36.

The method of rounding is to place the book with the back projecting a
little over the edge of the press or table, then to draw the back over
towards the workman,{\protect\hypertarget{Page_118}{}{{[}118{]}}} and,
while in this position, to tap it carefully with a hammer (see
\protect\hyperlink{Fig_36}{fig. 36}). This is repeated on the other side
of the book, and, if properly done, will give the back an even, convex
form that should be in section, a portion of a circle. Rounding and
backing are best done after the glue has ceased to be tacky, but before
it has set hard.{\protect\hypertarget{Page_119}{}{{[}119{]}}}

\protect\hypertarget{Fig_37}{}{}
\includegraphics[width=2.60417in,height=3.05208in]{images/gs120.jpg}

Fig. 37.

Backing is perhaps the most difficult and important operation in
forwarding. The sewing threads in the back cause that part to be thicker
than the rest of the book. Thus in a book with twenty sections there
will be in the back, in addition to the thickness of the paper, twenty
thicknesses of thread.

If the boards were laced on to the book without rounding or backing, and
the book were pressed, the additional thickness of the back, having to
go somewhere, would cause it to go either convex or concave, or else
perhaps to crease up{\protect\hypertarget{Page_120}{}{{[}120{]}}} (see
\protect\hyperlink{Fig_37}{fig. 37}). The object of rounding is to
control the distribution of this swelling, and to make the back take an
even and permanently convex form.

\protect\hypertarget{Fig_38}{}{}
\includegraphics[width=1.04167in,height=1.23958in]{images/gs121.jpg}

Fig. 38.

If the boards were merely laced on after rounding, there would be a gap
between the square ends of the board and the edge of the back (see
\protect\hyperlink{Fig_38}{fig. 38}), though the convexity and even
curve of the back would be to some extent assured. What is done in
backing is to make a groove, into which the edges of the board will fit
neatly, and to hammer the backs of the sections over one another from
the centre outwards on both sides to form the ``groove,'' to ensure that
the back shall return to the same form after the book has been opened.

\protect\hypertarget{Fig_39}{}{}
\includegraphics[width=1.04167in,height=1.375in]{images/gs121a.jpg}

Fig. 39.

To back the book, backing boards are placed on each side (leaving the
slips outside) a short distance below the edge of the back
(\protect\hyperlink{Fig_39}{fig. 39}). The amount to leave here must be
decided by the thickness of the boards to be used. When the
backing{\protect\hypertarget{Page_121}{}{{[}121{]}}} boards are in
position, the book and boards must be carefully lowered into the lying
press and screwed up very tight, great care being taken to see that the
boards do not slip, and that the book is put in evenly. Even the most
experienced forwarder will sometimes have to take a book out of the
press two or three times before he gets it in quite evenly and without
allowing the boards to slip. Unless the back has a perfectly even curve
when put in the press for backing, no amount of subsequent hammering
will put it permanently right.

\protect\hypertarget{Fig_40}{}{}
\includegraphics[width=3.125in,height=2.23958in]{images/gs122.jpg}

Fig. 40.

\protect\hypertarget{Fig_41}{}{}
\includegraphics[width=2.60417in,height=3.20833in]{images/gs123.jpg}

Fig. 41.

The backs of the sections should be evenly fanned out one over the other
from the centre outwards on both
sides.{\protect\hypertarget{Page_122}{}{{[}122{]}}} This is done by side
strokes of the hammer, in fact by a sort of ``riveting'' blow, and not
by a directly crushing blow (see \protect\hyperlink{Fig_41}{fig. 41}, in
which the arrows show the direction of the hammer strokes). If the
sections are not evenly fanned out from the centre, but are either
zigzagged by being crushed by direct blows of the hammer, as shown in
\protect\hyperlink{Fig_42}{fig. 42}, A, or are unevenly fanned over more
to one side than the other, as shown in \protect\hyperlink{Fig_42}{fig.
42}, B, the back, although it may be even enough when first done, will
probably become uneven{\protect\hypertarget{Page_123}{}{{[}123{]}}} with
use. A book in which the sections have been crushed down, as at
\protect\hyperlink{Fig_42}{fig. 42}, A, will be disfigured inside by
creases in the paper.\protect\hypertarget{Fig_43}{}{}

\protect\hypertarget{Fig_42}{}{}
\includegraphics[width=3.125in,height=1.69792in]{images/gs124.jpg}

Fig. 42.

\begin{longtable}[]{@{}
  >{\raggedright\arraybackslash}p{(\columnwidth - 2\tabcolsep) * \real{0.5000}}
  >{\raggedright\arraybackslash}p{(\columnwidth - 2\tabcolsep) * \real{0.5000}}@{}}
\toprule()
\endhead
\begin{minipage}[t]{\linewidth}\raggedright
\includegraphics[width=1.04167in,height=1.89583in]{images/gs124a.jpg}

Fig. 43.
\end{minipage} & \begin{minipage}[t]{\linewidth}\raggedright
\protect\hypertarget{Fig_44}{}{}
\includegraphics[width=1.5625in,height=2.11458in]{images/gs124b.jpg}

Fig. 44.
\end{minipage} \\
\bottomrule()
\end{longtable}

It is a mistake to suppose that a
very{\protect\hypertarget{Page_124}{}{{[}124{]}}} heavy hammer is
necessary for backing any but the largest books. For flexible books a
hammer with a comparatively small face should be used, as by its use the
book can be backed without flattening the bands. It is well to have a
hammer head of the shape shown in \protect\hyperlink{Fig_43}{fig. 43}.
By using the thin end, the force of a comparatively light blow, because
concentrated on a small surface, is effective.

At \protect\hyperlink{Fig_44}{fig. 44} is shown an ordinary backing
hammer.

\hypertarget{chapter-ix}{%
\subsection[CHAPTER
IX]{\texorpdfstring{\protect\hypertarget{CHAPTER_IX}{}{}CHAPTER
IX}{CHAPTER IX}}\label{chapter-ix}}

Cutting and Attaching Boards---Cleaning off Back---Pressing

\hypertarget{cutting-and-attaching-boards}{%
\subparagraph{CUTTING AND ATTACHING
BOARDS}\label{cutting-and-attaching-boards}}

{The} first quality of the best black board made from old rope is the
best to use for ``extra'' binding. It will be found to be very hard, and
not easily broken or bent at the corners. In selecting the thickness
suitable for any given book, the size and thickness of the volume should
be taken into account. The tendency of most modern binders is to use a
rather{\protect\hypertarget{Page_125}{}{{[}125{]}}} over thick board,
perhaps with a view to bulk out the volume. For manuscripts, or other
books on vellum, it is best to use wooden boards, which should be
clasped. From their stability they form a kind of permanent press, in
which the vellum leaves are kept flat. In a damp climate like that of
England, vellum, absorbing moisture from the atmosphere, soon cockles up
unless it is held tightly in some way; and when it is once cockled, the
book cannot be made to shut properly, except with very special
treatment. Then also dust and damp have ready access to the interstices
of the crinkled pages, resulting in the disfigurement so well known and
so deplored by all lovers of fine books.

For large books a ``made'' board, that is, two boards pasted together,
is better than a single board of the same thickness. In making boards a
thin and a thick board should be pasted together, the thin board to go
nearest the book. It will not be necessary to put a double lining on the
inside of such boards, as a thin board will always draw a thick one.

\protect\hypertarget{Fig_45}{}{}
\includegraphics[width=2.60417in,height=3.57292in]{images/gs127.jpg}

Fig. 45.

If mill-boards are used they are first cut roughly to size with the
mill-board{\protect\hypertarget{Page_126}{}{{[}126{]}}} shears, screwed
up in the ``lying'' press. The straight arm of the shears is the one to
fix in the press, for if the bent arm be undermost, the knuckles are apt
to be severely bruised against the end. A better way of fixing the
shears is shown at \protect\hyperlink{Fig_45}{fig. 45}. Any blacksmith
will bend the arm of the shears and make the necessary clips. This
method saves trouble and
considerable{\protect\hypertarget{Page_127}{}{{[}127{]}}} wear and tear
to the ``lying'' press. Where a great many boards are needed, they may
be quickly cut in a board machine, but for ``extra'' work they should be
further trimmed in the plough, in the same way as those cut by the
shears. After the boards have been roughly cut to size, they should have
one edge cut straight with the plough. To do this one or two pairs of
boards are knocked up to the back and inserted in the cutting side of
the press, with those edges projecting which are to be cut off, and
behind them, as a ``cut against,'' a board protected by a waste piece of
mill-board.

The plough, held by the screw and handle, and guided by the runners on
the press, is moved backwards and forwards. A slight turn of the screw
at each movement brings the knife forward. In cutting mill-boards which
are very hard, the screw should be turned very little each time. If
press and plough are in proper order, that part of the board which
projects above the cheek of the press should be cut off, leaving the
edges perfectly square and straight. If the edge of the press has been
damaged, or is out of ``truth,''
a{\protect\hypertarget{Page_128}{}{{[}128{]}}} cutting board may be used
between the cheek of the press and the board to be cut, making a true
edge for the knife to run on.

\protect\hypertarget{Fig_46}{}{}
\includegraphics[width=3.125in,height=4.26042in]{images/gs129.jpg}

Fig. 46.---Lying or Cutting Press

The position of the plough on the
press{\protect\hypertarget{Page_129}{}{{[}129{]}}} is shown at
\protect\hyperlink{Fig_46}{fig. 46}. The side of the press with runners
should be reserved for cutting, the other side used for all other work.

\protect\hypertarget{Fig_47}{}{}
\includegraphics[width=1.04167in,height=1.36458in]{images/gs130.jpg}

Fig. 47.

The plough knife for mill-boards should not be ground at too acute an
angle, or the edge will most likely break away at the first cut. The
shape shown at \protect\hyperlink{Fig_47}{fig. 47} is suitable. The
knife should be very frequently ground, as it soon gets blunt, which
adds greatly to the labour of cutting.

After an edge has been cut, each side should be well rubbed with a
folder to smooth down any burr left by the plough knife. Then a piece of
common paper with one edge cut straight is pasted on to one side of the
board, with the straight edge exactly up to the cut edge of the board.
Then a piece of paper large enough to cover both sides of the board is
pasted round it, and well rubbed down at the cut edge. After having been
lined, the boards are nipped in the press to ensure that the lining
paper shall stick.{\protect\hypertarget{Page_130}{}{{[}130{]}}} They are
stood up to dry, with the doubly lined side outwards. The double paper
is intended to warp the board slightly to that side, to compensate for
the pull of the leather when the book is covered. If the board is a
double one, a single lining paper will be sufficient, the thinner board
helping to draw the thicker. The paste for lining boards must be fairly
thin, and very well beaten up so as to be free from lumps. It is of the
utmost importance that the lining papers should stick properly, for
unless they stick, no subsequent covering of leather or paper can be
made to lie flat.

When the lined boards are quite dry, they should be paired with the
doubly lined sides together, and the top back corner marked to
correspond with the marks on the top back corners of the book. Then near
the top edge, with the aid of a carpenter's square, two points are
marked in a line at right angles to the cut edge. The pair of boards is
then knocked up to the back and lowered into the press as before, so
that the plough knife will exactly cut through the points. The same
operation is repeated on the two remaining uncut edges. In marking
out{\protect\hypertarget{Page_131}{}{{[}131{]}}} those for the
fore-edge, the measurement is taken with a pair of compasses
(\protect\hyperlink{Fig_48}{fig. 48}) from the joint of the book to the
fore-edge of the first section. If the book has been trimmed, or is to
remain uncut, a little more must be allowed for the ``squares,'' and if
it is to be cut in the plough, it must be now decided how much is to be
cut off, remembering that it is much better to have the boards a little
too large, and so have to reduce them after the book is cut, than to
have them too small, and either be obliged to get out a new pair of
boards, or unduly cut down the book.

\protect\hypertarget{Fig_48}{}{}
\includegraphics[width=1.5625in,height=2.71875in]{images/gs132.jpg}

Fig. 48.

\protect\hypertarget{Fig_49}{}{}
\includegraphics[width=1.04167in,height=1.3125in]{images/gs133.jpg}

Fig. 49.

The height of the boards for a book that has been trimmed, or is to
remain uncut, will be the height of the page with a small allowance at
each end for the squares. When a pair of boards has been cut all round,
it can be tested for squareness by reversing one board, when any
inequality that there may be will appear doubled. If the boards are out
of truth{\protect\hypertarget{Page_132}{}{{[}132{]}}} they should
generally be put on one side, to be used for a smaller book, and new
boards got out. To correct a badly cut pair of boards, it is necessary
to reduce them in size, and the book consequently suffers in proportion.
If the boards have been found to be truly cut, they are laid on the
book, and the position of the slips marked on them by lines at right
angles to the back. A line is then made parallel to the back, about half
an inch in (see \protect\hyperlink{Fig_49}{fig. 49}). At the points
where the lines cross, a series of holes is punched from the front with
a binder's bodkin on a lead plate, then the board is turned over, and a
second series is punched from the back about half an inch from the
first. If the groove of the back is shallower than the thickness of the
board, the top back edge of the board should be bevelled off with a
file. This will not be necessary if the groove is the exact depth. When
the holes have been punched, it is well to cut a series of V-shaped
depressions from the first series of holes to the back to
receive{\protect\hypertarget{Page_133}{}{{[}133{]}}} the slips, or they
may be too prominent when the book is bound. It will now be necessary to
considerably reduce the slips that were frayed out after sewing, and to
remove all glue or any other matter attached to them. The extent to
which they may be reduced is a matter of nice judgment. In the desire to
ensure absolute neatness in the covering, modern binders often reduce
the slips to almost nothing. On the other hand, some go to the other
extreme, and leave the cord entire, making great ridges on the sides of
the book where it is laced in. It should be possible with the aid of the
depressions, cut as described, to use slips with sufficient margin of
strength, and yet to have no undue projection on the cover. A slight
projection is not unsightly, as it gives an assurance of sound
construction and strength, and, moreover, makes an excellent
starting-point for any pattern that may be used. When the slips have
been scraped and reduced, the portion left should consist of long
straight silky fibres. These must be well pasted, and the ends very
slightly twisted. The pointed ends are then threaded through the first
series of holes in the front of the board,
and{\protect\hypertarget{Page_134}{}{{[}134{]}}} back again through the
second (\protect\hyperlink{Fig_50}{fig. 50}). In lacing-in the slips
must not be pulled so tight as to prevent the board from shutting
freely, nor left so loose as to make a perceptible interval in the joint
of the book. The pasted slips having been laced in, their ends are cut
off with a sharp knife, flush with the surface of the board. The
laced-in slips are then well hammered on a knocking-down iron (see
\protect\hyperlink{Fig_51}{fig. 51}), first from the front and then from
the back, care being taken that the hammer face should fall squarely, or
the slips may be cut. This should rivet them into the board, leaving
little or no projection.{\protect\hypertarget{Page_135}{}{{[}135{]}}} If
in lacing in the fibres should get twisted, no amount of hammering will
make them flat, so that it is important in pointing the ends for lacing
in, that only the points are twisted just sufficiently to facilitate the
threading through the holes, and not enough to twist the whole slip.

\protect\hypertarget{Fig_50}{}{}
\includegraphics[width=3.125in,height=2.41667in]{images/gs135.jpg}

Fig. 50.

\protect\hypertarget{Fig_51}{}{}
\includegraphics[width=3.125in,height=2.69792in]{images/gs136.jpg}

Fig. 51.

To lace slips into wooden boards, holes are made with a brace and fine
twist bit, and the ends of the frayed out slips may be secured with a
wooden plug (see \protect\hyperlink{Fig_52}{fig. 52}).

Old books were sometimes sewn
on{\protect\hypertarget{Page_136}{}{{[}136{]}}} bands of leather, but as
those sewn on cord seem to have lasted on the whole much better, and as,
moreover, modern cord is a far more trustworthy material than modern
leather, it is better to use cord for any books bound now.

\protect\hypertarget{Fig_52}{}{}
\includegraphics[width=3.125in,height=4.4375in]{images/gs137.jpg}

Fig. 52.

{\protect\hypertarget{Page_137}{}{{[}137{]}}}

\hypertarget{cleaning-off-the-back-and-pressing}{%
\subparagraph{CLEANING OFF THE BACK AND
PRESSING}\label{cleaning-off-the-back-and-pressing}}

\protect\hypertarget{Fig_53}{}{}
\includegraphics[width=1.04167in,height=1.75in]{images/gs138.jpg}

Fig. 53.

When the boards have been laced on and the slips hammered down, the book
should be pressed. Before pressing, a tin is put on each side of both
boards, one being pushed right up into the joint on the inside, and the
other up to the joint, or a little over it, on the outside. While in the
press, the back should be covered with paste and left to soak for a few
minutes. When the glue is soft the surplus on the surface can be scraped
off with a piece of wood shaped as shown in
\protect\hyperlink{Fig_53}{fig. 53}. For important books it is best to
do this in the lying press, but some binders prefer first to build up
the books in the standing press, and then to paste the backs and clean
them off there. This has the advantage of being a quicker method, and
will, in many cases, answer quite well. But for books that require nice
adjustment it will be found better to clean off each volume separately
in the lying press, and afterwards to build up the books
and{\protect\hypertarget{Page_138}{}{{[}138{]}}} boards in the standing
press, putting the larger books at the bottom. It must be seen that the
entire pile is exactly in the centre under the screw, or the pressure
will be uneven. To ascertain if the books are built up truly, the pile
must be examined from both the front and side of the press. Each volume
must also be looked at carefully to see that it lies evenly, and that
the back is not twisted or out of shape. This is important, as any form
given to the book when it is pressed at this stage will be permanent.

Any coloured or newly printed plates will need tissues, as in the former
pressing; and any folded plates or diagrams or inserted letters will
need a thin tin on each side of them to prevent them from marking the
book.

Again, the pressure on hand-printed books must not be excessive.

The books should be left in the press at least a night. When taken out
they will be ready for headbanding, unless the edges are to be cut in
boards.

\hypertarget{chapter-x139}{%
\subsection[CHAPTER
X]{\texorpdfstring{\protect\hypertarget{CHAPTER_X}{}{}CHAPTER
X{\protect\hypertarget{Page_139}{}{{[}139{]}}}}{CHAPTER X{[}139{]}}}\label{chapter-x139}}

Cutting in Boards---Gilding and Colouring Edges

\hypertarget{cutting-in-boards}{%
\subparagraph{CUTTING IN BOARDS}\label{cutting-in-boards}}

{The} knife for cutting edges may be ground more acutely than for
cutting boards, and should be very sharp, or the paper may be torn. The
plough knife should never be ground on the under side, as if the under
side is not quite flat, it will tend to run up instead of cutting
straight across. Before beginning to cut edges, the position of the
knife should be tested carefully by screwing the plough up, with the
press a little open, and noting whereabouts on the left-hand cheek the
point of the knife comes. In a press that is true the knife should just
clear the edge of the press. If there is too much packing the knife will
cut below the edge of the press, and if too little, it will cut above.

``Packing'' is paper inserted between the knife and the metal plate on
the plough, to correct the position of the knife. When by experiment the
exact thickness of paper necessary for any
given{\protect\hypertarget{Page_140}{}{{[}140{]}}} knife is found, the
packing should be carefully kept when the knife is taken out for
grinding, and put back with it into the plough.

The first edge to be cut is the top, and the first thing to do is to
place the boards in the position they will hold when the book is bound.
The front board is then dropped the depth of the square required, care
being taken that the back edge of the board remains evenly in the joint.
A piece of cardboard, or two or three thicknesses of paper, are then
slipped in between the end paper and the back board to prevent the
latter from being cut by the knife. The book is then carefully lowered
into the press, with the back towards the workman, until the top edge of
the front board is exactly even with the right-hand cheek, and the press
screwed up evenly. The back board should show the depth of the square
above the left-hand cheek. It is very important that the edge of the
back board should be exactly parallel with the press, and if at first it
is not so, the book must be twisted until it is right.

The edges can now be cut with the plough as in cutting mill-boards. The
tail of the book is cut in the same
way,{\protect\hypertarget{Page_141}{}{{[}141{]}}} still keeping the back
of the book towards the workman, but cutting from the back board.

\protect\hypertarget{Fig_54}{}{}
\includegraphics[width=1.5625in,height=1.53125in]{images/gs142.jpg}

Fig. 54.

\protect\hypertarget{Fig_55}{}{}
\includegraphics[width=1.25in,height=2.32292in]{images/gs143.jpg}

Fig. 55.

Cutting the fore-edge is more difficult. The waste sheets at each end of
the book should be cut off flush with the edge of the board, and marks
made on them below the edge showing the amount of the square, and
consequently how much is to be cut off. The curve of the back, and
consequent curve of the fore-edge, must first be got rid of, by
inserting a pair of pieces of flat steel called ``trindles''
(\protect\hyperlink{Fig_54}{fig. 54}) across the back, from the inside
of the boards. When these are inserted the back must be knocked quite
flat, and, in the case of a heavy book, a piece of tape may be tied
round the leaves (see \protect\hyperlink{Fig_55}{fig. 55}) to keep them
in position. A pair of cutting boards is placed one on each side of the
leaves, the back one exactly up to the point that the edge of the board
came to, and the front one as much below that point as it is desired the
square of the fore-edge
should{\protect\hypertarget{Page_142}{}{{[}142{]}}} be. The trindles are
removed while the book is held firmly between the cutting boards by the
finger and thumb; book and boards are then lowered very carefully into
the press. The top edge of the front cutting board should be flush with
the right-hand cheek of the press, and that of the back a square above
the left-hand cheek (see \protect\hyperlink{Fig_56}{fig. 56}). A further
test is to look along the surface of the right-hand cheek, when, if the
book has been inserted truly, the amount of the back cutting board in
sight should exactly correspond with the amount of the paper to be cut
showing above the front board. It will also be necessary before cutting
to look at the back, and to see that it has remained flat. If it has
gone back to its old curve, or the book has been put into the press
crookedly, it must be taken right out again and
the{\protect\hypertarget{Page_143}{}{{[}143{]}}} trindles inserted
afresh, as it is usually a waste of time to try to adjust the book when
it is in the press. The leaves are cut in the same way as those of the
head and tail.

\protect\hypertarget{Fig_56}{}{}
\includegraphics[width=2.60417in,height=3.57292in]{images/gs144.jpg}

Fig. 56.

\hypertarget{gilding-or-colouring-the-edges-of-a-cut-book144}{%
\subparagraph[GILDING OR COLOURING THE EDGES OF A CUT
BOOK]{\texorpdfstring{GILDING OR COLOURING THE EDGES OF A CUT
BOOK{\protect\hypertarget{Page_144}{}{{[}144{]}}}}{GILDING OR COLOURING THE EDGES OF A CUT BOOK{[}144{]}}}\label{gilding-or-colouring-the-edges-of-a-cut-book144}}

Gilding the edges of a book cut in boards is much the same process as
that described for the trimmed book, excepting that when gilt in boards
the edges can be scraped and slightly sand-papered. It is the custom to
admire a perfectly solid gilt edge, looking more like a solid sheet of
metal, than the leaves of a book. As the essential characteristic of a
book is, that it is composed of leaves, this fact is better accepted and
emphasised by leaving the edges a little rough, so that even when gilt
they are evidently the edges of leaves of paper, and not the sides of a
block, or of something solid.

To gild the edges of a cut book the boards should be turned back, and
cutting boards put on each side of the book flush with the edge to be
gilt. For the fore-edge the book must be thrown up with trindles first,
unless it is desired to gild in the round, a process which gives the
objectionable solid metallic edge.

After the edges have been gilt they may be decorated by tooling, called
``gauffering.{\protect\hypertarget{Page_145}{}{{[}145{]}}}''

This may be done, either by tooling with hot tools directly on the gold
while the leaves are screwed up tightly in the press, or by laying
another coloured gold on the top of the first and tooling over that,
leaving the pattern in the new gold on the original colour. But, to my
mind, edges are best left undecorated, except for plain gold or colour.

If the edges are to be coloured, they should be slightly scraped, and
the colour put on with a sponge, commencing with the fore-edge, which
should be slightly fanned out, and held firmly, by placing a
pressing-board above it, and pressing with the hand on this. The colour
must be put on very thinly, commencing from the centre of the fore-edge
and working to either end, and as many coats put on as are necessary to
get the depth of colour required. The head and tail are treated in the
same way, excepting that they cannot be fanned out, and the colour
should be applied from the back to the fore-edge. If in the fore-edge an
attempt is made to colour from one end to the other, and if in the head
or tail from the fore-edge to the back, the result will almost certainly
be that the sponge will leave a
thick{\protect\hypertarget{Page_146}{}{{[}146{]}}} deposit of colour
round the corner from which it starts.

For colouring edges almost any stain will answer, or ordinary
water-colours may be used if moistened with size.

When the colour is dry the edge should be lightly rubbed over with a
little beeswax, and burnished with a tooth burnisher (see
\protect\hyperlink{Fig_57}{fig. 57}).

\protect\hypertarget{Fig_57}{}{}
\includegraphics[width=1.04167in,height=0.27083in]{images/gs147.jpg}

Fig. 57.

In addition to plain colour and gilding, the edges of a book may be
decorated in a variety of ways. The fore-edge may be fanned out and
painted in any device in water-colour and afterwards gilded; the
painting will only show when the book is open. The fore-edge for this
must be cut very solid, and if the paper is at all absorbent, must be
sized with vellum size before being painted. The paints used must be
simple water-colour, and the edge must not be touched with the hand
before gilding, as if there is any grease or finger-mark on it, the gold
will not stick evenly. Painting on the fore-edge should only be
attempted when the paper of the book is thin and of good quality. More
common methods of decorating edges are by marbling and sprinkling, but
they are both{\protect\hypertarget{Page_147}{}{{[}147{]}}} inferior to
plain colouring. Some pleasant effects are sometimes obtained by
marbling edges and then gilding over the marbling.

\hypertarget{chapter-xi}{%
\subsection[CHAPTER
XI]{\texorpdfstring{\protect\hypertarget{CHAPTER_XI}{}{}CHAPTER
XI}{CHAPTER XI}}\label{chapter-xi}}

Headbanding

\hypertarget{headbands}{%
\subparagraph{HEADBANDS}\label{headbands}}

{Modern} headbands are small pieces of vellum, gut, or cord sewn on to
the head and tail of a book with silk or thread. They resist the strain
on the book when it is taken from the shelf. The vellum slip or cord
must be of such a depth, that when covered with silk it will be slightly
lower than the square of the boards. The cut edge of the vellum always
slants, and the slip must be placed in position so that it tilts back
rather than forward on the book.

To start, ease the boards slightly on the slips and pull them down with
the top edges flush with the top edge of the leaves. If this is not done
the silk catches on the projecting edges as the band is worked. Stand
the book in a finishing{\protect\hypertarget{Page_148}{}{{[}148{]}}}
press, fore-edge to the worker, and tilted forward so as to give a good
view of the headband as it is worked. The light must come from the left,
and well on to the work. A needle threaded with silk is put in at the
head of the book, and through the centre of the first section after the
end papers, and drawn out at the back below the kettle stitch with about
two-thirds of the silk. The needle is again inserted in the same place,
and drawn through until a loop of silk is left. The vellum slip is
placed in the loop, with the end projecting slightly to the left. It
must be held steady by a needle placed vertically behind it, with its
point between the leaves of the first section. The needle end of silk is
then behind the headband, and the shorter end in front. The needle end
is brought over from the back with the right hand, passed into the left
hand, and held taut. The short end is picked up with the right hand,
brought over the needle end under the vellum, and pulled tight from the
back. This is repeated; the back thread is again drawn up and over the
band to the front, the needle end crosses it, and is drawn behind under
the vellum slip, and so on. The crossing
of{\protect\hypertarget{Page_149}{}{{[}149{]}}} the threads form a
``bead,'' which must be watched, and kept as tight as possible, and well
down on the leaves of the book. Whenever the vellum or string begins to
shift in position, it must be tied down. This is done when the needle
end of silk is at the back. A finger of the left hand is placed on the
thread of silk at the back, and holds it firmly just below the slip. The
needle end is then brought up and over the slip, but instead of crossing
it with the front thread, the needle is passed between the leaves and
out at the back of the book, below the kettle stitch, and the thread
gradually drawn tight, and from under the left-hand finger. The loop so
made will hold the band firmly, and the silk can then be brought up and
over the slip and crossed in the usual way. The band should be worked as
far as the end papers, and should be finished with a double ``tie
down,'' after which the front thread is drawn under the slip to the
back. Both the ends of silk are then cut off to about half an inch,
frayed out, and pasted down as flatly as possible on the back of the
book.

The band should be tied down frequently. It is not too much to tie
down{\protect\hypertarget{Page_150}{}{{[}150{]}}} every third time the
needle end of the silk comes to the back. To make good headbands the
pull on the silk must be even throughout.

When the ends of the silk are pasted down, the ends of the vellum slip
are cut off as near the silk as possible. The correct length of the
headband is best judged by pressing the boards together with thumb and
finger at the opposite ends of the band, so as to compress the sections
into their final compass. If the band then buckles in the least, it is
too long and must be shortened.

The mediæval headbands were sewn with the other bands (see
\protect\hyperlink{Fig_32}{fig. 32}), and were very strong, as they were
tied down at every section. Modern worked headbands, although not so
strong, are, if frequently tied down, strong enough to resist any
reasonable strain. There are many other ways of headbanding, but if the
one described is mastered, the various other patterns will suggest
themselves if variety is needed. For very large books a double headband
may be worked on two pieces of gut or string---a thick piece with a thin
piece in front. The string should first be soaked in thin glue and left
to{\protect\hypertarget{Page_151}{}{{[}151{]}}} dry. Such a band is
worked with a figure of eight stitch. Headbands may also be worked with
two or three shades of silk. As vellum is apt to get hard and to break
when it is used for headbanding, it is well to paste two pieces together
with linen in between, and to cut into strips as required.

Machine-made headbands can be bought by the yard. Such bands are merely
glued on, but as they have but little strength, should not be used.

Where leather joints are used, the headbands may be worked on pieces of
soft leather sized and screwed up. If the ends are left long and tied in
front while the book is being covered, they may be conveniently let into
grooves in the boards before the leather joint is pasted down. This
method, I think, has little constructive value, but it certainly avoids
the rather unfinished look of the cut-off headband.

\hypertarget{chapter-xii152}{%
\subsection[CHAPTER
XII]{\texorpdfstring{\protect\hypertarget{CHAPTER_XII}{}{}CHAPTER
XII{\protect\hypertarget{Page_152}{}{{[}152{]}}}}{CHAPTER XII{[}152{]}}}\label{chapter-xii152}}

Preparing for Covering---Paring Leather---Covering---Mitring
Corners---Filling-in Boards

\hypertarget{preparing-for-covering}{%
\subparagraph{PREPARING FOR COVERING}\label{preparing-for-covering}}

{After} the headband is worked, a piece of brown or other stout paper
should be well glued on at the head and tail, care being taken that it
is firmly attached to the back and the headband. When dry, the part
projecting above the headband is neatly cut off, and the part on the
back well sand-papered, to remove any irregularity caused by the
tie-downs attaching the headband. For most books this will be quite
sufficient lining up, but very heavy books are best further lined up
between the bands with linen, or thin leather. This can be put on by
pasting the linen or leather and giving the back a very thin coat of
glue.

\protect\hypertarget{Fig_58}{}{}
\includegraphics[width=1.04167in,height=0.90625in]{images/gs154.jpg}

Fig. 58.

The only thing now left to do before covering will be to set the squares
and to cut off a small piece of the back corner of each board at the
head and tail, to make it possible for the boards to open
and{\protect\hypertarget{Page_153}{}{{[}153{]}}} shut without dragging
the head-cap out of place. The form of the little piece to be cut off
varies with each individual binder, but I have found for an octavo book
that a cut slightly sloping from the inside cutting off the corner about
an eighth of an inch each way, gives the best result (see
\protect\hyperlink{Fig_58}{fig. 58}). When the corner has been cut off,
the boards should be thrown back, and the slips between the book and the
board well pasted. When these have soaked a little, the squares of the
boards are set; that is, the boards are fixed so that exactly the same
square shows on each board above head and tail. A little larger square
is sometimes an advantage at the tail to keep the head-cap well off the
shelf, the essential thing being that both head and both tail squares
should be the same. In the case of an old book that has not been recut,
the edges will often be found to be uneven. In such cases the boards
must be made square, and so set that the book stands up straight.

When the slips have been pasted and the squares set, tins can be put
inside and outside the boards, and the book given
a{\protect\hypertarget{Page_154}{}{{[}154{]}}} slight nip in the press
to flatten the slips. Only a comparatively light pressure should be
given, or the lining up of the headbands or back will become cockled and
detached.

\hypertarget{paring-leather}{%
\subparagraph{PARING LEATHER}\label{paring-leather}}

While the slips are being set in the press the cover can be got out.
Judgment is necessary in cutting out covers. One workman will be able,
by careful cutting, to get six covers out of a skin where another will
only get four. The firm part of the skin is the back and sides, and this
only should be used for the best books. The fleshy parts on the flanks
and belly will not wear sufficiently well to be suitable for good
bookbinding.

The skin should be cut out leaving about an inch all round for turning
in when the book is covered, and when cut out it must be pared. If the
leather is of European manufacture most of the paring will have been
done before it is sold, and the leather manufacturer will have shaved it
to any thickness required. This is a convenience that is partly
responsible for the unduly thin leather that
is{\protect\hypertarget{Page_155}{}{{[}155{]}}} commonly used. The
better plan is to get the leather rather thick, and for the binder to
pare it down where necessary. For small books it is essential, in order
that the covers may open freely, and the boards not look clumsy, that
the leather should be very thin at the joint and round the edges of the
boards. For such books it is very important that a small, naturally thin
skin should be used that will not have to be unduly pared down, and that
the large and thicker skins should be kept for large books.

Binders like using large skins because there is much less waste, but if
these skins are used for small books, so much of the leather substance
has to be pared away, that only the comparatively brittle grained
surface remains. By the modern process of dyeing this surface is often
to some extent injured, and its strength sometimes totally destroyed.

When the cover has been cut to size the book is laid on it with the
boards open, and a pencil line drawn round them, a mark being made to
show where the back comes. The skin is then pared, making it thin where
the edge of the boards will come. Great care must be taken that
the{\protect\hypertarget{Page_156}{}{{[}156{]}}} thinning does not
commence too abruptly, or a ridge will be apparent when the leather is
on the book.

The paring must be done quite smoothly and evenly. Every unevenness
shows when the cover is polished and pressed. Care is needed in
estimating the amount that will have to be pared off that part of the
leather that covers the back and joints. The object of the binder should
be to leave these portions as thick as he can consistently with the free
opening of the boards. The leather at the head-caps must be pared quite
thin, as the double thickness on the top of the headband is apt to make
this part project above the edges of the board. This is a great trouble,
especially at the tail, where, if the head-cap projects beyond the
boards, the whole weight of the book rests on it, and it is certain to
be rubbed off when the book is put on the shelf.

\protect\hypertarget{Fig_59}{}{}
\includegraphics[width=2.08333in,height=4.23958in]{images/gs158.jpg}

Fig. 59.

The method of paring with a French knife
(\protect\hyperlink{Fig_60}{fig. 60}, A)---the only form of
knife{\protect\hypertarget{Page_157}{}{{[}157{]}}} in use by binders
that gives sufficient control over the leather---is shown at
\protect\hyperlink{Fig_59}{fig. 59}. To use this knife properly,
practice is required. The main thing to learn is that the knife must be
used quite flat, and made to cut by having a very slight
burr{\protect\hypertarget{Page_158}{}{{[}158{]}}} on the under side.
This burr is got by rubbing the knife on the lithographic stone on which
the paring is done. The handle of the knife should never be raised to
such a height above the surface of the stone that it is possible to get
the under fingers of the right hand over the edge of the stone. Another
form of knife suitable for paring the edges of leather is shown at
\protect\hyperlink{Fig_60}{fig. 60}, B.

\protect\hypertarget{Fig_60}{}{}
\includegraphics[width=2.08333in,height=1.04167in]{images/gs159.jpg}

Fig. 60.

To test if the leather has been sufficiently pared, fold it over where
the edge of the board will come, and run the finger along the folded
leather. If the paring has been done properly it will feel quite even
the whole length of the fold; but if there are any irregularities, they
will be very apparent, and the paring must be gone over again till they
have disappeared.{\protect\hypertarget{Page_159}{}{{[}159{]}}} When
even, the book must be again laid on the leather with the boards open,
and a pencil line drawn round as before. If there are leather joints
they will have been pared before the book was sewn, and care must be
taken in paring the turn-in of the cover that it is of the same
thickness as the leather joint, or it will be impossible to make a neat
mitre at the back corners.

\hypertarget{covering}{%
\subparagraph{COVERING}\label{covering}}

Before covering, the book must be looked at to see that the bands are
quite square and at equal distances apart. Any slight errors in this
respect can be corrected by holding the book in the lying press between
backing boards and gently tapping the bands from one side or the other
with a piece of wood struck with a hammer. This is best done when the
back is cleaned off, but by damping the bands slightly it may be done
just before covering. The squares must be looked to, and the edges of
the board well rubbed with a folder, or tapped with a hammer, to remove
any burr that may have been caused by the plough knife, or any
chance{\protect\hypertarget{Page_160}{}{{[}160{]}}} blow. The back is
then moistened with paste, or, in the case of a very large book, with
thin glue, and left to soak. The cover can then be well pasted with
thickish paste, that has been previously well beaten up. When the cover
is pasted, it can be folded with the pasted sides together and left to
soak for a few minutes while the back is again looked to, and any
roughness smoothed down with the folder. Before covering, the bands
should be nipped up with band nippers (see
\protect\hyperlink{Fig_61}{fig. 61}) to make sure that they are sharp.
The coverer should have ready before covering a clean paring stone, one
or two folders, a pair of nickeled-band nippers, a clean sponge, a
little water in a saucer, a piece of thread, and a strip of smooth wood
(boxwood for preference), called a band stick, used for smoothing the
leather between the bands, a pair of scissors, and a small sharp knife,
a pair of waterproof sheets the size of the book, and, if
the{\protect\hypertarget{Page_161}{}{{[}161{]}}} book is a large one, a
pair of tying up boards, with tying up string, and two strips of wood
covered in blotting-paper or leather. It is best to have the band
nippers for covering nickeled to prevent the iron from staining the
leather. The waterproof sheets recommended are thin sheets of celluloid,
such as are used by photographers.

\protect\hypertarget{Fig_61}{}{}
\includegraphics[width=2.60417in,height=0.82292in]{images/gs161.jpg}

Fig. 61.

When these things are ready, the pasted cover should be examined and
repasted if it has dried in any place. The amount of paste to be used
for covering can only be learned by experience. A thick leather will
take more than a thin one, but, provided the cover sticks tight at every
point, the less paste used the better. If there is too much, it will rub
up and make very ugly, uneven places under the leather; and if there is
too little, the cover will not stick.

\protect\hypertarget{Fig_62}{}{}
\includegraphics[width=3.125in,height=2.11458in]{images/gs163.jpg}

Fig. 62.

Take the pasted cover and look to see which is the better side of the
leather. Lay the front of the book down on this exactly up to the marks
that show the beginning of the turn-in. Then draw the leather over the
back and on to the other side, pulling it slightly, but not dragging it.
Then stand the book on{\protect\hypertarget{Page_162}{}{{[}162{]}}} its
fore-edge on a piece of waste paper, with the leather turned out on
either side, as shown at \protect\hyperlink{Fig_62}{fig. 62}, and nip up
the bands with nickeled band nippers (see
\protect\hyperlink{Fig_63}{fig. 63}). After this is done there will
probably be a good deal of loose leather on the back. This can be got
rid of by dragging the leather on to the side; but by far the better
plan, when the back is large enough to allow it, is to work up the
surplus leather on to the back between the panels. This requires a good
deal of practice, and is very seldom done; but it can be done with most
satisfactory results. The book should now have the leather on the
back{\protect\hypertarget{Page_163}{}{{[}163{]}}} stretched lengthways
to make it cover the bands, but not stretched the other way, and the
leather on the boards should lie perfectly flat and not be stretched at
all. The leather on the fore-edge of the board is then rubbed with the
hand on the outside, and then on to the edge, and then on the inside.
The edge and the inside are smoothed down with a folder, and any
excessive paste on the{\protect\hypertarget{Page_164}{}{{[}164{]}}}
inside squeezed out and removed. When the fore-edge of both boards has
been turned in, the head and tail must also be turned in. A little paste
is put on to that part of the leather that will turn in below the
headband, and this portion is neatly tucked in between the boards and
the back. The turned-in edge must lie quite evenly, or it will result in
a ridge on the back. The leather is turned in on the two boards in the
same way as described for the fore-edge, and the edge rubbed square with
a folder. At \protect\hyperlink{Fig_64}{fig. 64} is shown a convenient
form of folder for covering. At the corners the leather must be pulled
over as far as possible with two folders meeting at the extreme point,
the object being to avoid a cut in the leather at the corner of the
board. The folds so formed must be cut off with the scissors (see
\protect\hyperlink{Fig_65}{fig. 65}, A), then one edge tucked neatly
under the other, (B). Care must be taken throughout not to soil the
edges of the leaves.

\protect\hypertarget{Fig_63}{}{}
\includegraphics[width=3.125in,height=3.375in]{images/gs164.jpg}

Fig. 63.

\protect\hypertarget{Fig_64}{}{}
\includegraphics[width=2.08333in,height=0.3125in]{images/gs165.jpg}
{Fig. 64.}

At the headband the fold of
leather,{\protect\hypertarget{Page_165}{}{{[}165{]}}} pared thin for the
purpose, must be squeezed together with a folder and pulled out a little
to leave an even projection that can be turned over to form a head-cap.
When both ends have been turned in, in this way, the boards must each be
opened and pressed against a straight-edge held in the joint
(\protect\hyperlink{Fig_66}{fig. 66}) to ensure that there is enough
leather in the turn-in of the joint to allow the cover to open freely;
and the leather of the turn-in at the head and tail
must{\protect\hypertarget{Page_166}{}{{[}166{]}}} be carefully smoothed
down with a folder.

\protect\hypertarget{Fig_65}{}{}
\includegraphics[width=3.125in,height=2.08333in]{images/gs166.jpg}

Fig. 65.

\protect\hypertarget{Fig_66}{}{}
\includegraphics[width=3.125in,height=0.90625in]{images/gs166a.jpg}

Fig. 66.

\protect\hypertarget{Fig_67}{}{}
\includegraphics[width=1.04167in,height=1.38542in]{images/gs167.jpg}

Fig. 67.

The book may now be shut up if a waterproof sheet is put at each end to
prevent the damp of the cover from cockling the paper. It must then be
stood on its fore-edge and the bands again nipped up with a pair of
nickeled band nippers, and the panels between the bands well pressed
down with the band stick to cause the leather to stick at every point. A
piece of thread is tied round the back from head to tail, squeezing the
leather in the gap caused by the corners of the board having been cut
off. The book is then turned up on end, resting the tail on a folder or
anything that will keep the projecting leather for the head-cap from
being prematurely flattened. The head-caps
(\protect\hyperlink{Fig_67}{fig. 67}) must now be set. To do this the
first finger of the left hand is placed behind it, and a sharp folder is
pressed into the corners of the head-cap between the headband and the
thread. The leather{\protect\hypertarget{Page_167}{}{{[}167{]}}} is then
tapped over the headband, and the whole turned over on the stone and
rubbed at the back with a folder. This operation requires great nicety.
The shape of head-cap is shown at \protect\hyperlink{Fig_67}{fig. 67}.
The nice adjustment of head-caps and corners, although of no
constructional value, are the points by which the forwarding of a book
is generally valued.

\protect\hypertarget{Fig_68}{}{}
\includegraphics[width=3.125in,height=2.55208in]{images/gs168.jpg}

Fig. 68.

If the book is a large one, it will be best to tie it up. The method of
tying up is shown in \protect\hyperlink{Fig_68}{fig. 68}. The tying up
cords will make marks at the side of the bands, that are not unpleasant
on a large{\protect\hypertarget{Page_168}{}{{[}168{]}}} book. If they
are objected to, it is best to tie the book up for about half-an-hour,
and then to untie it, and smooth out the marks with the band stick. Even
with small books, if the leather seems inclined to give trouble, it is
well to tie them up for a short time, then to untie them, to smooth out
any marks or inequalities, and to tie them up again.

\hypertarget{mitring-corners-and-filling-in}{%
\subparagraph{MITRING CORNERS AND FILLING
IN}\label{mitring-corners-and-filling-in}}

A book that has been covered should be left under a light weight until
the next day, with waterproof sheets between the damp cover and the end
paper to prevent the sheets of the book from cockling through the damp.
When the cover is thoroughly set the boards should be carefully opened,
pressing them slightly to the joint to ensure a square and even joint.
If, as is sometimes the case, the turn-in of the leather over the joint
seems to be inclined to bind, the cover should be merely opened
half-way, and the leather of the turns-in of the joint damped with a
sponge, and left to soak for a short time, and then the cover
can{\protect\hypertarget{Page_169}{}{{[}169{]}}} usually be opened
without any dragging. A section of a good joint is shown at
\protect\hyperlink{Fig_69}{fig. 69}, A, and a bad one at B.

\protect\hypertarget{Fig_69}{}{}
\includegraphics[width=3.125in,height=0.71875in]{images/gs170.jpg}

Fig. 69.

\protect\hypertarget{Fig_70}{}{}
\includegraphics[width=3.125in,height=0.91667in]{images/gs171.jpg}

Fig. 70.

The next operation will be to fill in the board and mitre the corners.
To fill in the boards, a piece of paper as thick as the turn-in of the
leather (engineer's cartridge paper answers very well) should be cut a
little smaller than the board, with one edge cut straight; then with the
straight edge adjusted to the back of the board, and a weight placed on
the centre, the paper is marked round with dividers set to the intended
width of the turn-in of the leather. Then with a sharp knife, paper and
leather may be cut through together. The paper should then be marked to
show its position on the board, and the ragged edges of the leather
trimmed off. This will leave an even margin of leather on three sides of
the inside of the board, and a piece of paper that will exactly fit the
remaining space. The corners must next be
mitred.{\protect\hypertarget{Page_170}{}{{[}170{]}}} To do this, both
thicknesses of leather are cut through from the corner of the board to
the corner of the inside margin. The knife should be held slightly
slanting to make a cut, as shown at \protect\hyperlink{Fig_70}{fig. 70}.
The corners should then be thoroughly damped, and the overlapping
leather from both sides removed, leaving what should be a neat and
straight join. If the leather at the extreme corner should prove to be,
as is often the case, too thick to turn in neatly, the corners should be
opened out and the leather pared against the thumb nail, and then well
pasted and turned back again. The extreme corner may be slightly tapped
on the stone with a hammer, and the sides rubbed with a folder, to
ensure squareness and sharpness. When all four corners have been mitred,
the filling in papers can be pasted in. As they will probably stretch a
little with the paste, it will be well to cut off a slight shaving, and
they should then fit exactly. When the boards have been filled in and
well rubbed{\protect\hypertarget{Page_171}{}{{[}171{]}}} down, the book
should be left for some hours with the boards standing open to enable
the filling-in papers to draw the boards slightly inwards to overcome
the pull of the leather.

In cases where there are leather joints the operation is as follows: The
waste end paper is removed, and the edge of the board and joint
carefully cleaned from glue and all irregularities, and if, as is most
likely, it is curved from the pull of the leather, the board must be
tapped or ironed down until it is perfectly straight. If there is
difficulty in making the board lie straight along the joint before
pasting down, it will be well first to fill in with a well pasted and
stretched thin paper, which, if the boards are left open, will draw them
inwards. If the leather joint is pasted down while the board is curved,
the result will be a most unsightly projection on the outside. When the
joint has been cleaned out, and the board made to lie flat, the leather
should be pasted down and mitred. The whole depth of the turn-in of the
covering leather in the joint must not be removed, or it will be unduly
weakened. The mitring line should not come from the extreme
corner,{\protect\hypertarget{Page_172}{}{{[}172{]}}} but rather farther
down, and there it is well to leave a certain amount of overlap in the
joint, for which purpose the edge of the turn-in leather and the edge of
the leather joint should be pared thin. After pasting down the leather
joints the boards should be left open till they are dry (see
\protect\hyperlink{Fig_71}{fig. 71}). The turn-in and leather joint are
then trimmed out, leaving an even margin of leather all round the inside
of the{\protect\hypertarget{Page_173}{}{{[}173{]}}} board, and the panel
in the centre filled in with a piece of thick paper.

\protect\hypertarget{Fig_71}{}{}
\includegraphics[width=3.125in,height=3.79167in]{images/gs173.jpg}

Fig. 71.

When corners and filling in are dry, the boards may be shut up, and the
book is ready for finishing.

It is a common practice to wash up the covers of books that have become
stained with a solution of oxalic acid in water. This is a dangerous
thing to do, and is likely to seriously injure the leather. Leather,
when damp, must not be brought in contact with iron or steel tools, or
it may be badly stained.

\hypertarget{chapter-xiii}{%
\subsection[CHAPTER
XIII]{\texorpdfstring{\protect\hypertarget{CHAPTER_XIII}{}{}CHAPTER
XIII}{CHAPTER XIII}}\label{chapter-xiii}}

Library Binding---Binding very Thin Books---Scrap-Books---Binding on
Vellum---Books covered with Embroidery

\hypertarget{library-binding}{%
\subparagraph{LIBRARY BINDING}\label{library-binding}}

\hypertarget{specifications-iii-and-iv}{%
\subparagraph{\texorpdfstring{\emph{Specifications III and
IV}}{Specifications III and IV}}\label{specifications-iii-and-iv}}

{To} produce cheaper bindings, as must be done in the case of large
libraries, some alteration of design is necessary. Appearance must to
some extent be sacrificed to strength and durability, and not, as is
too{\protect\hypertarget{Page_174}{}{{[}174{]}}} often the case,
strength and durability sacrificed to appearance. The essentials of any
good binding are, that the sections should be sound in themselves, and
that there should be no plates or odd sheets ``pasted on,'' or anything
that would prevent any leaf from opening right to the back; the sewing
must be thoroughly sound; the sewing materials of good quality; the
slips firmly attached to the boards; and the leather fairly thick and of
a durable kind, although for the sake of cheapness it may be necessary
to use skins with flaws on the surface. Such flawed skins cost half, or
less than half, the price of perfect skins, and surface flaws do not
injure the strength of the leather. By sewing on tape, great flexibility
of the back is obtained, and much time, and consequent expense, in
covering is saved. By using a French joint much thicker leather than
usual can be used, with corresponding gain in strength.

To bind an octavo or smaller book according to the specification given
(III, page \protect\hyperlink{Page_307}{307}); first make all sections
sound, and guard all plates or maps. Make end papers with zigzags. After
the sections have been thoroughly pressed, the book will be
ready{\protect\hypertarget{Page_175}{}{{[}175{]}}} for marking up and
sewing. In marking up for sewing on tapes, two marks will be necessary
for each tape. When there are several books of the same size to be sewn,
they may be placed one above the other in the sewing press, and sewn on
to the same tapes. It will be found that the volumes when sewn can
easily be slid along the tapes, which must be long enough to provide
sufficient for the slips of each. The split boards may be ``made'' of a
thin black mill-board with a thicker straw-board. To ``make'' a pair of
split boards the pieces of straw-and mill-board large enough to make the
two are got out, and the straw-board well glued, except in the centre,
which should previously be covered with a strip of thin mill-board or
tin about four inches wide. The strip is then removed, and the thin
black board laid on the glued straw-board and nipped in the press. When
dry, the made board is cut down the centre, which will leave two boards
glued together all over except for two inches on one side of each. The
boards then are squared to the book in a mill-board machine. The back of
the book is glued up, and in the ordinary way rounded and backed. The
edges may be{\protect\hypertarget{Page_176}{}{{[}176{]}}} cut with a
guillotine. The ends of the tapes are glued on the waste end paper,
which should be cut off about an inch and a half from the back. The
split boards are then opened and glued, and the waste end papers with
slips attached are placed in them (see \protect\hyperlink{Fig_72}{fig.
72}), and the book nipped in the press. To form a ``French joint'' the
boards should be kept about an eighth of an inch from the back of the
book. The book is then ready for covering. The leather must not be pared
too thin, as the French joint will give plenty of play and allow the use
of much thicker leather than usual. If time and money can be spared,
headbands can be worked, but they are not absolutely necessary, and a
piece of string may be inserted into the turning of the leather at head
and tail in the place of them. When the book
is{\protect\hypertarget{Page_177}{}{{[}177{]}}} covered, a piece of
string should be tied round the joints, and the whole given a nip in the
press. The corners of the boards should be protected by small tips of
vellum or parchment. The sides may be covered with good paper, which
will wear quite as well as cloth, look better, and cost less.

\protect\hypertarget{Fig_72}{}{}
\includegraphics[width=3.125in,height=1.30208in]{images/gs177.jpg}

Fig. 72.

The lettering of library books is very important (see
\protect\hyperlink{CHAPTER_XV}{Chapter XV}).

\hypertarget{binding-very-thin-books}{%
\subparagraph{BINDING VERY THIN BOOKS}\label{binding-very-thin-books}}

Books consisting of only one section may be bound as follows:---A sheet
of paper to match the book, and two coloured sheets for end papers, are
folded round the section, and a ``waste'' paper put over all. A strip of
linen is pasted to the back of the waste, and the whole sewn together by
stitching through the fold. The waste may be cut off and inserted with
the linen in a split board, as for library bindings. The back edges of
the board should be filed thin, and should not be placed quite up to the
back, to allow for a little play in the joints.

The leather is put on in the
ordinary{\protect\hypertarget{Page_178}{}{{[}178{]}}} way, except that
the linen at the head and tail must be slit a little to allow for the
turn in. If waterproof sheets are first inserted, the ends may be
pasted, the boards shut, and the book nipped in the press. By
substituting a piece of thin leather for the outside coloured paper, a
leather joint can be made.

\hypertarget{scrap-books}{%
\subparagraph{SCRAP-BOOKS}\label{scrap-books}}

Scrap-books, into which autograph letters, sketches, or other papers can
be pasted, may be made as follows:---Enough paper of good quality is
folded up to the size desired, and pieces of the same paper, of the same
height, and about two inches wide, are folded down the centre and
inserted between the backs of the larger sheets, as shown at
\protect\hyperlink{Fig_73}{fig. 73}. It is best not to insert these
smaller pieces in the centre of the section, as they would be
troublesome in sewing. If, after sewing, the book is filled up with
waste paper laid between the leaves, it will make it manageable while
being forwarded.

It is best to use a rather darkly-toned or coloured paper, as, if a
quite white paper is used, any letters or papers
that{\protect\hypertarget{Page_179}{}{{[}179{]}}} have become soiled,
will look unduly dirty.

\protect\hypertarget{Fig_73}{}{}
\includegraphics[width=2.08333in,height=0.5625in]{images/gs180.jpg}

Fig. 73.

Autograph letters may be mounted in the following ways:---If the letter
is written upon both sides of a single leaf, it may be either
``inlaid,'' or guarded, as shown at \protect\hyperlink{Fig_74}{fig. 74},
A. A letter on a folded sheet of notepaper should have the folds
strengthened with a guard of strong thin paper, and be attached by a
guard made, as shown at \protect\hyperlink{Fig_74}{fig. 74}, B; or if on
very heavy paper, by a double guard, as shown at
\protect\hyperlink{Fig_74}{fig. 74}, C. Torn edges of letters may be
strengthened with thin Japanese paper.

\protect\hypertarget{Fig_74}{}{}
\includegraphics[width=3.125in,height=0.6875in]{images/gs180a.jpg}

Fig. 74.

Thin paper, written or printed only on one side, may be mounted on a
page of the book. It is better to attach these
by{\protect\hypertarget{Page_180}{}{{[}180{]}}} their extreme edges
only, as if pasted down all over they may cause the leaves to curl up.

Letters or any writing or drawing in lead pencil should be fixed with
size before being inserted.

Silver prints of photographs are best mounted with some very
quick-drying paste, such as that sold for the purpose by the
photographic dealers. If the leaf on which they are mounted is slightly
damped before the photograph is pasted down, it will be less likely to
cockle. If this is done, waterproof sheets should be put on each side of
the leaf while it dries. If photographs are attached by the edges only,
they will not be so liable to draw the paper on which they are mounted;
but sometimes they will not lie flat themselves.

In cases where very thick letters or papers have to be pasted in, a few
more leaves of the book should be cut out, to make a corresponding
thickness at the back.

\hypertarget{vellum-bindings}{%
\subparagraph{VELLUM BINDINGS}\label{vellum-bindings}}

Vellum covers may be limp without boards, and merely held in place by
the{\protect\hypertarget{Page_181}{}{{[}181{]}}} slips being laced
through them, or they may be pasted down on boards in much the same way
as leather.

If the edges of a book for limp vellum binding are to be trimmed or
gilt, that should be done before sewing. For the ends a folded piece of
thin vellum may replace the paste-down paper. The sewing should be on
strips of vellum. The back is left square after glueing, and headbands
are worked as for leather binding, or may be worked on strips of
leather, with ends left long enough to lace into the vellum (see p.
\protect\hyperlink{Page_151}{151}). The back and headbands are lined
with leather, and the book is ready for the cover.

A piece of vellum should be cut out large enough to cover the book, and
to leave a margin of an inch and a half all round. This is marked with a
folder on the under side, as shown at \protect\hyperlink{Fig_75}{fig.
75}, A. Spaces 1 and 2 are the size of the sides of the book with
surrounding squares; space 3 is the width of the back, and space 4 the
width for the overlaps on the fore-edge. The corners are cut, as shown
at 5, and the edges are folded over, as at B. The overlap 4 is then
turned over, and the back folded, as at C. The slips are
now{\protect\hypertarget{Page_182}{}{{[}182{]}}} laced through slits
made in the vellum.

\protect\hypertarget{Fig_75}{}{}
\includegraphics[width=3.125in,height=4.55208in]{images/gs183.jpg}

Fig. 75.

A piece of loose, toned paper may be
put{\protect\hypertarget{Page_183}{}{{[}183{]}}} inside the cover to
prevent any marks on the book from showing through; and pieces of silk
ribbon of good quality are laced in as shown, going through both cover
and vellum ends, if there are any, and are left with ends long enough to
tie (see \protect\hyperlink{Fig_76}{fig. 76}).

\protect\hypertarget{Fig_76}{}{}
\includegraphics[width=1.5625in,height=1.3125in]{images/gs184.jpg}

Fig. 76.

If paper ends are used, the silk tape need only be laced through the
cover, and the end paper pasted over it on the inside.

Another simple way of keeping a vellum book shut is shown at
\protect\hyperlink{Fig_77}{fig. 77}. A bead is attached to a piece of
gut laced into the vellum, and a loop of catgut is laced in the other
side, and looped over the bead as shown.

If the book is to have stiff boards,
and{\protect\hypertarget{Page_184}{}{{[}184{]}}} the vellum is to be
pasted to them, it is best to sew the sections on tapes or vellum slips,
to back the book as for leather, and to insert the ends of the slips in
a split board, leaving a French joint, as described for library
bindings. Vellum is very stiff, and, if it is pasted directly to the
back, the book would be hard to open. It is best in this case to use
what is known as a hollow back.

\protect\hypertarget{Fig_77}{}{}
\includegraphics[width=0.78125in,height=0.94792in]{images/gs185.jpg}

Fig. 77.

To make a hollow back, a piece of stout paper is taken which measures
once the length of the back and three times the width. This is folded in
three. The centre portion is glued to the back and well rubbed down, and
the overlapping edges turned back and glued one to the other
(\protect\hyperlink{Fig_78}{fig. 78}). This will leave a flat, hollow
casing, formed by the single paper glued to the back of the book and the
double paper to which the vellum may be attached. Or it is better to
line up the back with leather, and to place a piece of thick paper the
size of the back on to the pasted vellum where the back will be when the
book is covered.

\protect\hypertarget{Fig_78}{}{}
\includegraphics[width=1.04167in,height=2.15625in]{images/gs186.jpg}

Fig. 78.

When the book is ready for covering, the vellum should be cut out and
lined with paper. In lining vellum the
paste{\protect\hypertarget{Page_185}{}{{[}185{]}}} must be free from
lumps, and great care must be taken not to leave brush marks. To avoid
this, when the lining paper has been pasted it can be laid, paste
downwards, on a piece of waste paper and quickly pulled up again; this
should remove surplus paste and get rid of any marks left by the brush.
When the vellum has been lined with paper, it should be given a light
nip in the press between blotting-paper, and while still damp it is
pasted, the book covered, and the corners mitred. A piece of thin string
is tied round the head-caps and pressed into the French joint.

Waterproof sheets are placed inside the covers, and the book then nipped
in the press and left to dry under a light weight. If the vellum is very
stiff and difficult to turn in, it may be moistened with a little warm
water to soften it.

Books with raised bands have sometimes been covered with vellum, but the
back{\protect\hypertarget{Page_186}{}{{[}186{]}}} becomes so stiff and
hard, that this method, though it looks well enough, cannot be
recommended. Vellum is a durable material, and can be had of good
quality, but it is so easily influenced by changes of temperature, that
it is rather an unsuitable material for most bindings.

\hypertarget{books-covered-with-embroidery-and-woven-material}{%
\subparagraph{BOOKS COVERED WITH EMBROIDERY AND WOVEN
MATERIAL}\label{books-covered-with-embroidery-and-woven-material}}

To cover a book with embroidered material bind it with split boards, a
French joint, and a hollow back, as described for vellum (see
\protect\hyperlink{Fig_78}{fig. 78}). Glue the back of the book with
thin glue well worked up, and turning in the head and tail of the
embroidery, put the book down on it so that the back will come exactly
in the right place. Press down the embroidery with the hand to make sure
that it sticks. When it is firmly attached to the back, first one board
and then the other should be glued, and the embroidery laid down on it.
Lastly, the edges are glued and stuck down on the inside of the board,
and the corners mitred. Velvet or any other thick material can be put
down in the same way. For very thin
material{\protect\hypertarget{Page_187}{}{{[}187{]}}} that the glue
would penetrate and soil, the cover should be left loose, and only
attached where it turns in. A loose lining of good paper may be put
between the book and the cover.

The inside corners where the cover has been cut should be neatly sewn
up. The edges of the boards and head-caps may be protected all round
with some edging worked in metal thread. It is well in embroidering book
covers to arrange for some portion of the pattern to be of raised metal
stitches, forming bosses that will protect the surface from wear.

Should any glue chance to get on the surface, the cover should be held
in the steam of a kettle and the glue wiped off, and the cover again
steamed.

\hypertarget{chapter-xiv188}{%
\subsection[CHAPTER
XIV]{\texorpdfstring{\protect\hypertarget{CHAPTER_XIV}{}{}CHAPTER
XIV{\protect\hypertarget{Page_188}{}{{[}188{]}}}}{CHAPTER XIV{[}188{]}}}\label{chapter-xiv188}}

Decoration---Tools---Finishing---Tooling on Vellum---Inlaying on Leather

\hypertarget{decoration-of-bindingtools}{%
\subparagraph{DECORATION OF
BINDING---TOOLS}\label{decoration-of-bindingtools}}

{The} most usual, and perhaps the most characteristic, way of decorating
book covers is by ``tooling.'' Tooling is the impression of heated
(finishing) tools. Finishing tools are stamps of metal that have a
device cut on the face, and are held in wooden handles
(\protect\hyperlink{Fig_79}{fig. 79}).

\protect\hypertarget{Fig_79}{}{}
\includegraphics[width=3.125in,height=0.5625in]{images/gs189.jpg}

Fig. 79.

Tooling may either be blind tooling, that is, a simple impression of the
hot tools, or gold tooling, in which the impression of the tool is left
in gold on the leather.

Tools for blind tooling are best ``die-sunk,'' that is, cut like a seal.
The ``sunk'' part of the face of the tool, which may be more or less
modelled, forms the pattern, and the higher
part{\protect\hypertarget{Page_189}{}{{[}189{]}}} depresses the leather
to form a ground. In tools for gold tooling, the surface of the tool
gives the pattern.

Tools may be either complex or simple in design, that is to say, each
tool may form a complete design with enclosing border, as the lower ones
on page \protect\hyperlink{Page_323}{323}, or it may be only one element
of a design, as at \protect\hyperlink{Fig_100}{fig. 100}. Lines may be
run with a fillet (see \protect\hyperlink{Fig_88}{fig. 88}), or made
with gouges or pallets.

Gouges are curved line tools. They are made in sets of arcs of
concentric circles (see \protect\hyperlink{Fig_80}{fig. 80}, A). The
portion of the curves cut off by the dotted line C will make a second
set with flatter curves. Gouges are used for tooling curved lines.

\protect\hypertarget{Fig_80}{}{}
\includegraphics[width=2.08333in,height=3.10417in]{images/gs190.jpg}

Fig. 80.

A ``pallet'' may be described as a segment of a roll or fillet set in a
handle, and{\protect\hypertarget{Page_190}{}{{[}190{]}}} used chiefly
for putting lines or other ornaments across the backs of books (see
\protect\hyperlink{Fig_81}{fig. 81}). A set of one-line pallets is shown
at \protect\hyperlink{Fig_80}{fig. 80}, B.

Fillets are cut with two or more lines on the edge. Although the use of
double-line fillets saves time, I have found that a few single-line
fillets with edges of different gauges are sufficient for running all
straight lines, and that the advantage of being able to alter the
distances between any parallel lines is ample compensation for the extra
trouble involved by their use. In addition to the rigid stamps, an
endless pattern for either blind or gold tooling may be engraved on the
circumference of a roll, and impressed on the leather by wheeling.

\protect\hypertarget{Fig_81}{}{}
\includegraphics[width=0.78125in,height=1.66667in]{images/gs191.jpg}

Fig. 81.

The use of a roll in finishing dates from the end of the fifteenth
century, and some satisfactory bindings were decorated with its aid. The
ease with which it can be used has led in modern times to its abuse, and
I hardly know of a single instance of a modern binding on which rolls
have been used for the decoration with satisfactory results. The gain in
time and trouble is at the expense of freedom and life in
the{\protect\hypertarget{Page_191}{}{{[}191{]}}} design; and for extra
binding it is better to build up a pattern out of small tools of simple
design, which can be arranged in endless variety, than to use rolls.

Tools for hand-tooling must not be too large, or it will be impossible
to obtain clear impressions. One inch square for blind tools, or
three-quarters of an inch for gold tools, is about the maximum size for
use with any certainty and comfort. Tools much larger than this have to
be worked with the aid of a press, and are called blocks.

\hypertarget{finishing}{%
\subparagraph{FINISHING}\label{finishing}}

The first thing the finisher does to a book is to go over the back with
a polisher and smooth out any irregularities.

Two forms of polisher are shown at \protect\hyperlink{Fig_82}{fig. 82}.
The lower one is suitable for polishing backs and inside margins, and
the upper for sides. Polishers must be used warm, but not too hot, or
the leather may be scorched, and they must be kept moving on the
leather. Before using they should be rubbed bright on a piece of the
finest emery paper, and polished on a piece of leather.
New{\protect\hypertarget{Page_192}{}{{[}192{]}}} polishers often have
sharp edges that would mark the leather. These must be rubbed down with
files and emery-paper.

Leathers with a prominent grained surface, such as morocco, seal or pig
skin, may either have the grain rough or crushed flat. If there is to be
much finishing, the grain had better be crushed, but for large books
that are to have only a small amount of finishing, the grain is best
left unflattened.

\protect\hypertarget{Fig_82}{}{}
\includegraphics[width=3.125in,height=1.07292in]{images/gs193.jpg}

Fig. 82.

If the grain of the leather is to be ``crushed,'' it may be done at this
stage. To do this, one board at a time is damped with a sponge and put
in the standing-press, with a pressing plate on the grained side, and a
pad of blotting-paper, or some such yielding substance, on the other
(see \protect\hyperlink{Fig_83}{fig. 83}). The press is then screwed up
tight, and the board left for a
short{\protect\hypertarget{Page_193}{}{{[}193{]}}} time. For some
leathers this operation is best done after the binding has been finished
and varnished, in which case, of course, the boards cannot be damped
before pressing. No flexibly sewn book should be subject to great
pressure after it has been covered, or the leather on the back may
crinkle up and become detached.

The next thing will be to decide what lettering and what decoration, if
any, is to be put on the volume. The lettering should be made out first
(see page \protect\hyperlink{Page_215}{215}). If the book is to be at
all elaborately decorated, paper patterns must be made out, as described
in Chapter XVI.

\protect\hypertarget{Fig_83}{}{}
\includegraphics[width=1.5625in,height=1.72917in]{images/gs194.jpg}

Fig. 83.

For tooling the back, the book is held in the finishing press between a
pair of backing boards lined with leather (see
\protect\hyperlink{Fig_84}{fig. 84}), and the paper pattern put across
the back, with the ends either slightly pasted to the backing boards, or
caught between them and the book.

For the sides, the pattern is very slightly pasted on to the leather at
the four{\protect\hypertarget{Page_194}{}{{[}194{]}}} corners. The book
is then put in the finishing press, with the board to be tooled open and
flat on the cheek of the press, unless the book is a large one, when it
is easier to tool the sides out of the press.

\protect\hypertarget{Fig_84}{}{}
\includegraphics[width=3.125in,height=1.4375in]{images/gs195.jpg}

Fig. 84.

The selected tools, which should be ready on the stove (see
\protect\hyperlink{Fig_85}{fig. 85}), are one at a time cooled on a wet
pad, and then pressed in their former impressions upon the paper. The
degree of heat required varies a good deal with the leather used, and
will only be learned by experience. It is better to have the tool too
cool than{\protect\hypertarget{Page_195}{}{{[}195{]}}} too hot, as it is
easy to deepen impressions after the paper is removed; but if they are
already too deep, or are burnt, it will be impossible to finish clearly.
Generally speaking, tools should hiss very slightly when put on the
cooling pad. In cooling, care must be taken to put the shank of the
tools on to the wet pad, as, if the
end{\protect\hypertarget{Page_196}{}{{[}196{]}}} only is cooled, the
heat is apt to run down again, and the tool will still be too hot.

\protect\hypertarget{Fig_85}{}{}
\includegraphics[width=2.08333in,height=3.125in]{images/gs196.jpg}

Fig. 85.---Finishing Stove

Before removing the paper, one corner at a time should be lifted up, and
the leather examined to see that no part of the pattern has been missed.

In some patterns where the design is close, or in which the background
is dotted in, it will not be necessary to blind in every leaf and dot
through the paper. If the lines with perhaps the terminal leaves are
blinded in, the rest can be better worked directly through the gold.
This method implies the ``glairing in'' of the whole surface. It is not
suitable for open patterns, where the glaire might show on the surface
of the leather.

If the book is only to have lines, or some simple straight line pattern,
it is often easier to mark it up without the paper, with a straight-edge
and folder. In panelling a back, the side lines of all the panels should
be marked in at the same time with a folder, working against the
straight-edge, held firmly at the side of the back. If the panels are
worked separately, it is difficult to get the side lines squarely above
each other. The lines at{\protect\hypertarget{Page_197}{}{{[}197{]}}}
the top and bottom of the panel may be marked in with a folder, guided
by a piece of stiff vellum held squarely across the back. If there are
lines to be run round the board, they can be marked in with a pair of
dividers guided by the edge of the board, except those at the back.
These must be measured from the fore-edge of the board and run in with
straight-edge and folder.

When straight lines occur in patterns that are blinded through the
paper, it will be enough if the ends only are marked through with a
small piece of straight line, and the lines completed with straight-edge
and folder, after the paper has been removed.

Unless the finisher has had considerable experience, it is best to
deepen all folder lines by going over them in blind with a fillet or
piece of straight line.

When the pattern has been worked in blind, either through a paper
pattern or directly on to the leather with the tools, and any inlays
stuck on (see page \protect\hyperlink{Page_213}{213}), the cover should
be well washed with clean water. Some finishers prefer to use common
vinegar or diluted acetic acid for washing up books. If vinegar is used
it{\protect\hypertarget{Page_198}{}{{[}198{]}}} must be of the best
quality, and must not contain any sulphuric acid. Cheap, crude vinegar
is certain to be injurious to the leather. Porous leather, such as calf
or sheep skin, will need to be washed over with paste-water, and then
sized.

Paste-water is paste and water well beaten up to form a milky liquid,
and is applied to the leather as evenly as possible with a sponge. When
the paste-water is dry, the leather should be washed with size. Size can
be made by boiling down vellum cuttings, or by dissolving gelatine or
isinglass in warm water.

For the less porous leathers, such as morocco, seal, or pig skin, no
paste-water or size is necessary, unless the skin happens to be a
specially open one, or the cover has been cut from the flank or belly.
Then it is best to put a little paste in the vinegar or water used for
washing up. When the leather is nearly, but not quite, dry the
impressions of the tools must be painted with glaire. Finishers' glaire
may be made from the white of eggs well beaten up, diluted with about
half as much vinegar, and allowed to settle. Some finishers prefer to
use old, evil-smelling glaire, but provided it is a day old, and has
been well{\protect\hypertarget{Page_199}{}{{[}199{]}}} beaten up, fresh
glaire will work quite well.

The impressions of any heavy or solid tools should be given a second
coat of glaire when the first has ceased to be ``tacky,'' and if the
leather is at all porous, all impressions had better have a second coat.

As glaire is apt to show and disfigure the leather when dry, it is best
to use it as sparingly as possible, and, excepting where the pattern is
very close, to confine it to the impressions of the tools. It is not at
all an uncommon thing to see the effect of an otherwise admirably tooled
binding spoilt by a dark margin round the tools, caused by the careless
use of glaire. Glaire should not be used unless it is quite liquid and
clean. Directly it begins to get thick it should be strained or thrown
away.

The finisher should not glaire in more than he can tool the same day.
When the glaire has ceased to be ``tacky,'' the gold is laid on.

\protect\hypertarget{Fig_86}{}{}
\includegraphics[width=3.125in,height=0.94792in]{images/gs201.jpg}

Fig. 86.

At first it will be found difficult to manage gold leaf. The essential
conditions are, that there should be no draught, and that the cushion
and knife should be quite free from grease. The gold
cushion{\protect\hypertarget{Page_200}{}{{[}200{]}}} and knife are shown
at \protect\hyperlink{Fig_86}{fig. 86}. A little powdered bath-brick
rubbed into the cushion will make it easier to cut the gold cleanly. The
blade of the gold knife should never be touched with the hand, and
before using it, both sides should be rubbed on the cushion. A book of
gold is laid open on the cushion, and a leaf of gold is lifted up on the
gold knife, which is slipped under it, and turned over on to the
cushion. A light breath exactly in the centre of the sheet should make
it lie flat, when it may be cut into pieces of any size with a slightly
sawing motion of the knife. The book with the pattern ready prepared,
and the glaire sufficiently dry (not sticky), is rubbed lightly with a
small piece of cotton-wool greased with a little cocoanut oil. The back
of the hand is greased in the same way, and a pad of clean cotton-wool
is held in the right hand, and having been made as flat as possible
by{\protect\hypertarget{Page_201}{}{{[}201{]}}} being pressed on the
table, is drawn over the back of the hand. This should make it just
greasy enough to pick up the gold, but not too greasy to part with it
readily when pressed on the book. As little grease as possible should be
used on the book, as an excess is apt to stain the leather and to make
the gold dull. After experiment it has been found that cocoanut oil
stains the leather less than any other grease in common use by
bookbinders, and is more readily washed out by benzine.

\protect\hypertarget{Fig_87}{}{}
\includegraphics[width=3.125in,height=1.45833in]{images/gs202.jpg}

Fig. 87.

If the gold cracks, or is not solid when pressed on the book, a second
thickness should be used. This will stay down if the under piece is
lightly breathed upon.

For narrow strips of gold for lines, a little pad covered with soft
leather may be made, as in \protect\hyperlink{Fig_87}{fig. 87}.

It will be found of advantage to first use the bottom leaf of gold in
the book{\protect\hypertarget{Page_202}{}{{[}202{]}}} and then to begin
at the top and work through, or else the bottom leaf will almost
certainly be found to be damaged by the time it is reached. The gold
used should be as nearly pure as it can be got. The gold-beaters say
that they are unable to beat pure gold as thin as is usual for gold
leaf; but the quite pure gold is a better colour than when alloyed, and
the additional thickness, although costly, results in a more solid
impression of the tools.

The cost of a book of twenty-four leaves three and a half inches square
of English gold leaf of good ordinary quality is from 1s. 3d. to 1s.
6d., whereas the cost of a book of double thick pure gold leaf is 3s. to
3s. 6d. For tooled work it is worth paying the increased price for the
sake of the advantages in colour and solidity; but for lines and edges,
which use up an immense amount of gold, the thinner and cheaper gold may
quite well be used.

Besides pure gold leaf, gold alloyed with various metals to change its
colour can be had. None of the alloys keep their colour as well as pure
gold, and some of them, such as those alloyed with copper for red gold,
and with silver for pale gold,
tarnish{\protect\hypertarget{Page_203}{}{{[}203{]}}} very quickly. These
last are not to be recommended.

For silver tooling aluminium leaf may be used, as silver leaf tarnishes
very quickly.

When the gold is pressed into the impressions of the tools with the pad
of cotton-wool, they should be plainly visible through it.

The pattern must now be worked through the gold with the hot tools. The
tools are taken from the stove, and if too hot cooled on a pad as for
blinding-in. The heat required to leave the gold tooling solid and
bright and the impressions clear will vary for different leathers, and
even for different skins of the same leather. For trial a tool may be
laid on the pad until it ceases to hiss, and one or two impressions
worked with it. If the gold fails to stick, the heat may be slightly
increased.

If the leather is slightly damp from the preparation the tools will
usually work better, and less heat is required than if it has been
prepared for some time and has got dry.

Before using, the faces of all tools must be rubbed bright on the flesh
side of a piece of leather. It is impossible to
tool{\protect\hypertarget{Page_204}{}{{[}204{]}}} brightly with dirty
tools. A tool should be held in the right hand, with the thumb on the
top of the handle, and steadied with the thumb or first finger of the
left hand. The shoulder should be brought well over the tool, and the
upper part of the body used as a press. If the weight of the body is
used in finishing, the tools can be worked with far greater firmness and
certainty, and with less fatigue, than if the whole work is done with
the muscles of the arms.

Large and solid tools will require all the weight that can be put on
them, and even then the gold will often fail to stick with one
impression. Tools with small surfaces, such as gouges and dots, must not
be worked too heavily, or the surface of the leather may be cut.

To strike a large or solid tool, it should first be put down flat, and
then slightly rocked from side to side and from top to bottom, but must
not be twisted on the gold.

A tool may be struck from whichever side the best ``sight'' can be got,
and press and book turned round to the most convenient position.

It is difficult to impress some
tools,{\protect\hypertarget{Page_205}{}{{[}205{]}}} such as circular
flower tools, twice in exactly the same place. Such tools should have a
mark on one side as a guide. This should always be kept in the same
position when blinding-in and tooling, and so make it possible to
impress a second time without ``doubling.'' An impression is said to be
``doubled'' when the tool has been twisted in striking, or one
impression does not fall exactly over the other.

The hot tool should not be held hovering over the impression long, or
the preparation will be dried up before the tool is struck. Tooling will
generally be brighter if the tools are struck fairly sharply, and at
once removed from the leather, than if they are kept down a long time.

To ``strike'' dots, the book should be turned with the head to the
worker, and the tool held with the handle inclining slightly towards
him. This will make them appear bright when the book is held the right
way up.

Gouges must be ``sighted'' from the inside of the curve, and struck
evenly, or the points may cut into the leather. Short straight lines may
be put in with pieces of line, and longer ones with a
fillet.{\protect\hypertarget{Page_206}{}{{[}206{]}}}

A one line fillet is shown at \protect\hyperlink{Fig_88}{fig. 88}; the
space filed out of the circumference is to enable lines to be joined
neatly at the corners. That the lines may be clearly visible through the
gold, the book should be placed so that the light comes from the left
hand of the worker and across the line. It is well to have a basin of
water in which to cool fillets, as there is so much metal in them, that
the damp sponge or cotton used for cooling tools would very rapidly be
dried up. When the fillet has been cooled, the edge should be rubbed on
the cleaning pad, and the point exactly adjusted to the corner of the
line to be run (see \protect\hyperlink{Fig_88}{fig. 88}). The fillet is
then run along the line with even pressure.

\protect\hypertarget{Fig_88}{}{}
\includegraphics[width=3.125in,height=1.54167in]{images/gs207.jpg}

Fig. 88.

For slightly curved lines, a very small fillet may be
used.{\protect\hypertarget{Page_207}{}{{[}207{]}}}

When all the prepared part of a pattern has been tooled, it is well
rubbed to remove the loose gold with a slightly greasy rag, or with a
piece of bottle indiarubber which has been softened in paraffin. After a
time the rubber or rag may be sold to the gold-beater, who recovers the
gold. To prepare indiarubber for cleaning off gold, a piece of bottle
rubber is cut into small pieces and soaked in paraffin for some hours.
This should cause the pieces to reunite into a soft lump. This can be
used until it is yellow with gold throughout.

When all free gold is rubbed off, the finisher can see where the tooling
is imperfect. Impressions which are not ``solid'' must be reglaired,
have fresh gold laid on, and be retooled. But if, as will sometimes
happen with the best finishers, the gold has failed to stick properly
anywhere, it is best to wash the whole with water or vinegar, and
prepare afresh.

As an excess of grease is apt to dull the gold and soil the leather, it
is better to use it very sparingly when laying on fresh gold for
mending. For patching, benzine may be used instead of grease. When the
gold is picked up on the cotton-wool pad, rapidly go over
the{\protect\hypertarget{Page_208}{}{{[}208{]}}} leather with wool
soaked in benzine, and at once lay down the gold. Benzine will not hold
the gold long enough for much tooling, but it will answer for about
half-an-hour, and give plenty of time for patching.

Imperfect tooling arises from a variety of causes. If an impression is
clear, but the gold not solid, it is probably because the tool was not
hot enough, or was not put down firmly. If only one side of an
impression fails to stick, it is usually because the tool was unevenly
impressed. If an impression is blurred, and the gold has a frosted look,
it is because the leather has been burned, either because the tool was
too hot, or kept down too long, or the preparation was too fresh.

To mend double or burnt impressions the leather should be wetted and
left to soak a short time, and the gold can be picked out with a wooden
point. When nearly dry the impressions should be put in again with a
cool tool, reglaired and retooled.

It is very difficult to mend neatly if the leather is badly burnt.
Sometimes it may be advisable to paste a piece of new leather over a
burnt impression before
retooling.{\protect\hypertarget{Page_209}{}{{[}209{]}}}

If a tool is put down in the wrong place by mistake, it is difficult to
get the impression out entirely. The best thing to do is to damp the
leather thoroughly, leave it to soak for a little while, and pick up the
impression with the point of a pin. It is best not to use an iron point
for this, as iron is apt to blacken the leather.

Leather is difficult to tool if it has not a firm surface, or if it is
too thin to give a little when the tool is struck.

When the tooling is finished, and the loose gold removed with the
rubber, the leather should be washed with benzine, to remove any grease
and any fragments of gold that may be adhering by the grease only.

The inside margins of the boards are next polished and varnished, and
the end papers pasted down. Or if there is a leather joint, the panel
left on the board may be filled in (see
\protect\hyperlink{CHAPTER_XVII}{Chapter XVII}).

When the end papers are dry, the sides and back may be polished and
varnished.

It is important that the varnish should be of good quality, and not too
thick, or it will in time turn brown and cause the gold to look dirty.
Some of the light French spirit varnishes prepared for bookbinders
answer well. Varnish must
be{\protect\hypertarget{Page_210}{}{{[}210{]}}} used sparingly, and is
best applied with a pad of cotton-wool. A little varnish is poured on to
the pad, which is rubbed on a piece of paper until it is seen that the
varnish comes out thinly and evenly. It is then rubbed on the book with
a spiral motion. The quicker the surface is gone over, provided every
part is covered, the better. Varnish will not work well if it is very
cold, and in cold weather both the book and varnish bottle should be
slightly warmed before use. Should an excess of varnish be put on in
error, or should it be necessary to retool part of the book after it has
been varnished, the varnish can be removed with spirits of wine. Varnish
acts as a preservative to the leather, but has the disadvantage, if used
in excess, of making it rather brittle on the surface. It must,
therefore, be used very sparingly at the joints. It is to be hoped that
a perfectly elastic varnish, that will not tarnish the gold, will soon
be discovered.

As soon as the varnish is dry the boards may be pressed, one at a time,
to give the leather a smooth surface (see
\protect\hyperlink{Fig_83}{fig. 83}), leaving each board in the press
for some hours.

\protect\hypertarget{Fig_89}{}{}
\includegraphics[width=2.60417in,height=4.0625in]{images/gs212.jpg}

Fig. 89.

After each board has been pressed separately the book should be shut,
and pressed{\protect\hypertarget{Page_211}{}{{[}211{]}}} again with
pressing plates on each side of it, and with tins covered with paper
placed inside each board. Light pressure should be given to books with
tight backs, or the leather may become detached.

If, on removing from the press, the boards will not keep shut, the book
should{\protect\hypertarget{Page_212}{}{{[}212{]}}} be pressed again
with a folded sheet of blotting-paper in each end. The blotting-paper
should have the folded edge turned up, and be placed so that this
turned-up edge will be in the joint behind the back edge of the board
when the book is shut.

A small nipping-press suitable for giving comparatively light pressure,
is shown at \protect\hyperlink{Fig_89}{fig. 89}.

\hypertarget{tooling-on-vellum}{%
\subparagraph{TOOLING ON VELLUM}\label{tooling-on-vellum}}

Most covering vellum has a sticky surface, that marks if it is handled.
This should be washed off with clean water before tooling. The pattern
is blinded in through the paper as for leather, excepting that the paper
must not be pasted directly to the vellum, but may be held with a band
going right round the board or book. It is best to glaire twice, and to
lay on a small portion of gold at a time
with{\protect\hypertarget{Page_213}{}{{[}213{]}}} benzine. As vellum
burns very readily, the tools must not be too hot, and some skill is
needed to prevent them from slipping on the hard surface.

Vellum must not be polished or varnished.

\hypertarget{inlaying-on-leather}{%
\subparagraph{INLAYING ON LEATHER}\label{inlaying-on-leather}}

Inlaying or onlaying is adding a different leather from that of the
cover, as decoration. Thus on a red book, a panel or a border, or other
portion, may be covered with thin green leather, or only flowers or
leaves may be inlaid, while a jewel-like effect may be obtained by dots,
leaves, and flowers, tooled over inlays of various colours. Leather for
inlaying should be pared very thin. To do this the leather is cut into
strips, wetted, and pared on a stone with a knife shaped somewhat as at
\protect\hyperlink{Fig_60}{fig. 60}, B. When the thin leather is dry the
inlays of the leaves and flowers, \&c., may be stamped out with steel
punches cut to the shape of the tools; or if only a few inlays are
needed, the tools may be impressed on the thin leather, and the inlays
cut out with a sharp knife. The edges of the
larger{\protect\hypertarget{Page_214}{}{{[}214{]}}} inlays should be
pared round carefully. For inlaying a panel or other large surface, the
leather is pared very thin and evenly with a French knife, and a piece
of paper pasted on to the grained side and left to dry. When dry, the
shape of the panel, or other space to be inlaid, is marked on it through
the paper pattern, and leather and paper cut through to the shape
required. The edges must then be carefully pared, and the piece attached
with paste, and nipped in the press to make it stick. When the paste is
dry, the paper may be damped and washed off. The object of the paper is
to prevent the thin leather from stretching when it is pasted.

For white inlays it is better to use Japanese paper than leather, as
white leather, when pared very thin, will show the colours of the under
leather through, and look dirty. If paper is used, it should be sized
with vellum size before tooling.

When many dots or leaves are to be inlaid, the pieces of leather, cut
out with the punch, may be laid face downwards on a paring stone, and a
piece of paper, thickly covered with paste, laid on it. This, on being
taken up, will carry with{\protect\hypertarget{Page_215}{}{{[}215{]}}}
it the ``inlays,'' and they can be picked up one at a time on the point
of a fine folder, and stuck on the book.

``Inlays'' of tools are attached after the pattern has been ``blinded''
in, and must be again worked over with the tool, in blind, when the
paste is nearly dry.

On vellum an effect, similar to that of inlays on leather, can be
obtained by the use of stains.

\hypertarget{chapter-xv}{%
\subsection[CHAPTER
XV]{\texorpdfstring{\protect\hypertarget{CHAPTER_XV}{}{}CHAPTER
XV}{CHAPTER XV}}\label{chapter-xv}}

Lettering---Blind Tooling---Heraldic Ornament

\hypertarget{lettering-on-the-back}{%
\subparagraph{LETTERING ON THE BACK}\label{lettering-on-the-back}}

{Lettering} may be done either with separate letters, each on its own
handle, or with type set in a type-holder and worked across the back as
a pallet. Although by the use of type great regularity is ensured, and
some time saved, the use of handle letters gives so much more freedom of
arrangement, that their use is advocated for extra binding. Where a
great many copies of the same
work{\protect\hypertarget{Page_216}{}{{[}216{]}}} have to be lettered,
the use of type has obvious advantages.

A great deal depends on the design of the letters used. Nearly all
bookbinders' letters are made too narrow, and with too great difference
between the thick and thin strokes. At \protect\hyperlink{Fig_90}{fig.
90} is shown an alphabet, for which I am indebted to the kindness of Mr.
Emery Walker. The long tail of the Q is meant to go under the U. It
might be well to have a second R cut, with a shorter tail, to avoid the
great space left when an A happens to follow it. I have found that four
sizes of letters are sufficient for all
books.\protect\hypertarget{Fig_91}{}{}

\protect\hypertarget{Fig_90}{}{}
\includegraphics[width=3.125in,height=1.86458in]{images/gs217.jpg}

Fig. 90.

\begin{longtable}[]{@{}
  >{\raggedright\arraybackslash}p{(\columnwidth - 2\tabcolsep) * \real{0.5000}}
  >{\raggedright\arraybackslash}p{(\columnwidth - 2\tabcolsep) * \real{0.5000}}@{}}
\toprule()
\endhead
\begin{minipage}[t]{\linewidth}\raggedright
\includegraphics[width=2.08333in,height=0.9375in]{images/gs218.jpg}

Fig. 91.
\end{minipage} & \begin{minipage}[t]{\linewidth}\raggedright
\protect\hypertarget{Fig_92}{}{}
\includegraphics[width=2.08333in,height=0.77083in]{images/gs218a.jpg}
{Fig. 92.}
\end{minipage} \\
\bottomrule()
\end{longtable}

To make out a lettering paper for the back of a book, cut a strip of
good thin{\protect\hypertarget{Page_217}{}{{[}217{]}}} paper as wide as
the height of the panel to be lettered. Fold it near the centre, and
mark the fold with a pencil. This should give a line exactly at right
angles to the top and bottom of the strip. Then make another fold the
distance from the first of the width of the back; then bring the two
folds together, and make a third fold in the exact centre. The paper
should then be as shown at \protect\hyperlink{Fig_91}{fig. 91}.
Supposing the lettering to be THE WORKS OF ROBERT LOUIS STEVENSON,
select the size of letter you desire to use, and take an E and mark on a
piece of spare paper a line of E's, and laying your folded paper against
it, see how many letters will go in comfortably. Supposing you find that
four lines of five letters of the selected size can be put in, you must
see if your title can be conveniently cut up into four lines of five
letters, or less. It might be done as shown at
\protect\hyperlink{Fig_93}{fig. 93}. But if you prefer not to split
the{\protect\hypertarget{Page_218}{}{{[}218{]}}} name STEVENSON, a
smaller letter must be employed, and then the lettering may be as at
\protect\hyperlink{Fig_94}{fig. 94}.

To find out the position of the lines of lettering on a panel, the
letter E is again taken and impressed five times at the side of the
panel, as shown at \protect\hyperlink{Fig_92}{fig. 92}, leaving a little
greater distance between the lowest letter and the bottom of the panel,
than between the letters. The paper is then folded on the centre fold,
and, with dividers set to the average distance between the head of one
letter and the head of the next, five points are made through the folded
paper. The paper is opened, turned over, and the points joined with a
fine folder worked against the straight-edge. It should leave on the
front five raised lines, up to which the head of the letters must be
put.

\begin{longtable}[]{@{}
  >{\raggedright\arraybackslash}p{(\columnwidth - 2\tabcolsep) * \real{0.5000}}
  >{\raggedright\arraybackslash}p{(\columnwidth - 2\tabcolsep) * \real{0.5000}}@{}}
\toprule()
\endhead
\begin{minipage}[t]{\linewidth}\raggedright
\protect\hypertarget{Fig_93}{}{}
\includegraphics[width=1.5625in,height=1.5625in]{images/gs219.jpg}

Fig. 93.
\end{minipage} & \begin{minipage}[t]{\linewidth}\raggedright
\protect\hypertarget{Fig_94}{}{}
\includegraphics[width=1.52083in,height=1.5625in]{images/gs219a.jpg}

Fig. 94.
\end{minipage} \\
\bottomrule()
\end{longtable}

The letters in the top line are
counted,{\protect\hypertarget{Page_219}{}{{[}219{]}}} and the centre
letter marked. Spaces between words are counted as a letter; thus in
``THE WORKS,'' ``W'' will be the centre letter, and should be put on the
paper first, and the others added on each side of it. Some thought is
needed in judging where to put the centre, as the difference in the
width of such letters as ``M'' and ``W'' and ``I'' and ``J'' have to be
taken into account.

As a general rule, lettering looks best if it comfortably fills the
panel, but of course it cannot always be made to do this. The greatest
difficulty will be found in making titles of books that consist of a
single word, look well. Thus if you have ``CORIOLANUS'' to place on a
back which is not more than {5}⁄{8}-inch wide, if it is put across as
one word, as at \protect\hyperlink{Fig_95}{fig. 95} (1), it will be
illegible from the smallness of the type, and will tell merely as a gold
line at a little distance. If a reasonably large type is used, the word
must be broken up somewhat, as at (2), which is perhaps better, but
still not at all satisfactory. The word may be put straight along the
back, as at fig. (3), but this hardly looks well on a book with
raised{\protect\hypertarget{Page_220}{}{{[}220{]}}} bands, and should be
avoided unless necessary.

\protect\hypertarget{Fig_95}{}{}
\includegraphics[width=3.125in,height=2.90625in]{images/gs221.jpg}

Fig. 95.

The use of type of different sizes in lettering a book should be avoided
when possible, and on no account whatever should letters of different
design be introduced. Occasionally, when the reason for it is obvious,
it may be allowable to make a word shorter by putting in a small letter,
supposing that only thus could reasonably large type be used. It is
especially allowable in cases where, in a set of volumes, there is one
much thinner than the others. It is generally better to make some
compromise with the lettering of the thin volume, than to spoil
the{\protect\hypertarget{Page_221}{}{{[}221{]}}} lettering of the whole
set by using too small a letter throughout (see
\protect\hyperlink{Fig_115}{fig. 115}).

On very thin books it is sometimes hardly possible to get any lettering
at all on the back. In such cases the lettering is best put on the side.

In the case of some special books that are to have elaborately decorated
bindings, and are on that account sufficiently distinct from their
neighbours, a certain amount of freedom is permissible with the
lettering, and a little mystery is not perhaps out of place. But in most
cases books have to be recognised by their titles, and it is of the
utmost importance that the lettering should be as clear as possible, and
should fully identify the volume.

For lettering half-bindings and other books on which much time cannot be
spared, it would take too long to make out a paper, as described for
extra bindings, nor is there on such work much occasion for it. For such
books the lettering should be written out carefully, the whole panel
prepared and glaired in, and the gold laid on. Then with a piece of fine
silk or thread lines may be marked across the gold as a guide to the
finisher, and the letters worked from the
centre{\protect\hypertarget{Page_222}{}{{[}222{]}}} outward, as
described for making out the paper pattern. Of course this method does
not allow of such nice calculation and adjustment as when a paper
pattern is made out; but if a general principle of clear lettering is
recognised and accepted, very good results may be obtained.

\hypertarget{blind-tooling}{%
\subparagraph{BLIND TOOLING}\label{blind-tooling}}

\protect\hypertarget{Fig_96}{}{}
\includegraphics[width=2.08333in,height=2in]{images/gs223.jpg}

Fig. 96.

At the end of the book characteristic examples of blind-tooled books are
given (pages
\protect\hyperlink{Page_321}{321}-\protect\hyperlink{Page_325}{25}). It
will be seen that most of the tools form complete designs in themselves.
Although the use of detached die-sunk tools was general, there were also
simple tools used, which, when
combined,{\protect\hypertarget{Page_223}{}{{[}223{]}}} made up more or
less organic designs, and allowed more freedom to the finisher (see
\protect\hyperlink{Fig_96}{figs. 96} and
\protect\hyperlink{Fig_97}{97}).

\protect\hypertarget{Fig_97}{}{}
\includegraphics[width=3.125in,height=2.67708in]{images/gs224.jpg}

Fig. 97.

Some use may also be made of interlaced strap-work designs, either
worked with gouges, or a small fillet. A book bound in oaken boards,
with a leather back with knotted decoration, is shown at page
\protect\hyperlink{Page_330}{330}. I have found that such binding and
decoration is more satisfactory in scheme for old books, than most forms
of modern binding.{\protect\hypertarget{Page_224}{}{{[}224{]}}}

If a design is simple, the cover is marked up with dividers, and the
tools impressed direct upon the leather; or, if it is elaborate, a paper
pattern is made out, and the tools blinded through the paper, as
described for gold tooling. The leather is then damped with water, and
the impressions retooled.

\protect\hypertarget{Fig_98}{}{}
\includegraphics[width=0.41667in,height=1.04167in]{images/gs225.jpg}

Fig. 98.

The panel lines on most of the bindings before 1500 show evidence of
having been put in with a tool which has been pushed along the leather,
and not with a wheel. I have found that a tool guided by a
straight-edge, and ``jiggered'' backwards and forwards, makes by far the
best lines for blind-tool work. It should be borne in mind that the line
is formed by the raised portion of leather, and so the tool should be
cut somewhat as at \protect\hyperlink{Fig_98}{fig. 98}. This should
leave three ridges on the leather. Blind tooling may be gone over and
over until it is deep enough, and may be combined with various other
methods of working. For instance, in tooling such a spray as is shown at
\protect\hyperlink{Fig_99}{fig. 99}, the leaf would be formed by five
impressions of the second tool, shown at A, the extremity of the
impressions could be joined
with{\protect\hypertarget{Page_225}{}{{[}225{]}}} gouges, the stalk and
veining could either be run in with a fillet or worked with gouges. The
grapes would best be worked with a tool cut for the purpose. One edge of
all gouge or fillet impressions can be smoothed down with some such tool
as shown in section at B. This has to be worked round the gouge lines
with a steady hand, and may be fairly hot if it is kept moving. At C is
shown a section of a gouge impression before and after the use of this
tool. The ground can be dotted in, or otherwise gone over with some
small tool to throw up the pattern.

Blind tooling can sometimes be used in combination with gold tooling.

\protect\hypertarget{Fig_99}{}{}
\includegraphics[width=1.38542in,height=3.125in]{images/gs226.jpg}

Fig. 99.

In the fifteenth century the Venetian binders used little roundels of
some gesso-like substance, that were brightly coloured or gilt, in
combination with blind
tooling{\protect\hypertarget{Page_226}{}{{[}226{]}}} (see p.
\protect\hyperlink{Page_325}{325}). This is a method that might be
revived.

What is known as ``leather work'' is a further development of blind
tooling. This method of decoration has been revived lately, but not
generally with success. ``Leather work'' may be divided into two
branches; in one the surface of the leather is cut to outline the
pattern, and in the other the leather is embossed from the back, while
wet, and the pattern outlined by an indented line. Sometimes the two
methods are combined. As embossing from the back necessitates the work
being done before the leather is on the book, it is not very suitable
for decorating books. Leather first decorated and then stuck on the
book, never looks as if it was an integral part of the binding. The cut
leather work, which may be done after the book is bound, and leaves the
surface comparatively flat, is a better method to employ for books,
provided the cuts are not too deep, and are restricted to the boards, so
as not to weaken the leather at the back and joints. Much of the leather
used for ``leather work'' is of very poor quality, and will not last;
for modelling it must be thick on the side of the book, and
for{\protect\hypertarget{Page_227}{}{{[}227{]}}} the book to open it
must be pared thin at the joint, thus making it necessary to use a thick
skin very much pared down, and consequently weakened (see p.
\protect\hyperlink{Page_155}{155}). Another very common fault in
modelled ``leather work'' is, that the two sides and the back are often
worked separately and stuck together on the book, necessitating a join,
and consequently a weak place in the hinge, where strength is most
wanted. Again, in most modern ``leather work,'' those who do the
decoration do not, as a rule, do the binding, and often do not
understand enough of the craft to do suitable work.

All those engaged in leather work are advised to learn to bind their own
books, and to only use such methods of decoration, as can be carried out
on the bound book.

\hypertarget{heraldry-on-book-covers}{%
\subparagraph{HERALDRY ON BOOK COVERS}\label{heraldry-on-book-covers}}

It is an old and good custom to put the arms of the owner of a library
on the covers of the books he has bound. The traditional, and certainly
one of the best ways to do this, is to have an arms block designed and
cut. To design an arms{\protect\hypertarget{Page_228}{}{{[}228{]}}}
block, knowledge of heraldry is needed, and also some clear idea of the
effect to be aimed at. A very common mistake in designing blocks is to
try and get the effect of hand tooling. Blocks should be and look
something entirely different. In hand tooling much of the effect is got
from the impressions of small tools reflecting the light at slightly
different angles, giving the work life and interest. Blocked gold being
all in one plane, has no such lights in it, and depends entirely on its
design for its effect.

Provided the heraldry identifies the owner, it should be as simply drawn
as it can be; the custom of indicating the tinctures by lines and dots
on the charges, generally makes a design confused, obscuring the coat it
is intended to make clear. In designing heraldic blocks it is well to
get a good deal of solid flat surface of gold to make the blocked design
stand out from any gold-tooled work on the cover.

Another way of putting armorial bearings on covers, is to paint them in
oil paint. In the early sixteenth century the Venetians copied the
Eastern custom of sinking panels in their book covers,
and{\protect\hypertarget{Page_229}{}{{[}229{]}}} painted coats of arms
on these sunk portions very successfully. The groundwork of the shield
itself was usually raised a little, either by something under the
leather, or by some gesso-like substance on its surface.

Arms blocks should be placed a little above the centre of the cover.
Generally, if the centre of the block is in a line with the centre band
of a book with five bands, it will look right.

Blocks are struck with the aid of an arming or blocking press. The block
is attached to the movable plate of the press called the ``platen.'' To
do this some stout brown paper is first glued to the platen, and the
block glued to this, and the platen fixed in its place at the bottom of
the heating-box. In blocking arms on a number of books of different
sizes, some nice adjustment of the movable bed is needed to get the
blocks to fall in exactly the right place.

For blocking, one coat of glaire will be enough for most leathers. The
gold is laid on as for hand tooling. The block should be brought down
and up again fairly sharply. The heat needed is about the same as for
hand tooling.

\hypertarget{chapter-xvi230}{%
\subsection[CHAPTER
XVI]{\texorpdfstring{\protect\hypertarget{CHAPTER_XVI}{}{}CHAPTER
XVI{\protect\hypertarget{Page_230}{}{{[}230{]}}}}{CHAPTER XVI{[}230{]}}}\label{chapter-xvi230}}

Designing for Gold-Tooled Decoration

\hypertarget{designing-tools}{%
\subparagraph{DESIGNING TOOLS}\label{designing-tools}}

{For} gold tooling, such tools as gouges, dots, pieces of straight line,
and fillets are to be had ready-made at most dealers. Other tools are
best designed and cut to order. At first only a few simple forms will be
needed, such as one or two flowers of different sizes, and one or two
sets of leaves (see \protect\hyperlink{Fig_100}{fig. 100}).

\protect\hypertarget{Fig_100}{}{}
\includegraphics[width=3.125in,height=0.80208in]{images/gs231.jpg}

Fig. 100 (reduced)

In designing tools, it must be borne in mind that they may appear on the
book many times repeated, and so must be simple in outline and much
conventionalised. A more or less naturalistic drawing of a flower,
showing the natural irregularities, may look charming, but if
a{\protect\hypertarget{Page_231}{}{{[}231{]}}} tool is cut from it, any
marked irregularity becomes extremely annoying when repeated several
times on a cover. So with leaves, unless they are perfectly symmetrical,
there should be three of each shape cut, two curving in different
directions, and the third quite straight (see
\protect\hyperlink{Fig_101}{fig. 101}). To have only one leaf, and to
have that curved, produces very restless patterns. The essence of
gold-tool design, is that patterns are made up of repeats of impressions
of tools, and that being so, the tools must be so designed that they
will repeat pleasantly, and in practice it will be found that any but
simple forms will become aggressive in repetition.

\protect\hypertarget{Fig_101}{}{}
\includegraphics[width=2.08333in,height=0.86458in]{images/gs232.jpg}

Fig. 101.

Designs for tools should be made out with Indian ink on white paper, and
they may be larger than the size of the required tool. The tool-cutter
will reduce any drawing to any desired size, and will, from one drawing,
cut any number of tools of different sizes. Thus, if a set
of{\protect\hypertarget{Page_232}{}{{[}232{]}}} five leaves of the same
shape is wanted, it will only be necessary to draw one, and to indicate
the sizes the others are to be in some such way as shown at
\protect\hyperlink{Fig_102}{fig. 102}.

It is not suggested that special tools should be cut for each pattern,
but the need of new tools will naturally arise from time to time, and so
the stock be gradually increased. It is better to begin with a very few,
and add a tool or two as occasion arises, than to try to design a
complete set when starting.

\protect\hypertarget{Fig_102}{}{}
\includegraphics[width=3.125in,height=1.08333in]{images/gs233.jpg}

Fig. 102.

Tools may be solid or in outline. If in outline they may be used as
``inlay'' tools, and in ordering them the tool-cutter should be asked to
provide steel punches for cutting the inlays.

\hypertarget{combining-tools-to-form-patterns}{%
\subparagraph{COMBINING TOOLS TO FORM
PATTERNS}\label{combining-tools-to-form-patterns}}

It is well for the student to begin with patterns arranged on some very
simple{\protect\hypertarget{Page_233}{}{{[}233{]}}} plan, making slight
changes in each succeeding pattern. In this way an individual style may
be established. The usual plan of studying the perfected styles of the
old binders, and trying to begin where they left off, in practice only
leads to the production of exact imitations, or poor lifeless parodies,
of the old designs. Whereas a pattern developed by the student by slow
degrees, through a series of designs, each slightly different from the
one before it, will, if eccentricities are avoided, probably have life
and individual interest.

Perhaps the easiest way to decorate a binding is to cover it with some
small repeating pattern. A simple form of diaper as a beginning is shown
at \protect\hyperlink{Fig_104}{fig. 104}. To make such a pattern cut a
piece of good, thin paper to the size of the board of a book, and with a
pencil rule a line about an eighth of an inch inside the margin all
round. Then with the point of a fine folder that will indent, but not
cut the paper, mark up as shown in \protect\hyperlink{Fig_103}{fig.
103}. The position of the lines A A and B B are found by simply folding
the paper, first side to side, and then head to tail. The other lines
can be put in without any{\protect\hypertarget{Page_234}{}{{[}234{]}}}
measurement by simply joining all points where lines cross. By continual
re-crossing, the spaces into which the paper is divided can be reduced
to any desired size. If the construction lines are accurately put in,
the spaces will all be of the same size and shape. It is then evident
that a repeating design to fill any one of the spaces can be made to
cover the whole surface.

\protect\hypertarget{Fig_103}{}{}
\includegraphics[width=3.125in,height=2.32292in]{images/gs235.jpg}

Fig. 103.

In \protect\hyperlink{Fig_104}{fig. 104}, it is the diagonal lines only
that are utilised for the pattern. To avoid confusion, the cross lines
that helped to determine the position of the diagonals are not
shown.{\protect\hypertarget{Page_235}{}{{[}235{]}}}

\protect\hypertarget{Fig_104}{}{}
\includegraphics[width=3.125in,height=4.8125in]{images/gs236.jpg}

Fig. 104 (reduced)

The advantage of using the point of
a{\protect\hypertarget{Page_236}{}{{[}236{]}}} folder to mark up the
constructional lines of a pattern instead of a pencil, is that the lines
so made are much finer, do not rub out, and do not cause confusion by
interfering with the pattern. Any lines that will appear on the book,
such as the marginal lines, may be put in with a pencil to distinguish
them.

Having marked up the paper, select a flower tool and impress it at the
points where the diagonal lines cross, holding it in the smoke of a
candle between every two or three impressions. When the flower has been
impressed all over, select a small piece of straight line, and put a
stalk in below each flower; then a leaf put in on each side of the
straight line will complete the pattern.

\protect\hypertarget{Fig_105}{}{}
\includegraphics[width=3.125in,height=4.86458in]{images/gs238.jpg}

Fig. 105 (reduced)

A development of the same principle is shown at
\protect\hyperlink{Fig_105}{fig. 105}, in which some gouges are
introduced. Any number of other combinations will occur to any one using
the tools. Frequently questions will arise as to whether a tool is to be
put this way or that way, and whether a line is to curve up or down.
Whenever there is such an alternative open, there is the germ of another
pattern. All-over diaper
patterns{\protect\hypertarget{Page_237}{}{{[}237{]}}} may be varied in
any number of ways. One way is to vary the design in alternate spaces.
If this is done one of the designs should be such that it will divide
down the centre both ways and so finish off the pattern comfortably at
the edges. The pattern may be based on the upright and the cross-lines
of the marking up, or the marking up may be on a different
principle{\protect\hypertarget{Page_238}{}{{[}238{]}}} altogether. The
designer, after a little practice, will be bewildered by the infinite
number of combinations that occur to him.

\protect\hypertarget{Fig_106}{}{}
\includegraphics[width=3.125in,height=4.82292in]{images/gs240.jpg}

Fig. 106 (reduced)

The diaper is selected for a beginning, because it is the easiest form
of pattern to make, as there is no question of getting round corners,
and very little of studying proportion. It is selected also because it
teaches the student the decorative value of simple forms repeated on
some orderly system. When he has grasped this, he has grasped the
underlying principle of nearly all successful tooled ornament. Diapers
are good practice, because in a close, all-over pattern the tools must
be put down in definite places, or an appalling muddle will result. In
tooling; a repeat of the same few tools, is the best possible practice,
giving as it does the same
work{\protect\hypertarget{Page_239}{}{{[}239{]}}} over and over again
under precisely the same conditions, and concentrating, on one book
cover, the practice that might be spread over several backs and sides
more sparingly decorated, when
variety{\protect\hypertarget{Page_240}{}{{[}240{]}}} of conditions would
confuse the student.

\protect\hypertarget{Fig_107}{}{}
\includegraphics[width=0.52083in,height=0.97917in]{images/gs241.jpg}

Fig. 107.

When the principles of the diaper have been mastered, and the student
has become familiar with the limitations of his tools, other schemes of
decoration may be attempted, such as borders, centres, or panels.

A form of border connected with cross-lines is shown at
\protect\hyperlink{Fig_106}{fig. 106}. This is made up of a repeat of
the spray built up of three tools and four gouges shown at
\protect\hyperlink{Fig_107}{fig. 107}, with slight modification at the
corners. Other schemes for borders are those in which flowers grow
inwards from the edge of the boards, or outwards from a panel at the
centre, or on both sides of a line about half an inch from the edge. A
pattern may also be made to grow all round the centre panel. Borders
will be found more difficult to manage than simple diapers, and at
first, are best{\protect\hypertarget{Page_241}{}{{[}241{]}}} built up on
the same principle---the repeat of some simple element.

\protect\hypertarget{Fig_108}{}{}
\includegraphics[width=2.60417in,height=5.25in]{images/gs243.jpg}

Fig. 108 (reduced)

The decoration may be concentrated on parts of the cover, such as the
centre or corners. A design for a centre is shown at
\protect\hyperlink{Fig_108}{fig. 108}, and below is shown the way to
construct it. A piece of paper is folded, as shown by the dotted lines,
and an eighth of the pattern drawn with a soft pencil and folded over on
the line A, and transferred by being rubbed at the back with a folder.
This is lined in with a pencil, and folded over on the line B and rubbed
off. This is lined in and folded over on A and C, rubbed off as before,
and the whole lined in. The overs and unders of the lines are then
marked, and gouges selected to fit. Of course it will take several
trials before the lines will interlace pleasantly, and the tools fit in.
Another centre, in which a spray is repeated three times, is shown at
\protect\hyperlink{Fig_109}{fig. 109}, and any number of others will
occur to the student after a little practice. A change of tools, or the
slight alteration of a line, will give an entirely new aspect to a
pattern. At page \protect\hyperlink{Page_334}{334} is shown an all-over
pattern growing from the bottom centre of the board. In this design the
leather was{\protect\hypertarget{Page_242}{}{{[}242{]}}} dark green,
with a lighter green panel in the centre. The berries were inlaid in
bright red. Although at first glance it seems an intricate design, it is
made up like the others of repetitions of
simple{\protect\hypertarget{Page_243}{}{{[}243{]}}} forms.

\protect\hypertarget{Fig_109}{}{}
\includegraphics[width=3.125in,height=2.08333in]{images/gs244.jpg}

Fig. 109 (reduced)

When the student has become proficient in the arrangement of tools in
combination with lines, a design consisting entirely, or almost
entirely, of lines may be tried. This is more difficult, because the
limitations are not so obvious; but here again the principle of
repetition, and even distribution, should be followed. At
\protect\hyperlink{Fig_110}{fig. 110} is shown a design almost entirely
composed of lines, built up on the same principle as the centre at
\protect\hyperlink{Fig_108}{fig. 108}.

\protect\hypertarget{Fig_110}{}{}
\includegraphics[width=3.125in,height=4.30208in]{images/gs245.jpg}

Fig. 110 (reduced)

The ends of the bands form a
very{\protect\hypertarget{Page_244}{}{{[}244{]}}} pleasant
starting-place for patterns. At pp. \protect\hyperlink{Page_330}{330},
\protect\hyperlink{Page_332}{332}-\protect\hyperlink{Page_336}{6} are
shown ways of utilising{\protect\hypertarget{Page_245}{}{{[}245{]}}}
this method. To look right, a pattern must be consistent throughout. The
tools and their arrangement must have about the same amount of
convention. Gold tooling, dealing, as it does, with flat forms in
silhouette only, necessitates very considerable formality in the design
of the tools and of their arrangement on the cover. Modern finishers
have become so skilful, that they are able to produce in gold tooling
almost any design that can be drawn in lines with a pencil, and some
truly marvellous results are obtained by the use of inlays, and
specially cut gouges. As a rule, such patterns simply serve to show the
skill of the finisher, and to make one wonder who could have been
foolish enough to select so limited and laborious a method as gold
tooling for carrying them out.

Generally speaking, successful gold-tooled patterns show evidence of
having been designed with the tools; of being, in fact, mere
arrangements of the tools, and not of having been first designed with a
pencil, and then worked with tools cut to fit the drawing. This does not
of course apply to patterns composed entirely of lines, or to patterns
composed of lines of dots.{\protect\hypertarget{Page_246}{}{{[}246{]}}}

If artists wish to design for gold tooling without first mastering the
details, probably the safest way will be for them to design in lines of
gold dots. Some successful patterns carried out in this way were shown
at the Arts and Crafts Exhibition some years ago.

Designs for gold-tooled binding should always be constructed on some
geometrical plan, and whatever pattern there is, symmetrically
distributed over the cover.

If lettering can be introduced, it will be found to be most useful when
arranging a pattern. It gives dignity and purpose to a design, and is
also highly decorative. Lettering may be arranged in panels, as at page
\protect\hyperlink{Page_332}{332}, or in a border round the edge of the
board, and in many other ways. It may either consist of the title of the
book, or some line or verse from it or connected with it, or may refer
to its history, or to the owner. Anything that gives a personal interest
to a book, such as the arms of the owner, the initials or name of the
giver or receiver of a present, with perhaps the date of the gift, is of
value.

The use of the small fillet makes it possible to employ long,
slightly-curved{\protect\hypertarget{Page_247}{}{{[}247{]}}} lines.
Gold-tooled lines have in themselves such great beauty, that designers
are often tempted to make them meander about the cover in a weak and
aimless way. As the limitations enforced by the use of gouges tend to
keep the curves strong and small, and as the use of the small fillet
tends to the production of long, weak curves, students are advised at
first to restrict the curved lines in their patterns to such as can be
readily worked with gouges.

\protect\hypertarget{Fig_111}{}{}
\includegraphics[width=2.08333in,height=1.32292in]{images/gs248.jpg}

Fig. 111.

It must be remembered that a gouge or fillet line is very thin, and will
look weak if it goes far without support. For this reason interlaced
lines are advocated.

Gouge lines are easier to work, and look better, if a small space is
left where the gouges end. This is especially
the{\protect\hypertarget{Page_248}{}{{[}248{]}}} case where lines
bearing leaves or flowers branch from the main stem (see
\protect\hyperlink{Fig_111}{fig. 111}).

Gouges and fillets need not always be of the same thickness of line, and
two or three sets of different gauges may be kept. A finisher can always
alter the thickness of a gouge with emery paper.

One method of arranging gold-tooled lines is to treat them in design as
if they were wires in tension, and knot and twist them together.
Provided the idea is consistently adhered to throughout, such a pattern
is often very successful.

\protect\hypertarget{Fig_112}{}{}
\includegraphics[width=2.08333in,height=1.60417in]{images/gs249.jpg}

Fig. 112.

A simple arrangement of straight lines will be sufficient ornamentation
for most books. Three schemes for such ornamentation are shown. In
\protect\hyperlink{Fig_112}{fig. 112}
the{\protect\hypertarget{Page_249}{}{{[}249{]}}} ``tie-downs'' may be in
``blind'' and the lines in gold. The arrangement shown at
\protect\hyperlink{Fig_113}{fig. 113} leaves a panel at the top which
may be utilised for
lettering.{\protect\hypertarget{Page_250}{}{{[}250{]}}}

\protect\hypertarget{Fig_113}{}{}
\includegraphics[width=2.08333in,height=1.59375in]{images/gs250.jpg}

Fig. 113.

{\protect\hypertarget{Page_251}{}{{[}251{]}}}
\includegraphics[width=2.08333in,height=1.66667in]{images/gs250a.jpg}

Fig. 114.

\begin{longtable}[]{@{}
  >{\raggedright\arraybackslash}p{(\columnwidth - 2\tabcolsep) * \real{0.5000}}
  >{\raggedright\arraybackslash}p{(\columnwidth - 2\tabcolsep) * \real{0.5000}}@{}}
\toprule()
\endhead
\begin{minipage}[t]{\linewidth}\raggedright
\includegraphics[width=3.125in,height=5.47917in]{images/gs251.jpg}
\end{minipage} & \begin{minipage}[t]{\linewidth}\raggedright
\protect\hypertarget{Fig_115}{}{}
\includegraphics[width=2.82292in,height=5.45833in]{images/gs252.jpg}
\end{minipage} \\
\multicolumn{2}{@{}>{\raggedright\arraybackslash}p{(\columnwidth - 2\tabcolsep) * \real{1.0000} + 2\tabcolsep}@{}}{%
Fig. 115.} \\
\bottomrule()
\end{longtable}

\hypertarget{designing-for-backs252}{%
\subparagraph[DESIGNING FOR BACKS]{\texorpdfstring{DESIGNING FOR
BACKS{\protect\hypertarget{Page_252}{}{{[}252{]}}}}{DESIGNING FOR BACKS{[}252{]}}}\label{designing-for-backs252}}

The decoration of the back of a book is difficult owing to the very
small space usually available in the panels. The first consideration
must be the lettering, and when that has been arranged, as described in
Chapter XV, a second paper is got out for the pattern. The back panel
should generally be treated in the same style and, if possible, with the
same tools as the sides, if they are decorated. It will often be found
far easier to design a full-gilt side than a satisfactory back.

A design may be made to fit one panel of the book and repeated on all
those not required for lettering (see pages
\protect\hyperlink{Page_332}{332}-\protect\hyperlink{Page_334}{34}), or
it may be made to grow up from panel to panel (see
\protect\hyperlink{Fig_115}{fig. 115}). In the case of sets of books in
which the volumes vary very much in thickness, some pattern must be made
that can be contracted and expanded without altering the general look of
the back (see \protect\hyperlink{Fig_115}{fig.
115}).{\protect\hypertarget{Page_253}{}{{[}253{]}}}

\hypertarget{designing-for-inside-of-boards}{%
\subparagraph{DESIGNING FOR INSIDE OF
BOARDS}\label{designing-for-inside-of-boards}}

The inside margins of the board permit of a little delicate decoration.
At \protect\hyperlink{Fig_116}{fig. 116} are shown two ways of treating
this part of the binding. The inside of the board is sometimes covered
all over with leather, and tooled as elaborately, or more elaborately,
than the outside. If there are vellum ends, they may be enriched with a
little tooling.

\protect\hypertarget{Fig_116}{}{}
\includegraphics[width=3.125in,height=1.47917in]{images/gs254.jpg}

Fig. 116.

The edges of the boards may have a gold line run on them, and the
head-cap may be decorated with a few dots.

\hypertarget{chapter-xvii254}{%
\subsection[CHAPTER
XVII]{\texorpdfstring{\protect\hypertarget{CHAPTER_XVII}{}{}CHAPTER
XVII{\protect\hypertarget{Page_254}{}{{[}254{]}}}}{CHAPTER XVII{[}254{]}}}\label{chapter-xvii254}}

Pasting down End Papers---Opening Books

\hypertarget{pasting-down-end-papers}{%
\subparagraph{PASTING DOWN END PAPERS}\label{pasting-down-end-papers}}

{When} the finishing is done, the end papers should be pasted down on to
the board; or if there is a leather joint, the panel left should be
filled in to match the end paper.

To paste down end papers, the book is placed on the block with the board
open (see \protect\hyperlink{Fig_117}{fig. 117}, A), the waste sheets
are torn off, the joints cleared of any glue or paste, and the boards
flattened, as described at page \protect\hyperlink{Page_171}{171} for
pasting down leather joints. One of the paste-down papers is then
stretched over the board and rubbed down in the joint, and the amount to
be cut off to make it fit into the space left by the turn-in of the
leather is marked on it with dividers, measuring from the edge of the
board. A cutting tin is then placed on the book, the paste-down paper
turned over it, and the edges trimmed off to the divider points with
a{\protect\hypertarget{Page_255}{}{{[}255{]}}} knife and straight-edge,
leaving small pieces to cover the ends of the joint
(\protect\hyperlink{Fig_117}{fig. 117}, A, c).

The cutting and pasting down of these small pieces in the joint are
rather difficult; they should come exactly to the edges of the board.

\protect\hypertarget{Fig_117}{}{}
\includegraphics[width=3.125in,height=3.04167in]{images/gs256.jpg}

Fig. 117.

When both paste-down papers are trimmed to size, one of them is well
pasted with thin paste in which
there{\protect\hypertarget{Page_256}{}{{[}256{]}}} are no lumps, with a
piece of waste paper under it to protect the book. The joints should
also be pasted, and the paste rubbed in with the finger and any surplus
removed.

The pasted paper is then brought over on to the board, the edges
adjusted exactly to their places, and rubbed down. The joint must next
be rubbed down through paper. It is difficult to get the paper to stick
evenly in the joint, and great nicety is needed here. All rubbing down
must be done through paper, or the ``paste-down'' will be soiled or made
shiny.

Some papers stretch very much when pasted, and will need to be cut a
little smaller than needed, and put down promptly after pasting. Thin
vellum may be put down with paste in which there is a very little glue,
but thicker vellum is better put down with thin glue. In pasting vellum,
very great care is needed to prevent the brush-marks from showing
through. If the vellum is thin, the board must be lined with white or
toned paper with a smooth surface. This paper must be quite clean, as
any marks will show through the vellum, and make it look dirty.

When one side is pasted down the
book{\protect\hypertarget{Page_257}{}{{[}257{]}}} can be turned over
without shutting the board, and the other board opened and pasted down
in the same way (see \protect\hyperlink{Fig_117}{fig. 117}, B). In
turning over a book, a piece of white paper should be put under the
newly-pasted side, as, being damp, it will soil very readily. When both
ends have been pasted down the joints should be examined and rubbed down
again, and the book stood up on end with the boards open until the end
papers are dry. The boards may be held open with a piece of cardboard
cut as shown at \protect\hyperlink{Fig_71}{fig. 71}.

If there are cloth joints they are put down with glue, and the board
paper is placed nearly to the edge of the joint, leaving very little
cloth visible.

In the process of finishing, the boards of a book will nearly always be
warped a little outward, but the pasted end papers should draw the
boards a little as they dry, causing them to curve slightly towards the
book. With vellum ends there is a danger that the boards will be warped
too much.

\hypertarget{opening-newly-bound-books}{%
\subparagraph{OPENING NEWLY BOUND
BOOKS}\label{opening-newly-bound-books}}

Before sending out a newly bound book the binder should go through it,
opening{\protect\hypertarget{Page_258}{}{{[}258{]}}} it here and there
to ease the back. The volume is laid on a table, and the leaves opened a
short distance from the front, and then at an equal distance from the
back, and then in one or two places nearer the centre of the book, the
leaves being pressed down with the hand at each opening. If the book is
a valuable one, every leaf should then be turned over separately and
each opening pressed down, beginning from the centre and working first
one way and then the other. In this way the back will be bent evenly at
all points. When a book has been opened, it should be lightly pressed
for a short time without anything in the joints.

If a book is sent out unopened, the first person into whose hand it
falls will probably open it somewhere in the centre, bending the covers
back and ``breaking'' the back; and if any leaves chance to have been
stuck together in edge-gilding, they are likely to be torn if carelessly
opened. A book with a ``broken'' back will always have a tendency to
open in the same place, and will not keep its shape. It would be worth
while for librarians to have newly bound books carefully opened. An
assistant could ``open'' a large
number{\protect\hypertarget{Page_259}{}{{[}259{]}}} of books in a day,
and the benefit to the bindings would amply compensate for the small
trouble and cost involved.

\hypertarget{chapter-xviii}{%
\subsection[CHAPTER
XVIII]{\texorpdfstring{\protect\hypertarget{CHAPTER_XVIII}{}{}CHAPTER
XVIII}{CHAPTER XVIII}}\label{chapter-xviii}}

Clasps and Ties---Metal on Bindings

\hypertarget{clasps-and-ties}{%
\subparagraph{CLASPS AND TIES}\label{clasps-and-ties}}

{Some} books need to be clasped to keep the leaves flat. All books
written or printed on vellum should have clasps. Vellum unless kept flat
is apt to cockle, and this in a book will force the leaves apart and
admit dust. If a book is tightly wedged in a shelf the leaves will be
kept flat, but as the chance removal of any other book from the row will
remove the pressure, it is much better to provide clasps for vellum
books.

Very thick books, and those with a great many folded plates, are better
for having clasps to prevent the leaves from sagging. As nearly all
books are now kept in bookshelves, and as any projection on the side of
a book is likely to injure the
neighbouring{\protect\hypertarget{Page_260}{}{{[}260{]}}} volume, a form
of clasp should be used that has no raised parts on the boards.

\protect\hypertarget{Fig_118}{}{}
\includegraphics[width=1.04167in,height=1.63542in]{images/gs261.jpg}

Fig. 118.

At \protect\hyperlink{Fig_118}{fig. 118} is shown a simple clasp
suitable for small books with mill-board sides, with details of the
metal parts, made of thick silver wire below. Double boards must be
``made,'' and the flattened ends of the silver catch inserted between
the two thicknesses, and glued in place. About one-eighth of an inch of
the end should project. In covering, the leather must be pierced and
carefully worked round the catch. To make the plait, three strips of
thin leather are slipped through the ring, and the ends of each strip
pasted together. The three doubled strips are then plaited and the end
of the plait put through a hole in the lower board of the book about
half an inch from the edge, and glued down inside. A groove may be cut
in the mill-board from the hole to the edge before covering, to make a
depression{\protect\hypertarget{Page_261}{}{{[}261{]}}} in which the
plait will lie, and a depression may be scooped out of the inner surface
of the board to receive the ends.

\protect\hypertarget{Fig_119}{}{}
\includegraphics[width=2.08333in,height=2.44792in]{images/gs262.jpg}

Fig. 119.

At \protect\hyperlink{Fig_119}{fig. 119} is a somewhat similar clasp
with three plaits suitable for large books. The metal end and the method
of inserting it into wooden boards are shown below. The turned-down end
should go right through the board, and be riveted on the inside. When
the three plaits are worked, a little band of silver may be riveted on
just below the ring.

A very simple fastening that is sometimes useful is shown at
\protect\hyperlink{Fig_77}{fig. 77}. A very small bead is threaded on to
a piece of catgut, and the two ends of the gut brought together and put
through a larger bead. The ends of the gut with the beads on them are
laced into the top{\protect\hypertarget{Page_262}{}{{[}262{]}}} board of
the book, with the bead projecting over the edge, and a loop of gut is
laced into the bottom board. If the loop can be made exactly the right
length, this is a serviceable method.

Silk or leather ties may be used to keep books shut, but they are apt to
be in the way when the book is read, and as hardly anybody troubles to
tie them, they are generally of very little use.

\hypertarget{metal-on-bindings}{%
\subparagraph{METAL ON BINDINGS}\label{metal-on-bindings}}

Metal corners and bosses are a great protection to bindings, but if the
books are to go into shelves, the metal must be quite smooth and flat. A
metal shoe on the lower edge of the boards is an excellent thing for
preserving the binding of heavy books.

Bosses and other raised metal work should be restricted to books that
will be used on lecterns or reading desks. The frontispiece is from a
drawing of an early sixteenth-century book, bound in white pigskin, and
ornamented with brass corners, centres, and clasps; and at page
\protect\hyperlink{Page_323}{323} is shown a fifteenth-century binding
with plain protecting bosses. On this
book{\protect\hypertarget{Page_263}{}{{[}263{]}}} there were originally
five bosses on each board, but the centre ones have been lost.

Bindings may be entirely covered with metal, but the connection between
the binding and the book is in that case seldom quite satisfactory. The
most satisfactory metal-covered bindings that I have seen are those in
which the metal is restricted to the boards. The book is bound in wooden
boards, with thick leather at the back, and plaques of metal nailed to
the wood. The metal may be set with jewels or decorated with enamel, and
embossed or chased in various ways.

Jewels are sometimes set in invisible settings below the leather of
bindings, giving them the appearance of being set in the leather. This
gives them an insecure look, and it is better to frankly show the metal
settings and make a decorative feature of them.

\hypertarget{chapter-xix}{%
\subsection[CHAPTER
XIX]{\texorpdfstring{\protect\hypertarget{CHAPTER_XIX}{}{}CHAPTER
XIX}{CHAPTER XIX}}\label{chapter-xix}}

Leather

\hypertarget{leather}{%
\subparagraph{LEATHER}\label{leather}}

{Of} all the materials used by the bookbinders, leather is the most
important{\protect\hypertarget{Page_264}{}{{[}264{]}}} and the most
difficult to select wisely. It is extremely difficult to judge a leather
by its appearance.

``We find now, that instead of leather made from sheep, calf, goat, and
pigskins, each having, when finished, its own characteristic surface,
that sheepskins are got up to look like calf, morocco, or pigskin; that
calf is grained to resemble morocco, or so polished and flattened as to
have but little character left; while goatskins are grained in any
number of ways, and pigskin is often grained like levant morocco. So
clever are some of these imitations, that it takes a skilled expert to
identify a leather when it is on a book.''

There have been complaints for a long time of the want of durability of
modern bookbinding leather, but there has not been until lately any
systematic investigation into the causes of its premature decay.

By permission, I shall quote largely from the report of the committee
appointed by the Society of Arts to inquire into the subject. There are
on this special committee leather manufacturers, bookbinders,
librarians, and owners of libraries.
The{\protect\hypertarget{Page_265}{}{{[}265{]}}} report issued is the
result of an immense amount of work done. Many libraries were visited,
and hundreds of experiments and tests were carried out by the
sub-committees. There is much useful information in the report that all
bookbinders and librarians should read. The work of the committee is not
yet finished, but its findings may be accepted as conclusive as far as
they go.

The committee first set themselves to ascertain if the complaints of the
premature decay of modern bookbinding leather are justified by facts,
and on this point report that:---

``As regards the common belief that modern binding leather does decay
prematurely, the sub-committee satisfied themselves that books bound
during the last eighty or hundred years showed far greater evidence of
deterioration than those of an earlier date. Many recent bindings showed
evidence of decay after so short a period as ten, or even five years.
The sub-committee came to the conclusion that there is ample
justification for the general complaint that modern leather is not so
durable as that formerly used. To fix the date of the
commencement{\protect\hypertarget{Page_266}{}{{[}266{]}}} of this
deterioration was a difficult matter; but they came to the conclusion
that while leather of all periods showed some signs of decay, the
deterioration becomes more general on books bound after 1830, while some
leathers seem to be generally good until about 1860, after which date
nearly all leathers seem to get worse. The deterioration of calf
bindings at the latter end of the 19th century may be attributed as much
to the excessive thinness as to the poor quality of the material.''

The committee endeavoured to ascertain the relative durability of the
leathers used for bookbinding, and after visiting many libraries, and
comparing bindings, they report as follows:---

``As to the suitability of various leathers, the sub-committee came to
the conclusion that of the old leathers (15th and 16th century), white
pigskin, probably alum `tanned,' is the most durable, but its excessive
hardness and want of flexibility renders this leather unsuitable for
most modern work. Old brown calf has lasted fairly well, but loses its
flexibility, and becomes stiff and brittle when exposed to light and
air. Some of the white
tawed{\protect\hypertarget{Page_267}{}{{[}267{]}}} skins of the 15th and
16th century, other than white pigskin, and probably deerskin, have
lasted very well. Some 15th and 16th century sheepskin bindings have
remained soft and flexible, but the surface is soft, and usually much
damaged by friction. Vellum seems to have lasted fairly well, but is
easily influenced by atmospheric changes, and is much affected by light.
Early specimens of red morocco from the 16th to the end of the 18th
century were found in good condition, and of all the leathers noticed,
this seems to be the least affected by the various conditions to which
it had been subjected. In the opinion of the committee, most of this
leather has been tanned with sumach or some closely allied tanning
material. Morocco bindings earlier than 1860 were generally found to be
in fairly good condition, but morocco after that date seems to be much
less reliable, and in many cases has become utterly rotten. During the
latter part of the 18th century it became customary to pare down calf
until it was as thin as paper. Since about 1830 hardly any really sound
calf seems to have been used, as, whether thick or thin, it appears
generally to have perished.
Sheepskin{\protect\hypertarget{Page_268}{}{{[}268{]}}} bindings of the
early part of the century are many of them still in good condition.
Since about 1860 sheepskin as sheepskin is hardly to be found.
Sheepskins are grained in imitation of other leathers, and these
imitation-grained leathers are generally found to be in a worse
condition than any of the other bindings, except, perhaps, some of the
very thin calfskin. Undyed modern pigskin seems to last well, but some
coloured pigskin bindings had entirely perished. Modern leathers dyed
with the aid of sulphuric acid are all to be condemned. In nearly every
case Russia leather was found to have become rotten, at least in
bindings of the last fifty years.''

On the question of the causes of the decay noticed and the best methods
of preparing leather in the future, I may quote the following:---

``The work of a sub-committee, which was composed of chemists specially
conversant with the treatment of leather, was directed specially to the
elucidation of the following points: an investigation of the nature of
the decay of leather used for bookbinding; an examination of the causes
which produced this decay; a
research{\protect\hypertarget{Page_269}{}{{[}269{]}}} into the best
methods of preparing leather for bookbinding; and a consideration of the
points required to be dealt with in the preservation of books.

``Taking these points in order, the first one dealt with is the question
of the nature of the decay of leather. To arrive at their conclusions on
this subject, the sub-committee made a number of tests and analyses of
samples of decayed leather bookbindings, as well as of leathers used for
binding. The committee found that the most prevalent decay was what they
term a red decay, and this they think may be differentiated into old and
new, the old red decay being noticeable up to about 1830, and the new
decay since that date. In the old decay, the leather becomes hard and
brittle, the surface not being easily abraded by friction. The older
form is specially noticeable in calf-bound books, tanned presumably with
oak bark. The new form affects nearly all leathers, and in extreme cases
seems absolutely to destroy the fibres. Another form of deterioration,
more noticeable in the newer books, renders the grain of the leather
liable to peel off when exposed to the slightest friction. This is the
most{\protect\hypertarget{Page_270}{}{{[}270{]}}} common form of decay
noted in the more recent leathers. In nearly all samples of Russia
leather a very violent form of red decay was noticed. In many cases the
leather was found to be absolutely rotten in all parts exposed to light
and air, so that on the very slightest rubbing with a blunt instrument
the leather fell into fine dust....

``The second point is the cause of the decay. An extensive series of
experiments was carried out with a view of determining the causes of the
decay of bindings. The sub-committee find that this is caused by both
mechanical and by chemical influences. Of the latter, some are due to
mistakes of the leather manufacturer and the bookbinder, others to the
want of ventilation, and to improper heating and lighting of libraries.
In some cases inferior leathers are finished (by methods in themselves
injurious) so as to imitate the better class leathers, and of course
where these are used durability cannot be expected. But in the main the
injury for which the manufacturer and bookbinder are responsible must be
attributed rather to ignorance of the effect of the means employed to
give the leather the{\protect\hypertarget{Page_271}{}{{[}271{]}}}
outward qualities required for binding, than to the intentional
production of an inferior article.... Leathers produced by different
tanning materials, although they may be equally sound and durable
mechanically, vary very much in their resistance to other influences,
such as light, heat, and gas fumes.

``For bookbinding purposes, the sub-committee generally condemn the use
of tanning materials belonging to the catechol group, although the
leathers produced by the use of these materials are for many purposes
excellent, and indeed superior. The class of tanning materials which
produce the most suitable leather for this particular purpose belong to
the pyrogallol group, of which a well known and important example is
sumach. East Indian or `Persian' tanned sheep and goat skins, which are
suitable for many purposes, and are now used largely for cheap
bookbinding purposes, are considered extremely bad. Books bound in these
materials have been found to show signs of decay in less than twelve
months, and the sub-committee are inclined to believe that no book bound
in these leathers, exposed on a shelf to
sunlight{\protect\hypertarget{Page_272}{}{{[}272{]}}} or gas fumes, can
ever be expected to last more than five or six years. Embossing leather
under heavy pressure to imitate a grain has a very injurious effect,
while the shaving of thick skins greatly reduces the strength of the
leather by cutting away the tough fibres of the inner part of the skin.
The use of mineral acids in brightening the colour of leather, and in
the process of dyeing, has a serious effect in lessening its resistance
to decay. A good deal yet remains to be learned about the relative
permanency of the different dyes.''

On analysis free sulphuric acid was found to be present in nearly all
bookbinding leather, and it is the opinion of the committee that even a
small quantity of this acid materially lessens the durability of the
leather.

``It has been shown by careful experiment, that even a minute quantity
of sulphuric acid used in the dye bath to liberate the colour is at once
absorbed by the leather, and that no amount of subsequent washing will
remove it. In a very large proportion of cases the decay of modern
sumach-tanned leather has been due to the sulphuric acid used in the
dye{\protect\hypertarget{Page_273}{}{{[}273{]}}} bath, and retained in
the skin. We have examined very many samples of leather manufactured and
sold specially for bookbinding purposes, from different factories,
bought from different dealers, or kindly supplied by bookbinders and by
librarians, and have found them to contain, in a large number of cases,
free sulphuric acid, from 0.5 up to 1.6 per cent.''

The publication of the report should tend to fix a standard for
bookbinding leather. Hitherto there has been no recognised standard.
Bookbinders have selected leather almost entirely by its appearance. It
has now been shown that appearance is no test of durability, and the
mechanical test of tearing the leather is insufficient. Sound leather
should tear with difficulty, and the torn edges should be fringed with
long, silky fibres, and any leather which tears very easily, and shows
short, curled-up fibres at the torn edges, should be discarded. But
though good bookbinding leather will tear with difficulty, and show long
fibres where torn, that is in itself not a sufficient test; because it
has been shown that the leather that is mechanically the strongest, is
not necessarily the most{\protect\hypertarget{Page_274}{}{{[}274{]}}}
durable and the best able to resist the adverse influences to which
books are subject in libraries.

The report shows that bookbinders and librarians are not, as a general
rule, qualified to select leather for bookbinding. In the old days, when
the manufacture of leather was comparatively simple, a bookbinder might
reasonably be expected to know enough of the processes employed to be
able to select his leather. But now so complicated is the manufacture,
and so many are the factors to be considered, that an expert should be
employed.

``The committee have satisfied themselves that it is possible to test
any leather in such a way as to guarantee its suitability for
bookbinding. They have not come to any decision as to the desirability
of establishing any formal or official standard, though they consider
that this is a point which well deserves future consideration.''

It is to be hoped that some system of examining and hall-marking leather
by some recognised body, may be instituted. If librarians will specify
that the leather to be employed must be certified to be manufactured
according to the recommendations of the Society of
Arts{\protect\hypertarget{Page_275}{}{{[}275{]}}} Committee, there is no
reason why leathers should not be obtained as durable as any ever
produced. This would necessitate the examining and testing of batches of
leather by experts. At present this can be done more or less privately
at various places, such as the Yorkshire College, Leeds, or the Herolds'
Institute, Bermondsey. In the near future it is to be hoped that some
recognised public body, such as one of the great City Companies
interested in leather, may be induced to establish a standard, and to
test such leathers as are submitted to them, hall-marking those that
come up to the standard. This would enable bookbinders and librarians,
in ordering leather, to be sure that it had not been injured in its
manufacture. The testing, if done by batches, should not add greatly to
the cost of the leather.

On the question of the qualities of an ideal bookbinding leather the
committee report:---

``It is the opinion of the committee, that the ideal bookbinding leather
must have, and retain, great flexibility.... (It) must have a firm grain
surface, not easily damaged by friction, and should not
be{\protect\hypertarget{Page_276}{}{{[}276{]}}} artificially grained....
The committee is of opinion that a pure sumach tannage will answer all
these conditions, and that leather can, and will, be now produced that
will prove to be as durable as any made in the past.''

The committee has so far only dealt with vegetable-tanned leather. I
have used, with some success, chrome-tanned calfskin. Chrome leather is
difficult to pare, and to work, as it does not become soft when wet,
like vegetable-tanned leather. It will stand any reasonable degree of
heat, and so might perhaps be useful for top-shelf bindings and for
shelf edging. It is extremely strong mechanically, but without further
tests I cannot positively recommend it except for trial.

While the strength and probable durability of leather can only be judged
by a trained leather chemist, there remains for the binders selection,
the kind of leather to use, and its colour.

Most of the leather prepared for bookbinding is too highly finished. The
finishing processes add a good deal to the cost of the leather, and are
apt to be injurious to it, and as much of the
high{\protect\hypertarget{Page_277}{}{{[}277{]}}} finish is lost in
covering, it would be better for the bookbinder to get rougher leather
and finish it himself when it is on the book.

The leathers in common use for bookbinding are:---

\begin{itemize}
\tightlist
\item
  Goatskin, known as morocco.
\item
  Calf, known as calf and russia.
\item
  Sheepskin, known as roan, basil, skiver, \&c.
\item
  Pigskin, known as pigskin.
\item
  Sealskin, known as seal.
\end{itemize}

\emph{Morocco} is probably the best leather for extra binding if
properly prepared, but experiment has shown that the expensive Levant
moroccos are nearly always ruined in their manufacture. A great many
samples of the most expensive Levant morocco were tested, with the
result that they were all found to contain free sulphuric acid.

\emph{Calf.}---Modern vegetable-tanned calf has become a highly
unsatisfactory material, and until some radical changes are made in the
methods of manufacturing it, it should not be used for bookbinding.

\emph{Sheepskin.}---A properly tanned sheepskin makes a very durable,
though rather{\protect\hypertarget{Page_278}{}{{[}278{]}}} soft and
woolly, leather. Much of the bookbinding leather now made from sheepskin
is quite worthless. Bookbinders should refuse to have anything to do
with any leather that has been artificially grained, as the process is
apt to be highly injurious to the skin.

\emph{Pigskin.}---Pigskin is a thoroughly good leather naturally, and
very strong, especially the alumed skins; but many of the dyed pigskins
are found to be improperly tanned and dyed, and worthless for
bookbinding.

\emph{Sealskin} is highly recommended by one eminent librarian, but I
have not yet had any experience of its use for bookbinding.

The leather that I have found most useful is the Niger goatskin, brought
from Africa by the Royal Niger Company; it is a very beautiful colour
and texture, and has stood all the tests tried, without serious
deterioration. The difficulty with this leather is that, being a native
production, it is somewhat carelessly prepared, and is much spoiled by
flaws and stains on the surface, and many skins are quite worthless. It
is to be hoped that before long some of the
manufacturers{\protect\hypertarget{Page_279}{}{{[}279{]}}} interested
will produce skins as good in quality and colour as the best Niger
morocco, and with fewer flaws.

Much leather is ruined in order to obtain an absolutely even colour. A
slight unevenness of colours is very pleasing, and should rather be
encouraged than objected to. That the want of interest in absolutely
flat colours has been felt, is shown by the frequency with which the
binders get rid of flat, even colours by sprinkling and marbling.

On this point I may quote from the committee: ``The sprinkling of
leather, either for the production of `sprinkled' calf or `tree' calf,
with ferrous sulphate (green vitriol) must be most strongly condemned,
as the iron combines with and destroys the tan in the leather, and free
sulphuric acid is liberated, which is still more destructive. Iron
acetate or lactate is somewhat less objectionable, but probably the same
effects may be obtained with aniline colours without risk to the
leather.''

\hypertarget{chapter-xx280}{%
\subsection[CHAPTER
XX]{\texorpdfstring{\protect\hypertarget{CHAPTER_XX}{}{}CHAPTER
XX{\protect\hypertarget{Page_280}{}{{[}280{]}}}}{CHAPTER XX{[}280{]}}}\label{chapter-xx280}}

Paper---Pastes---Glue

\hypertarget{paper}{%
\subparagraph{PAPER}\label{paper}}

{Paper} may be made by hand or machinery, and either ``laid'' or
``wove.'' ``Laid'' papers are distinguished by wire marks, which are
absent in ``wove'' paper.

A sheet of hand-made paper has all round it a rough uneven edge called
the ``deckle,'' that is a necessary result of its method of manufacture.
The early printers looked upon this ragged edge as a defect, and almost
invariably trimmed most of it off before putting books into permanent
bindings. Book-lovers quite rightly like to find traces of the
``deckle'' edge, as evidence that a volume has not been unduly reduced
by the binder. But it has now become the fashion to admire the
``deckle'' for its own sake, and to leave books on hand-made paper
absolutely untrimmed, with ragged edges that collect the dirt, are
unsightly, and troublesome to turn over. So far has this craze gone,
that machine-made paper{\protect\hypertarget{Page_281}{}{{[}281{]}}} is
often put through an extra process to give it a sham deckle edge.

Roughly speaking, paper varies in quality according to the proportion of
fibrous material, such as rag, used in the manufacture. To make paper
satisfactorily by hand, a large proportion of such fibrous material is
necessary, so that the fact that the paper is hand-made is to some
extent a guarantee of its quality. There are various qualities of
hand-made paper, made from different materials, chiefly linen and cotton
rags. The best paper is made from pure linen rag, and poorer hand-made
paper from cotton rag, while other qualities contain a mixture of the
two or other substances.

It is possible to make a thoroughly good paper by machinery if good
materials are used. Some excellent papers are made by machinery; but the
enormous demand for paper, together with the fact that now almost any
fibrous material can be made into paper, has resulted in the production,
in recent years, of, perhaps, the worst papers that have ever been seen.

This would not matter if the use of the poor papers were restricted to
newspapers{\protect\hypertarget{Page_282}{}{{[}282{]}}} and other
ephemeral literature, but when, as is often the case, paper of very poor
quality is used for books of permanent literary interest, the matter is
serious enough.

Among the worst papers made are the heavily loaded ``Art'' papers that
are prepared for the printing of half-toned process blocks. It is to be
hoped that before long the paper makers will produce a paper that, while
suitable for printing half-toned blocks, will be more serviceable, and
will have a less unpleasant surface.

Several makers produce coloured handmade papers suitable for end papers.
Machine-made papers can be had in endless variety from any number of
makers.

The paper known as ``Japanese Vellum'' is a very tough material, and
will be found useful for repairing vellum books; the thinnest variety of
it is very suitable for mending the backs of broken sections, or for
strengthening weak places in paper.

The following delightful account of paper making by hand is quoted from
``Evelyn's Diary, 1641-1706.''

``I went to see my Lord of St. Alban's house at Byflete, an old large
building.{\protect\hypertarget{Page_283}{}{{[}283{]}}} Thence to the
paper mills, where I found them making a coarse white paper. They cull
the raggs, which are linnen, for white paper, woollen for brown, then
they stamp them in troughs to a papp with pestles or hammers like the
powder-mills, then put it into a vessell of water, in which they dip a
frame closely wyred with a wyre as small as a haire, and as close as a
weaver's reede; on this they take up the papp, the superfluous water
draining thro' the wyre; this they dextrously turning, shake out like a
pancake on a smooth board between two pieces of flannell, then press it
between a greate presse, the flannell sucking out the moisture; then
taking it out they ply and dry it on strings, as they dry linnen in the
laundry; then dip it in alum-water, lastly polish and make it up in
quires. They put some gum in the water in which they macerate the raggs.
The mark we find on the sheets is formed in the wyre.''

The following are the more usual sizes of printing papers---

\begin{longtable}[]{@{}ll@{}}
\toprule()
\endhead
~ & Inches. \\
Foolscap & 17 × 13½ \\
Crown & 20 × 15 \\
{\protect\hypertarget{Page_284}{}{{[}284{]}}}Post & 19¼ × 15½ \\
Demy & 22½ × 17½ \\
Medium & 24 × 19 \\
Royal & 25 × 20 \\
Double Pott & 25 × 15 \\
{"}{Foolscap} & 27 × 17 \\
Super Royal & 27 × 21 \\
Double Crown & 30 × 20 \\
Imperial & 30 × 22 \\
Double Post & 31½ × 19½ \\
\bottomrule()
\end{longtable}

The corresponding sizes of hand-made papers may differ slightly from the
above.

Although the above are the principal sizes named, almost any size can be
made to order.

The following is an extract from the report of the Committee of the
Society of Arts on the deterioration of paper, published in 1898: ``The
committee find that the paper-making fibres may be ranged into four
classes:---

\begin{itemize}
\tightlist
\item
  A. Cotton, flax, and hemp.
\item
  B. Wood, celluloses (\emph{a}) sulphite process, and (\emph{b}) soda
  and sulphate process.
\item
  C. Esparto and straw celluloses.
\item
  D. Mechanical wood pulp.
\end{itemize}

In regard, therefore, to papers
for{\protect\hypertarget{Page_285}{}{{[}285{]}}} books and documents of
permanent value, the selection must be taken in this order, and always
with due regard to the fulfilment of the conditions of normal treatment
above dealt with as common to all papers.''

``The committee have been desirous of bringing their investigations to a
practical conclusion in specific terms, viz. by the suggestion of
standards of quality. It is evident that in the majority of cases, there
is little fault to find with the practical adjustments which rule the
trade. They are, therefore, satisfied to limit their specific findings
to the following, viz., \emph{Normal standard of quality for book papers
required for publications of permanent value.} For such papers they
would specify as follows:---

``\emph{Fibres.} Not less than 70 per cent. of fibres of Class A.

``\emph{Sizing.} Not more than 2 per cent. rosin, and finished with the
normal acidity of pure alum.

``\emph{Loading.} Not more than 10 per cent. total mineral matter (ash).

``With regard to written documents, it must be evident that the proper
materials are those of Class A, and that the
paper{\protect\hypertarget{Page_286}{}{{[}286{]}}} should be pure, and
sized with gelatine, and not with rosin. All imitations of high-class
writing papers, which are, in fact, merely disguised printing papers,
should be carefully avoided.''

\hypertarget{pastes}{%
\subparagraph{PASTES}\label{pastes}}

To make paste for covering books, \&c., take 2 oz. of flour, and ¼ oz.
of powdered alum, and well mix with enough water to form a thin paste,
taking care to break up any lumps. Add a pint of cold water, and heat
gently in an enamelled saucepan. As it becomes warm, it should be
stirred from time to time, and when it begins to boil it should be
continually stirred for about five minutes. It should then form a thick
paste that can be thinned with warm water. Of course any quantity can be
made if the proportions are the same.

Paste for use is best kept in a wooden trough, called a ``paste tub.''
The paste tub will need to be cleaned out from time to time, and all
fragments of dry paste removed. This can easily be done if it is left,
overnight, filled with
water.{\protect\hypertarget{Page_287}{}{{[}287{]}}} Before using, the
paste should be well beaten up with a flat stick.

For pasting paper, it should have about the consistency and smoothness
of cream; for leather, it can be thicker. For very thick leather a
little thin glue may be added. Paste made with alum will keep about a
fortnight, but can be kept longer by the addition of corrosive sublimate
in the proportion of one part of corrosive sublimate to a thousand parts
of paste. Corrosive sublimate, being a deadly poison, will prevent the
attack of bookworms or other insects, but for the same reason must only
be used by responsible people, and paste in which it is used must be
kept out of the way of domestic animals.

Several makes of excellent prepared paste can be bought in London. These
pastes are as cheap as can be made, and keep good a long time.

Paste that has become sour should never be used, as there is danger that
the products of its acid fermentation may injure the leather.

Paste tubs as sold often have an iron bar across them to wipe the brush
on. This should be removed, and replaced by a piece of twisted cord.
Paste brushes{\protect\hypertarget{Page_288}{}{{[}288{]}}} should be
bound with string or zinc; copper or iron will stain the paste.

\hypertarget{white-paste-for-mending}{%
\subparagraph{WHITE PASTE FOR MENDING}\label{white-paste-for-mending}}

A good paste for mending is made from a teaspoonful of ordinary flour,
two teaspoonsful of cornflour, half a teaspoonful of alum, and three
ounces of water. These should be carefully mixed, breaking up all lumps,
and then should be heated in a clean saucepan, and stirred all the time
with a wooden or bone spoon. The paste should boil for about five
minutes, but not too fast, or it will burn and turn brown. Rice-flour or
starch may be substituted for cornflour, and for very white paper the
wheaten flour may be omitted. Ordinary paste is not nearly white enough
for mending, and is apt to leave unsightly stains.

Cornflour paste may be used directly after it is made, and will keep
good under ordinary circumstances for about a week. Directly it gets
hard or goes watery, a new batch must be
made.{\protect\hypertarget{Page_289}{}{{[}289{]}}}

\hypertarget{glue}{%
\subparagraph{GLUE}\label{glue}}

It is important for bookbinders that the glue used should be of good
quality, and the best hide glue will be found to answer well. To prepare
it for use, the glue should be broken up into small pieces and left to
soak overnight in water. In the morning it should be soft and greatly
swollen, but not melted, and can then be put in the glue-pot and gently
simmered until it is fluid. It is then ready for use. Glue loses in
quality by being frequently heated, so that it is well not to make a
great quantity at a time. The glue-pot should be thoroughly cleaned out
before new glue is put into it, and the old glue sticking round the
sides taken out.

Glue should be used hot and not too thick. If it is stringy and
difficult to work, it can be broken up by rapidly twisting the brush in
the glue-pot. For paper the glue should be very thin and well worked up
with the brush before using.

The following is quoted from ``Chambers' Encyclopædia'' article on
Glue:{\protect\hypertarget{Page_290}{}{{[}290{]}}}---

``While England does not excel in the manufacture, it is a recognised
fact that Scottish glue ... ranks in the front of the glues of all
countries. A light-coloured glue is not necessarily good, nor a
dark-coloured glue necessarily bad. A bright, clear, claret colour is
the natural colour of hide glue, which is the best and most economical.

``Light-coloured glues (as distinguished from gelatine) are made either
from bones or sheepskins. The glue yielded by these materials cannot
compare with the strength of that yielded by hides.

``A great quantity is now made in France and Germany from bones. It is
got as a by-product in the manufacture of animal charcoal. Although
beautiful to look at, it is found when used to be far inferior to
Scottish hide glue.''

\hypertarget{part-ii-care-of-books-when-bound}{%
\subsection[PART II\\
CARE OF BOOKS WHEN
BOUND]{\texorpdfstring{\protect\hypertarget{PART_II}{}{}PART II\\
CARE OF BOOKS WHEN
BOUND}{PART II CARE OF BOOKS WHEN BOUND}}\label{part-ii-care-of-books-when-bound}}

\hypertarget{chapter-xxi291}{%
\subsection[CHAPTER
XXI]{\texorpdfstring{\protect\hypertarget{CHAPTER_XXI}{}{}CHAPTER
XXI{\protect\hypertarget{Page_291}{}{{[}291{]}}}}{CHAPTER XXI{[}291{]}}}\label{chapter-xxi291}}

Injurious Influences to which Books are Subjected

\emph{Gas Fumes.}---The investigation of the Society of Arts Committee
shows that---

``Of all the influences to which books are exposed in libraries, gas
fumes---no doubt because of the sulphuric and sulphurous acid which they
contain---are shown to be the most injurious.''

The injurious effects of gas fumes on leather have been recognised for a
long time, and gas is being, very generally, given up in libraries in
consequence. If books must be kept where gas is used, they should not be
put high up in the{\protect\hypertarget{Page_292}{}{{[}292{]}}} room,
and great attention should be paid to ventilation. It is far better,
where possible, to avoid the use of gas at all in libraries.

\emph{Light.}---The committee also report that ``light, and especially
direct sunlight and hot air, are shown to possess deleterious influences
which had scarcely been suspected previously, and the importance of
moderate temperature and thorough ventilation of libraries cannot be too
much insisted on.''

The action of light on leather has a disintegrating effect, very plainly
seen when books have stood for long periods on shelves placed at right
angles to windows. At Oxford and Cambridge and at the British Museum
Library the same thing was noticed. The leather on that side, of the
backs of books, next to the light, was absolutely rotten, crumbling to
dust at the slightest friction, while at the side away from the light it
was comparatively sound. Vellum bindings were even more affected than
those of leather.

The committee advise that library windows exposed to the direct sunlight
should be glazed with tinted
glass.{\protect\hypertarget{Page_293}{}{{[}293{]}}}

``Some attempts have been made to determine the effect of light
transmitted through glasses of different colours, and they point to the
fact that blue and violet glass pass light of nearly as deleterious
quality as white glass; while leathers under red, green, and yellow
glasses were almost completely protected. There can be no doubt that the
use of pale yellow or olive-green glass in library windows exposed to
direct sunlight is desirable. A large number of experiments have been
made on the tinted `cathedral' glasses of Messrs. Pilkington Bros.,
Limited, with the result that Nos. 812 and 712 afforded almost complete
protection during two months' exposure to sunlight, while Nos. 704 and
804 may be recommended where only very pale shades are permissible. The
glasses employed were subjected to careful spectroscopic examination,
and to colour-measurement by the tintometer, but neither were found to
give precise indications as to the protective power of the glasses,
which is no doubt due to the absorption of the violet, and especially of
the invisible ultra-violet rays. An easy method of comparing glasses is
to expose under them to sunlight the ordinary
sensitised{\protect\hypertarget{Page_294}{}{{[}294{]}}} albumenised
photographic paper. Those glasses under which this is least darkened are
also most protective to leather.''

\emph{Tobacco.}---Smoking was found to be injurious, and it is certainly
a mistake to allow it in libraries.

``The effect of ammonia vapour, and tobacco fumes, of which ammonia is
one of the active ingredients, was also examined. The effect of ammonia
fumes was very marked, darkening every description of leather, and it is
known that in extreme cases it causes a rapid form of decay. Tobacco
smoke had a very similar darkening and deleterious effect (least marked
in the case of sumach tanned leathers), and there can be no doubt that
the deterioration of bindings in a library where smoking was permitted
and the rooms much used, must have been partly due to this cause.''

\emph{Damp.}---Books kept in damp places will develop mildew, and both
leather and paper will be ruined.

Where possible, naturally dry rooms should be used for libraries, and if
not naturally dry, every means possible should be taken to render them
so. It will some{\protect\hypertarget{Page_295}{}{{[}295{]}}}times be
found that the only way to keep the walls of an old house dry is to put
in a proper dampcourse. There are various other methods employed, such
as lining the walls with thin lead, or painting them inside and out with
some waterproofing preparation: but as long as a wall remains in itself
damp, it is doubtful if any of these things will permanently keep the
damp from penetrating.

Bookshelves should never be put against the wall, nor the books on the
floor. There should always be space for air to circulate on all sides of
the bookshelves. Damp is specially injurious if books are kept behind
closely-fitting doors. The doors of bookcases should be left open from
time to time on warm days.

Should mildew make its appearance, the books should be taken out, dried
and aired, and the bookshelves thoroughly cleaned. The cause of the damp
should be sought for, and measures taken to remedy it. Library windows
should not be left open at night, nor during damp weather, but in warm
fine weather the more ventilation there is, the better.

\emph{Heat.}---While damp is very injurious to books on account of the
development{\protect\hypertarget{Page_296}{}{{[}296{]}}} of mildew,
unduly hot dry air is almost as bad, causing leather to dry up and lose
its flexibility. On this point the Chairman of the Society of Arts
Committee says:---

``Rooms in which books are kept should not be subject to extremes,
whether of heat or cold, of moisture or dryness. It may be said that the
better adapted a room is for human occupation, the better for the books
it contains. Damp is, of course, most mischievous, but over-dryness
induced by heated air, especially when the pipes are in close proximity
to the bookcases, is also very injurious.''

\emph{Dust.}---Books should be taken from the shelves at least once a
year, dusted and aired, and the bindings rubbed with a preservative.

To dust a book, it should be removed from the shelf, and without being
opened, turned upside down and flicked with a feather duster. If a book
with the dust on the top is held loosely in the hand, and dusted right
way up, dust may fall between the leaves. Dusting should be done in
warm, dry weather; and afterwards, the books may be stood on the table
slightly open, to air, with their
leaves{\protect\hypertarget{Page_297}{}{{[}297{]}}} loose. Before being
returned to the shelves, the bindings should be lightly rubbed with some
preservative preparation (see \protect\hyperlink{CHAPTER_XXII}{chap.
XXII}). Any bindings that are broken, or any leaves that are loose
should be noted, and the books put on one side to be sent to the binder.
It would be best when the library is large enough to warrant it, to
employ a working bookbinder to do this work; such a man would be useful
in many ways. He could stick on labels, repair bindings, and do many
other odd jobs to keep the books in good repair.

A bookbinder could be kept fully employed, binding and repairing the
books of a comparatively small library under the direction of the
librarian.

\hypertarget{bookworms}{%
\subparagraph{BOOKWORMS}\label{bookworms}}

The insects known as bookworms are the larvæ of several sorts of
beetles, most commonly perhaps of \emph{Antobium domesticum} and
\emph{Niptus hololencus}. They are not in any way peculiar to books and
will infest the wood of bookshelves, walls, or floors. A good deal can
be done to keep ``worms'' away by using such
substances{\protect\hypertarget{Page_298}{}{{[}298{]}}} as camphor or
naphthaline in the bookcase. Bookworms do not attack modern books very
much; probably they dislike the alum put in the paste and the
mill-boards made of old tarred rope.

In old books, especially such as come from Italy, it is often found that
the ravages of the bookworms are almost entirely confined to the glue on
the backs of the books, and it generally seems that the glue and paste
attract them. Probably if corrosive sublimate were put in the glue and
paste used it would stop their attacks. Alum is said to be a preventive,
but I have known bookworms to eat their way through leather pasted on
with paste containing alum, when, in recovering, the old wooden boards
containing bookworms have been utilised in error.

When on shaking the boards of an old book dust flies out, or when little
heaps of dust are found on the shelf on which an old book has been
standing, it may be considered likely that there are bookworms present.
It is easy to kill any that may be hatched, by putting the book in an
air-tight box surrounded with cotton wool soaked in ether; but that will
not kill the eggs, and the treatment must be repeated
from{\protect\hypertarget{Page_299}{}{{[}299{]}}} time to time at
intervals of a few weeks.

Any book that is found to contain bookworms should be isolated and at
once treated. Tins may be put inside the boards to prevent the ``worms''
eating into the leaves.

Speaking of bookworms, Jules Cousin says:---

``One of the simplest means to be employed (to get rid of bookworms) is
to place behind the books, especially in the place where the insects
show their presence most, pieces of linen soaked with essence of
turpentine, camphor, or an infusion of tobacco, and to renew them when
the smell goes off. A little fine pepper might also be scattered on the
shelf, the penetrating smell of which would produce the same effect.''

Possibly Keating's Insect Powder would answer as well or better than
pepper.

\hypertarget{rats-and-mice}{%
\subparagraph{RATS AND MICE}\label{rats-and-mice}}

Rats and mice will gnaw the backs of books to get at the glue, so, means
should be taken to get rid of these vermin if they should appear. Mice
especially will{\protect\hypertarget{Page_300}{}{{[}300{]}}} nibble
vellum binding or the edges of vellum books that have become greasy with
much handling.

\hypertarget{cockroaches}{%
\subparagraph{COCKROACHES}\label{cockroaches}}

Cockroaches are very troublesome in libraries, eating the bindings.
Keating's Insect Powder will keep them away from books, but only so long
as it is renewed at short intervals.

\hypertarget{placing-the-books-in-the-shelves}{%
\subparagraph{PLACING THE BOOKS IN THE
SHELVES}\label{placing-the-books-in-the-shelves}}

The Chairman of the Society of Arts Special Committee says on this
point:---

``It is important that a just medium should be observed between the
close and loose disposition of books in the shelves. Tight packing
causes the pulling off of the tops of book-backs, injurious friction
between their sides, and undue pressure, which tends to force off their
backs. But books should not stand loosely on the shelves. They require
support and moderate lateral pressure, otherwise the leaves are apt to
open and admit dust, damp, and mildew. The weight of the leaves also
in{\protect\hypertarget{Page_301}{}{{[}301{]}}} good-sized volumes
loosely placed will often be found to be resting on the shelf, making
the backs concave, and spoiling the shape and cohesion of the books.

``In libraries where classification is attempted there must be a certain
number of partially filled shelves. The books in these should be kept in
place by some such device as that in use in the British Museum, namely,
a simple flat angle piece of galvanised iron, on the lower flange of
which the end books rest, keeping it down, the upright flange keeping
the books close and preventing them from spreading.''

He also speaks of the danger to bindings of rough or badly-painted
bookshelves:---

``Great care should be exercised when bookcases are painted or varnished
that the surface should be left hard, smooth, and dry. Bindings,
especially those of delicate texture, may be irreparably rubbed if
brought in contact with rough or coarsely-painted surfaces, while the
paint itself, years after its original application, is liable to come
off upon the books, leaving indelible marks. In such cases pasteboard
guards against the ends of the shelves are the only remedy.''

\hypertarget{chapter-xxii302}{%
\subsection[CHAPTER
XXII]{\texorpdfstring{\protect\hypertarget{CHAPTER_XXII}{}{}CHAPTER
XXII{\protect\hypertarget{Page_302}{}{{[}302{]}}}}{CHAPTER XXII{[}302{]}}}\label{chapter-xxii302}}

To Preserve Old Bindings---Re-backing

\hypertarget{to-preserve-old-bindings}{%
\subparagraph{TO PRESERVE OLD BINDINGS}\label{to-preserve-old-bindings}}

{It} is a well-known fact that the leather of bindings that are much
handled lasts very much better than that on books which remain untouched
on the shelves. There is little doubt that the reason for this is that
the slight amount of grease the leather receives from the hands
nourishes it and keeps it flexible. A coating of glair or varnish is
found to some extent to protect leather from adverse outside influences,
but, unfortunately, both glair and varnish tend rather to harden leather
than to keep it flexible, and they fail just where failure is most
serious, that is at the joints. In opening and shutting, any coat of
glair or varnish that has become hard will crack, and expose the leather
of the joint and back. Flexibility is an essential quality in
bookbinding leather, for as soon as the leather at the joint of a
binding becomes stiff it breaks away when the boards are
opened.{\protect\hypertarget{Page_303}{}{{[}303{]}}}

It would add immensely to the life of old leather bindings if librarians
would have them treated, say once a year, with some preservative. The
consequent expense would be saved many times over by the reduction of
the cost of rebinding. Such a preservative must not stain, must not
evaporate, must not become hard, and must not be sticky. Vaseline has
been recommended, and answers fairly well, but will evaporate, although
slowly. I have found that a solution of paraffin wax in castor oil
answers well. It is cheap and very simple to prepare. To prepare it,
some castor oil is put into an earthenware jar, and about half its
weight of paraffin wax shredded into it. On warming, the wax will melt,
and the preparation is ready for use.

A little of the preparation is well worked into a piece of flannel, and
the books rubbed with it, special attention being paid to the back and
joints. They may be further rubbed with the hand, and finally gone over
with a clean, soft cloth. Very little of the preparation need be used on
each book.

If bindings have projecting metal corners or clasps that are likely to
scratch the{\protect\hypertarget{Page_304}{}{{[}304{]}}} neighbouring
books, pieces of mill-board, which may be lined with leather or good
paper, should be placed next them, or they may have a cover made of a
piece of mill-board bent round as shown at
\protect\hyperlink{Fig_120}{fig. 120}, and strengthened at the folds
with linen. This may be slipped into the shelf with the book with the
open end outwards, and will then hardly be seen.

\protect\hypertarget{Fig_120}{}{}
\includegraphics[width=1.04167in,height=1.33333in]{images/gs305.jpg}

Fig. 120.

Bindings which have previously had metal clasps, \&c., often have
projecting fragments of the old nails. These should be sought for and
carefully removed or driven in, as they may seriously damage any
bindings with which they come in contact.

To protect valuable old bindings, cases may be made and lettered on the
back with the title of the book.

Loose covers that necessitate the bending back of the boards for their
removal are not
recommended.{\protect\hypertarget{Page_305}{}{{[}305{]}}}

\hypertarget{re-backing}{%
\subparagraph{RE-BACKING}\label{re-backing}}

Bindings that have broken joints may be re-backed. Any of the leather of
the back that remains should be carefully removed and preserved. It is
impossible to get some leathers off tight backs without destroying them,
but with care and by the use of a thin folder, many backs can be saved.
The leather on the boards is cut a little back from the joint with a
slanting cut, that will leave a thin edge, and is then lifted up with a
folder. New leather, of the same colour is pasted on the back, and
tucked in under the old leather on the board. The leather from the old
back should have its edges pared and any lumps of glue or paper removed
and be pasted on to the new leather and bound tightly with tape to make
sure that it sticks.

When the leather at the corners of the board needs repairing, the corner
is glued and tapped with a hammer to make it hard and square, and when
it is dry a little piece of new leather is slipped under the old and the
corner covered.

When the sewing cords or thread of
a{\protect\hypertarget{Page_306}{}{{[}306{]}}} book have perished it
should be rebound, but if there are any remains of the original binding
they should be preserved and utilised. If the old boards have quite
perished, new boards of the same nature and thickness should be got out
and the old cover pasted over them. Such places as the old leather will
not cover, must first be covered with new of the same colour. Generally
speaking, it is desirable that the characteristics of an old book should
be preserved, and that the new work should be as little in evidence as
possible. It is far more pleasant to see an old book in a patched
contemporary binding, than smug and tidy in the most immaculate modern
cover.

Part of the interest of any old book is its individual history, which
can be gathered from the binding, book-plates, marginal notes, names of
former owners, \&c., and anything that tends to obliterate these signs
is to be deplored.

\hypertarget{specifications307}{%
\subsection[SPECIFICATIONS]{\texorpdfstring{\protect\hypertarget{SPECIFICATIONS}{}{}SPECIFICATIONS{\protect\hypertarget{Page_307}{}{{[}307{]}}}}{SPECIFICATIONS{[}307{]}}}\label{specifications307}}

\hypertarget{specifications-for-bookbinding}{%
\subsection{SPECIFICATIONS FOR
BOOKBINDING}\label{specifications-for-bookbinding}}

These specifications will require modification in special cases, and are
only intended to be a general guide.

\begin{longtable}[]{@{}lcccc@{}}
\toprule()
\endhead
~ & I. For Extra Binding suitable for Valuable Books. Whole
Leather.{\protect\hypertarget{Page_308}{}{{[}308{]}}} & II. For Good
Binding for Books of Reference, Catalogues, \&c., and other heavy Books
that may have a great deal of use. Whole or Half Leather. & III. For
Binding for Libraries, for Books in current use. Half Leather. & IV. For
Library Bindings of Books of little Interest or Value, Cloth or Half
Linen. \\
SHEETS. & To be carefully folded, or, if an old book, all damaged leaves
to be carefully mended, the backs where damaged to be made sound. Single
leaves to be guarded round the sections next them. All plates to be
guarded. Guards to be sewn through. No pasting on or overcasting to be
allowed. & As No. I., excepting that any mending may be done rather with
a view to strength than extreme neatness. & Same as No. II. & Any leaves
damaged at the back or plates to be overcast into
sections.{\protect\hypertarget{Page_309}{}{{[}309{]}}} \\
END PAPERS. & To be sewn on. To be of good paper made with zigzag, with
board papers of self-coloured paper of good quality, or vellum. Or to be
made with leather joint. & To be of good paper made with zigzag, with
board papers of self-coloured paper of good quality. Large or heavy
books to have a cloth joint. To be sewn on. & To be of good paper, sewn
on, made with zigzag. & Same as No. III. \\
PRESSING. & Books on handmade paper not to be pressed
unduly.{\protect\hypertarget{Page_310}{}{{[}310{]}}} & Same as No. I. &
Same as No. I. & ~ \\
EDGES. & To be trimmed and gilt before sewing. To be uncut. & To be cut
and gilt in boards or coloured, or to be uncut. & To be uncut, or to be
cut in guillotine and gilt or coloured, or to have top edge only gilt. &
May be cut smooth in guillotine. \\
SEWING. & To be with ligature silk, flexible, round five bands of best
sewing cord. & To be with unbleached thread, flexible, round five bands
of best sewing cord. & To be with unbleached thread across not less than
four unbleached linen tapes. & With unbleached thread over three
unbleached linen tapes. \\
BACK. & To be kept as flat as it can be without forcing it and without
danger of its becoming concave in use. & Same as for No. I. & Same as
for Nos. I. and II. & Back to be left square after glueing
up.{\protect\hypertarget{Page_311}{}{{[}311{]}}} \\
BOARDS. & To be of the best black mill-board. Two boards to be made
together for large books, and all five bands laced in through two holes.
& Same as No. I., or may be of good grey board. & To be split grey
boards, or straw-board with black board liner, with ends of tapes
attached to portion of waste sheet, inserted between them. Boards to be
left a short distance from the joint to form a French joint. & To be
split boards, two straw-boards made together and ends of slips inserted.
French joint to be left. \\
HEADBANDS. & To be worked with silk on strips of vellum or catgut or
cord, with frequent tie-downs. The headbands to be ``set'' by pieces of
good paper or leather glued at head and tail. The back to be lined up
with leather all over if the book is large. & Same as No. I. & To be
worked with thread or vellum or cord, or to be omitted and a piece of
cord inserted into the turn in of the leather at head and tail in their
place. & No headbands. \\
COVERS. & Goatskin (morocco), pigskin or seal-skin manufactured
according to the recommendations of the Society of Arts' Committee on
Leather for Bookbinding. Whole binding; leather to be attached directly
to the back. & Same as No. I., excepting that properly prepared
sheepskin may be added. Half-binding, leather only at back. Corners to
be strengthened with tips of vellum. Sides covered with good paper or
linen. & Same as Nos. I. and II., but skins may be used where there are
surface flaws that do not affect the strength. Leather to be used
thicker than is usual, there being French joints. Leather at back only;
paper sides; vellum tips. & Whole buckram or half linen and paper
sides. \\
LETTERING. & To be legible and to identify the
volume.{\protect\hypertarget{Page_312}{}{{[}312{]}}} & Same as No. I. &
Same as Nos. I. and II. & Same as Nos. I. II. and III. \\
DECORATION. & To be as much or as little as the nature of the book
warrants. & To be omitted, or only to consist of a few lines or dots or
other quite simple ornament. & To be omitted. & To be omitted. \\
~ & All work to be done in the best manner. & Work may be a little
rougher, but not careless or dirty. & Same as No. II. & Same as No.
II. \\
\bottomrule()
\end{longtable}

\hypertarget{glossary313}{%
\subsection[GLOSSARY]{\texorpdfstring{\protect\hypertarget{GLOSSARY}{}{}GLOSSARY{\protect\hypertarget{Page_313}{}{{[}313{]}}}}{GLOSSARY{[}313{]}}}\label{glossary313}}

\emph{Arming press}, a small blocking press used for striking
arms-blocks on the sides of books.

\emph{Backing boards}, wedge-shaped bevelled boards used in backing (see
\protect\hyperlink{Fig_40}{Fig. 40}).

\emph{Backing machine}, used for backing cheap work in large quantities;
it often crushes and damages the backs of the sections.

\emph{Bands}, (1) the cords on which a book is sewn. (2) The ridges on
the back caused by the bands showing through the leather.

\emph{Band nippers}, pincers with flat jaws, used for straightening the
bands (see \protect\hyperlink{Fig_61}{Fig. 61}). For nipping up the
leather after covering, they should be nickelled to prevent the iron
staining the leather.

\emph{Beating stone}, the ``stone'' on which books were formerly beaten;
now generally superseded by the rolling machine and standing press.

\emph{Blind tooling}, the impression of finishing tools without gold.

\emph{Blocking press}, a press used for impressing blocks such as those
used in decorating cloth cases.

\emph{Board papers}, the part of the end papers pasted on to the boards.

\emph{Bodkin}, an awl used for making the holes in the boards for the
slips.

\emph{Bolt}, folded edge of the sheets in an unopened book.

\emph{Cancels}, leaves containing errors, which have to
be{\protect\hypertarget{Page_314}{}{{[}314{]}}} discarded and replaced
by corrected sheets. Such leaves are marked by the printer with a star.

\emph{Catch-word}, a word printed at the foot of one page indicating the
first word of the page following, as a guide in collating.

\emph{Cutting boards}, wedge-shaped boards somewhat like backing boards,
but with the top edge square; used in cutting the edge of a book and in
edge-gilding.

\emph{Cutting in boards}, cutting the edges of a book after the boards
are laced on.

\emph{Cutting press}, when the lying press is turned, so that the side
with the runners is uppermost, it is called a cutting press (see
\protect\hyperlink{Fig_46}{Fig. 46}).

\emph{Diaper}, a term applied to a small repeating all-over pattern.
From woven material decorated in this way.

\emph{Doublure}, the inside face of the boards, especially applied to
them when lined with leather and decorated.

\emph{End papers}, papers added at the beginning and end of a book by
the binder.

\emph{Extra binding}, a trade term for the best work.

\emph{Finishing}, comprises lettering, tooling, and polishing, \&c.

\emph{Finishing press}, a small press used for holding books when they
are being tooled (see \protect\hyperlink{Fig_84}{Fig. 84}).

\emph{Finishing stove}, used for heating finishing tools.

\emph{Folder}, a flat piece of ivory or bone, like a paper knife, used
in folding sheets and in various other operations.

\emph{Foredge} (fore edge), the front edge of the leaves. Pronounced
``forrege.''

\emph{Forwarding}, comprises all the operations between sewing and
finishing, excepting headbanding.

\emph{Gathering}, collecting one sheet from each pile in a printer's
warehouse to make up a
volume.{\protect\hypertarget{Page_315}{}{{[}315{]}}}

\emph{Glaire}, white of eggs beaten up, and used in finishing and edge
gilding.

\emph{Half binding}, when the leather covers the back and only part of
the sides, a book is said to be half bound.

\emph{Head band}, a fillet of silk or thread, worked at the head and
tail of the back.

\emph{Head cap}, the fold of leather over the head band (see
\protect\hyperlink{Fig_67}{Fig. 67}).

\emph{Head and tail}, the top and bottom of a book.

\emph{Imperfections}, sheets rejected by the binder and returned to the
printer to be replaced.

\emph{India proofs}, strictly first proofs only of an illustration
pulled on ``India paper,'' but used indiscriminately for all
illustrations printed on India paper.

\emph{Inset}, the portion of a sheet cut off and inserted in folding
certain sizes, such as duodecimo, \&c. (see
\protect\hyperlink{Fig_4}{Fig. 4}).

\emph{Inside margins}, the border made by the turn in of the leather on
the inside face of the boards (see \protect\hyperlink{Fig_116}{Fig.
116}).

\emph{Joints}, (1) the groove formed in backing to receive the ends of
the mill-boards. (2) The part of the binding that bends when the boards
are opened. (3) Strips of leather or cloth used to strengthen the end
papers.

``\emph{Kettle stitch},'' catch stitch formed in sewing at the head and
tail.

\emph{Lacing in}, lacing the slips through holes in the boards to attach
them.

\emph{Lying press}, the term applied to the under side of the cutting
press used for backing, usually ungrammatically called ``laying press.''

\emph{Marbling}, colouring the edges and end papers in various patterns,
obtained by floating colours on a gum solution.

\emph{Millboard machine}, machine used for squaring
boards;{\protect\hypertarget{Page_316}{}{{[}316{]}}} should only be used
for cheap work, as an edge cut by it will not be as square as if cut by
the plough.

\emph{Mitring}, (1) lines meeting at a right angle without overrunning
are said to be mitred. (2) A join at 45° as in the leather on the inside
of the boards.

\emph{Overcasting}, over-sewing the back edges of single leaves or weak
sections.

\emph{Peel}, a thin board on a handle used for hanging up sheets for
drying.

\emph{Plate}, an illustration printed from a plate. Term often
incorrectly applied to illustrations printed from woodcuts. Any
full-page illustration printed on different paper to the book is usually
called a ``plate.''

\emph{Pressing plates}, plates of metal japanned or nickelled, used for
giving finish to the leather on a book.

\emph{Press pin}, an iron bar used for turning the screws of presses.

\emph{Proof}, edges left uncut as ``proof'' that the book has not been
unduly cut down.

\emph{Register}, (i.) when the print on one side of a leaf falls exactly
over that on the other it is said to register. (ii.) Ribbon placed in a
book as a marker.

\emph{Rolling machine}, a machine in which the sheets of a book are
subject to heavy pressure by being passed between rollers.

\emph{Sawing in}, when grooves are made in the back with a saw to
receive the bands.

\emph{Section}, the folded sheet.

\emph{Semée} or \emph{Semis}, an heraldic term signifying sprinkled.

\emph{Set off}, print is said to ``set off'' when part of the ink from a
page comes off on an opposite page. This will happen if a book is
pressed too soon after
printing.{\protect\hypertarget{Page_317}{}{{[}317{]}}}

\emph{Sheet}, the full size of the paper as printed, forming a section
when folded.

\emph{Signature}, the letter or figure placed on the first page of each
sheet.

\emph{Slips}, the ends of the sewing cord or tape that are attached to
the boards.

\emph{Squares}, the portion of the boards projecting beyond the edges of
the book.

\emph{Start}, when, after cutting, one or more sections of the book come
forward, making the fore edge irregular, they are said to have started.

\emph{Straight edge}, a flat ruler.

\emph{Tacky}, sticky.

\emph{T.~E.~G.}, top-edge gilt.

\emph{Trimmed.} The edges of a book are said to be trimmed when the
edges of the larger (or projecting) leaves only have been cut.

\emph{Tub}, the stand which supports the lying press. Originally an
actual tub to catch the shavings.

\emph{Uncut}, a book is said to be uncut when the edges of the paper
have not been cut with the plough or guillotine.

\emph{Unopened}, the book is said to be unopened if the bolts of the
sheets have not been cut.

\emph{Waterproof sheets}, sheets of celluloid, such as are used by
photographers.

\emph{Whole binding}, when the leather covers the back and sides of a
volume.

\emph{Wire staples} are used by certain machines in the place of thread
for securing the sections.

\emph{Groove}, that part of the sections which is turned over in backing
to receive the board.{\protect\hypertarget{Page_318}{}{{[}318{]}}}

\hypertarget{reproductions-of-bindings319}{%
\subsection[REPRODUCTIONS OF
BINDINGS]{\texorpdfstring{\protect\hypertarget{REPRODUCTIONS_OF_BINDINGS}{}{}REPRODUCTIONS
OF
BINDINGS{\protect\hypertarget{Page_319}{}{{[}319{]}}}}{REPRODUCTIONS OF BINDINGS{[}319{]}}}\label{reproductions-of-bindings319}}

I., II., AND III.\\
{Fifteenth Century Blind-Tooled Bindings}

IV.\\
{Sixteenth Century Binding with Simple
Gold-Tooling}{\protect\hypertarget{Page_320}{}{{[}320{]}}}

V., VI., VII., AND VIII.\\
{Modern Bindings Designed by the
Author{\protect\hypertarget{Page_321}{}{{[}321{]}}}}

\href{images/plate01.jpg}{\includegraphics{images/plate01_th.jpg}}{\protect\hypertarget{Page_322}{}{{[}322{]}}}

I.---German Fifteenth Century. Pigskin. Actual size, 8¾″ ×
6¼″.{\protect\hypertarget{Page_323}{}{{[}323{]}}}

\href{images/plate02.jpg}{\includegraphics{images/plate02_th.jpg}}{\protect\hypertarget{Page_324}{}{{[}324{]}}}

II.---German Fifteenth Century. Calf. Actual size 12½″ ×
8½″.{\protect\hypertarget{Page_325}{}{{[}325{]}}}

\href{images/plate03.jpg}{\includegraphics{images/plate03_th.jpg}}{\protect\hypertarget{Page_326}{}{{[}326{]}}}

III.---Italian Fifteenth Century. Sheepskin, with coloured roundels.
Actual size, 11½″ × 8¼″.{\protect\hypertarget{Page_327}{}{{[}327{]}}}

~{\protect\hypertarget{Page_328}{}{{[}328{]}}}

\href{images/plate04.jpg}{\includegraphics{images/plate04_th.jpg}}{\protect\hypertarget{Page_329}{}{{[}329{]}}}

IV.---Italian Sixteenth Century. Actual size, 12½″ × 8½″.
Goatskin.{\protect\hypertarget{Page_330}{}{{[}330{]}}}

\href{images/plate05.jpg}{\includegraphics{images/plate05_th.jpg}}{\protect\hypertarget{Page_331}{}{{[}331{]}}}

V.---Half Niger morocco, with sides of English oak. Actual size, 17″ ×
11½″.{\protect\hypertarget{Page_332}{}{{[}332{]}}}

\href{images/plate06.jpg}{\includegraphics{images/plate06_th.jpg}}{\protect\hypertarget{Page_333}{}{{[}333{]}}}

VI.---Niger morocco, inlaid green leaves. Actual size, 8¼″ ×
5½″.{\protect\hypertarget{Page_334}{}{{[}334{]}}}

\href{images/plate07.jpg}{\includegraphics{images/plate07_th.jpg}}{\protect\hypertarget{Page_335}{}{{[}335{]}}}

VII.---Green levant, inlaid with lighter green panel and red dots.
Actual size, 6¾″ × 4½″.{\protect\hypertarget{Page_336}{}{{[}336{]}}}

\href{images/plate08.jpg}{\includegraphics{images/plate08_th.jpg}}

VIII.---Niger morocco, executed by a student of the Central School of
Arts and Crafts. Actual size, 11¾″ × 9¼″.

\hypertarget{index337}{%
\subsection[INDEX]{\texorpdfstring{\protect\hypertarget{INDEX}{}{}INDEX{\protect\hypertarget{Page_337}{}{{[}337{]}}}}{INDEX{[}337{]}}}\label{index337}}

\begin{itemize}
\tightlist
\item
  {Arming press}, \protect\hyperlink{Page_229}{229},
  \protect\hyperlink{Page_313}{313}
\item
  Arms blocks, \protect\hyperlink{Page_228}{228}
\item
  Art paper, \protect\hyperlink{Page_48}{48},
  \protect\hyperlink{Page_282}{282}
\item
  Autograph letters,
  \protect\hyperlink{Page_179}{179}{\protect\hypertarget{Page_338}{}{{[}338{]}}}
\item
  ~
\item
  {Backing}, \protect\hyperlink{Page_117}{117}
\item
  Backing hammer, \protect\hyperlink{Page_123}{123}
\item
  Back, lining up, \protect\hyperlink{Page_152}{152}
\item
  Band nippers, \protect\hyperlink{Page_160}{160},
  \protect\hyperlink{Page_163}{163}
\item
  Bands, \protect\hyperlink{Page_313}{313}
\item
  Bandstick, \protect\hyperlink{Page_160}{160}
\item
  Beating, \protect\hyperlink{Page_90}{90}
\item
  Beating stone, \protect\hyperlink{Page_90}{90},
  \protect\hyperlink{Page_313}{313}
\item
  Benzine, \protect\hyperlink{Page_207}{207},
  \protect\hyperlink{Page_209}{209}
\item
  Binding, decoration of, \protect\hyperlink{Page_21}{21},
  \protect\hyperlink{Page_30}{30}, \protect\hyperlink{Page_188}{188},
  \protect\hyperlink{Page_233}{233}
\item
  Binding, collotype reproductions of,
  \protect\hyperlink{Page_321}{321}-\protect\hyperlink{Page_336}{336}
\item
  Binding, embroidered, \protect\hyperlink{Page_186}{186}
\item
  Binding early printed books, \protect\hyperlink{Page_31}{31},
  \protect\hyperlink{Page_46}{46}, \protect\hyperlink{Page_113}{113}
\item
  Binding, extra, \protect\hyperlink{Page_308}{308}
\item
  Binding, jewelled, \protect\hyperlink{Page_263}{263}
\item
  Binding, library, \protect\hyperlink{Page_27}{27},
  \protect\hyperlink{Page_173}{173}, \protect\hyperlink{Page_308}{308}
\item
  Binding, manuscripts, \protect\hyperlink{Page_31}{31},
  \protect\hyperlink{Page_108}{108}, \protect\hyperlink{Page_113}{113},
  \protect\hyperlink{Page_125}{125}, \protect\hyperlink{Page_135}{135},
  \protect\hyperlink{Page_223}{223}
\item
  Binding, metal-covered, \protect\hyperlink{Page_263}{263}
\item
  Binding, vellum, \protect\hyperlink{Page_180}{180}
\item
  Binding very thin books, \protect\hyperlink{Page_177}{177}
\item
  Blind tooling, \protect\hyperlink{Page_188}{188},
  \protect\hyperlink{Page_222}{222}
\item
  Blocking press, \protect\hyperlink{Page_229}{229},
  \protect\hyperlink{Page_313}{313}
\item
  Blocks, striking, \protect\hyperlink{Page_229}{229}
\item
  Boards, \protect\hyperlink{Page_124}{124}
\item
  Boards, attaching, \protect\hyperlink{Page_132}{132}
\item
  Boards, cutting, \protect\hyperlink{Page_125}{125}
\item
  Boards, filling in, \protect\hyperlink{Page_170}{170}
\item
  Boards, lining, \protect\hyperlink{Page_129}{129}
\item
  Boards, pressing, \protect\hyperlink{Page_193}{193},
  \protect\hyperlink{Page_210}{210}
\item
  Boards, split, \protect\hyperlink{Page_28}{28},
  \protect\hyperlink{Page_175}{175}, \protect\hyperlink{Page_311}{311}
\item
  Bodkin, \protect\hyperlink{Page_114}{114}
\item
  Bookbinding as a profession, \protect\hyperlink{Page_32}{32}
\item
  Books in sheets, \protect\hyperlink{Page_34}{34}
\item
  Bookworms, \protect\hyperlink{Page_297}{297}
\item
  Borders, designing, \protect\hyperlink{Page_240}{240}
\item
  Borders, inside, \protect\hyperlink{Page_253}{253}
\item
  ~
\item
  {Calf}, \protect\hyperlink{Page_27}{27},
  \protect\hyperlink{Page_277}{277}
\item
  Cancelled sheets, \protect\hyperlink{Page_43}{43}
\item
  Cased books, \protect\hyperlink{Page_19}{19},
  \protect\hyperlink{Page_49}{49}
\item
  Castor oil, \protect\hyperlink{Page_303}{303}
\item
  Catch stitch, \protect\hyperlink{Page_99}{99}
\item
  Catch words, \protect\hyperlink{Page_314}{314}
\item
  Celluloid, sheets of, \protect\hyperlink{Page_161}{161}
\item
  Centres, designing, \protect\hyperlink{Page_241}{241}
\item
  Chrome leather, \protect\hyperlink{Page_276}{276}
\item
  Clasps and ties, \protect\hyperlink{Page_183}{183},
  \protect\hyperlink{Page_259}{259}
\item
  Cleaning off back, \protect\hyperlink{Page_137}{137}
\item
  Cloth casing, \protect\hyperlink{Page_19}{19},
  \protect\hyperlink{Page_49}{49}
\item
  Cloth joints, \protect\hyperlink{Page_86}{86},
  \protect\hyperlink{Page_257}{257}
\item
  Cobden-Sanderson, T.~J., xii., \protect\hyperlink{Page_22}{22}
\item
  Cockroaches, \protect\hyperlink{Page_300}{300}
\item
  Cocoanut oil, \protect\hyperlink{Page_200}{200}
\item
  {\protect\hypertarget{Page_339}{}{{[}339{]}}}Collating,
  \protect\hyperlink{Page_43}{43}
\item
  Colouring edges, \protect\hyperlink{Page_144}{144}
\item
  Combining tools to form patterns, \protect\hyperlink{Page_232}{232}
\item
  Compasses, \protect\hyperlink{Page_131}{131}
\item
  Cord sewing, \protect\hyperlink{Page_111}{111}
\item
  Corners, mitring, \protect\hyperlink{Page_165}{165},
  \protect\hyperlink{Page_168}{168}
\item
  Cousin, Jules, \protect\hyperlink{Page_74}{74},
  \protect\hyperlink{Page_299}{299}
\item
  Covering, \protect\hyperlink{Page_23}{23},
  \protect\hyperlink{Page_159}{159}, \protect\hyperlink{Page_176}{176},
  \protect\hyperlink{Page_310}{310}
\item
  Crushing the grain of leather, \protect\hyperlink{Page_192}{192}
\item
  Cutting in boards, \protect\hyperlink{Page_139}{139}
\item
  Cutting mill-boards, \protect\hyperlink{Page_124}{124}
\item
  Cutting press, \protect\hyperlink{Page_128}{128}
\item
  ~
\item
  {Damp}, effect of, on bindings, \protect\hyperlink{Page_294}{294}
\item
  Decoration of bindings, \protect\hyperlink{Page_21}{21},
  \protect\hyperlink{Page_30}{30}, \protect\hyperlink{Page_188}{188},
  \protect\hyperlink{Page_233}{233}
\item
  Designing tools, \protect\hyperlink{Page_230}{230}
\item
  Diaper patterns, \protect\hyperlink{Page_236}{236}
\item
  Dividers, \protect\hyperlink{Page_51}{51}
\item
  Dots, striking, \protect\hyperlink{Page_205}{205}
\item
  Doubluves, \protect\hyperlink{Page_253}{253},
  \protect\hyperlink{Page_314}{314}
\item
  Dressing for old bindings, \protect\hyperlink{Page_302}{302}
\item
  Dust and dusting, \protect\hyperlink{Page_296}{296}
\item
  ~
\item
  {Early} printed books, binding, \protect\hyperlink{Page_31}{31},
  \protect\hyperlink{Page_46}{46}, \protect\hyperlink{Page_113}{113}
\item
  Edge colouring, \protect\hyperlink{Page_144}{144}
\item
  Edge gauffering, \protect\hyperlink{Page_144}{144}
\item
  Edge gilding, \protect\hyperlink{Page_95}{95},
  \protect\hyperlink{Page_144}{144}
\item
  Edge sizing, \protect\hyperlink{Page_95}{95},
  \protect\hyperlink{Page_146}{146}
\item
  Edges, painted, \protect\hyperlink{Page_146}{146}
\item
  Embroidered bindings, \protect\hyperlink{Page_186}{186}
\item
  End papers, \protect\hyperlink{Page_80}{80},
  \protect\hyperlink{Page_254}{254}
\item
  End, painted, \protect\hyperlink{Page_83}{83}
\item
  End, vellum, \protect\hyperlink{Page_84}{84}
\item
  Ends, silk, \protect\hyperlink{Page_84}{84}
\item
  Entering, \protect\hyperlink{Page_33}{33}
\item
  Evelyn's Diary (quotation), \protect\hyperlink{Page_282}{282}
\item
  ``Extra'' binding, \protect\hyperlink{Page_308}{308},
  \protect\hyperlink{Page_314}{314}
\item
  ~
\item
  {False} bands, \protect\hyperlink{Page_26}{26}
\item
  Fillet, \protect\hyperlink{Page_190}{190},
  \protect\hyperlink{Page_206}{206}
\item
  Fillet, small, \protect\hyperlink{Page_206}{206},
  \protect\hyperlink{Page_246}{246}
\item
  Filling in boards, \protect\hyperlink{Page_170}{170}
\item
  Finishing, \protect\hyperlink{Page_191}{191}
\item
  Finishing press, \protect\hyperlink{Page_194}{194}
\item
  Finishing tools, \protect\hyperlink{Page_188}{188}
\item
  Finishing stove, \protect\hyperlink{Page_195}{195}
\item
  Flattening vellum, \protect\hyperlink{Page_65}{65}
\item
  Folder, \protect\hyperlink{Page_164}{164}
\item
  Folding, \protect\hyperlink{Page_36}{36}
\item
  Fraying out slips, \protect\hyperlink{Page_114}{114}
\item
  French joint, \protect\hyperlink{Page_176}{176}
\item
  French paring knife, \protect\hyperlink{Page_156}{156}
\item
  French standing press, \protect\hyperlink{Page_91}{91}
\item
  ~
\item
  {Gas} fumes, effect of, \protect\hyperlink{Page_291}{291}
\item
  Gathering, \protect\hyperlink{Page_35}{35}
\item
  Gauffering edges, \protect\hyperlink{Page_144}{144}
\item
  Gelatine, \protect\hyperlink{Page_70}{70}
\item
  Gilding edges, \protect\hyperlink{Page_95}{95},
  \protect\hyperlink{Page_144}{144}
\item
  Gilt top, \protect\hyperlink{Page_92}{92}
\item
  Glaire, \protect\hyperlink{Page_97}{97},
  \protect\hyperlink{Page_198}{198}
\item
  Glass, tinted, for libraries, \protect\hyperlink{Page_292}{292}
\item
  Glossary, \protect\hyperlink{Page_313}{313}
\item
  Glue, \protect\hyperlink{Page_289}{289}
\item
  Glueing up, \protect\hyperlink{Page_115}{115}
\item
  Goatskin, \protect\hyperlink{Page_277}{277}
\item
  Gold cushion, \protect\hyperlink{Page_200}{200}
\item
  Gold leaf, \protect\hyperlink{Page_199}{199}
\item
  Gold knife, \protect\hyperlink{Page_200}{200}
\item
  Gold, net for, \protect\hyperlink{Page_96}{96}
\item
  Gold, pad for, \protect\hyperlink{Page_201}{201}
\item
  Gold tooling, \protect\hyperlink{Page_188}{188},
  \protect\hyperlink{Page_191}{191}
\item
  Gouges, \protect\hyperlink{Page_189}{189},
  \protect\hyperlink{Page_205}{205}, \protect\hyperlink{Page_247}{247}
\item
  Groove (\emph{see} \protect\hyperlink{Joint}{Joint})
\item
  Guarding, \protect\hyperlink{Page_42}{42},
  \protect\hyperlink{Page_53}{53}
\item
  Guarding plates, \protect\hyperlink{Page_50}{50},
  \protect\hyperlink{Page_56}{56}, \protect\hyperlink{Page_316}{316}
\item
  ~
\item
  {Hammer}, backing, \protect\hyperlink{Page_123}{123}
\item
  Hand-made paper, \protect\hyperlink{Page_280}{280}
\item
  Headbanding, \protect\hyperlink{Page_108}{108},
  \protect\hyperlink{Page_147}{147}, \protect\hyperlink{Page_176}{176}
\item
  Headcaps, \protect\hyperlink{Page_156}{156},
  \protect\hyperlink{Page_166}{166}
\item
  Heat, effect of, on bindings, \protect\hyperlink{Page_295}{295}
\item
  {\protect\hypertarget{Page_340}{}{{[}340{]}}}Heraldry on bindings,
  \protect\hyperlink{Page_227}{227}
\item
  Hinging plates, \protect\hyperlink{Page_57}{57}
\item
  Hollow backs, \protect\hyperlink{Page_25}{25},
  \protect\hyperlink{Page_185}{185}
\item
  ~
\item
  {Imperfections}, \protect\hyperlink{Page_35}{35}
\item
  India proofs, soaking off, \protect\hyperlink{Page_62}{62}
\item
  India proofs, mounting, \protect\hyperlink{Page_63}{63}
\item
  Indiarubber for gold, \protect\hyperlink{Page_207}{207}
\item
  Inlaying leather, \protect\hyperlink{Page_213}{213},
  \protect\hyperlink{Page_232}{232}, \protect\hyperlink{Page_243}{243}
\item
  Inlaying leaves or plates, \protect\hyperlink{Page_64}{64}
\item
  Inset, \protect\hyperlink{Page_40}{40},
  \protect\hyperlink{Page_315}{315}
\item
  Inside margins, \protect\hyperlink{Page_253}{253}
\item
  ~
\item
  {Jaconet}, \protect\hyperlink{Page_60}{60},
  \protect\hyperlink{Page_64}{64}
\item
  Japanese paper, \protect\hyperlink{Page_282}{282}
\item
  Japanese vellum, \protect\hyperlink{Page_282}{282}
\item
  Jewelled bindings, \protect\hyperlink{Page_263}{263}
\item
  \protect\hypertarget{Joint}{}{}Joint,
  \protect\hyperlink{Page_165}{165}, \protect\hyperlink{Page_169}{169}
\item
  Joint, cloth, \protect\hyperlink{Page_86}{86},
  \protect\hyperlink{Page_257}{257}
\item
  Joint, French, \protect\hyperlink{Page_176}{176}
\item
  Joint, knocking out, \protect\hyperlink{Page_53}{53}
\item
  Joint, leather, \protect\hyperlink{Page_86}{86},
  \protect\hyperlink{Page_171}{171}
\item
  ~
\item
  {Kettle} stitch, \protect\hyperlink{Page_49}{49},
  \protect\hyperlink{Page_99}{99}, \protect\hyperlink{Page_105}{105}
\item
  Keys, sewing, \protect\hyperlink{Page_101}{101}
\item
  Knife, mountcutters', \protect\hyperlink{Page_54}{54}
\item
  Knife, French paring, \protect\hyperlink{Page_156}{156}
\item
  Knife, gold, \protect\hyperlink{Page_200}{200}
\item
  Knife, plough, \protect\hyperlink{Page_129}{129},
  \protect\hyperlink{Page_139}{139}
\item
  Knocking down iron, \protect\hyperlink{Page_53}{53},
  \protect\hyperlink{Page_134}{134}
\item
  Knocking out joints, \protect\hyperlink{Page_53}{53}
\item
  Knot, \protect\hyperlink{Page_100}{100},
  \protect\hyperlink{Page_106}{106}
\item
  ~
\item
  {Lacing} in slips, \protect\hyperlink{Page_132}{132}
\item
  Lay cords, \protect\hyperlink{Page_100}{100}
\item
  Laying press (\emph{see} \protect\hyperlink{Lying_press}{Lying press})
\item
  Leather, \protect\hyperlink{Page_27}{27},
  \protect\hyperlink{Page_263}{263}
\item
  Leather, chrome, \protect\hyperlink{Page_276}{276}
\item
  Leather, crushing grain of, \protect\hyperlink{Page_192}{192}
\item
  Leather, inlaying, \protect\hyperlink{Page_213}{213},
  \protect\hyperlink{Page_232}{232}, \protect\hyperlink{Page_243}{243}
\item
  Leather joints, \protect\hyperlink{Page_86}{86},
  \protect\hyperlink{Page_171}{171}
\item
  Leather, paring, \protect\hyperlink{Page_154}{154}
\item
  Leather, polishing, \protect\hyperlink{Page_191}{191}
\item
  Leather, sprinkling and marbling, \protect\hyperlink{Page_27}{27},
  \protect\hyperlink{Page_279}{279}
\item
  Leather, stretching, \protect\hyperlink{Page_23}{23},
  \protect\hyperlink{Page_161}{161}
\item
  Leather, testing, \protect\hyperlink{Page_274}{274}
\item
  Leather work, \protect\hyperlink{Page_226}{226}
\item
  Leaves, inlaying, \protect\hyperlink{Page_64}{64}
\item
  Lettering, \protect\hyperlink{Page_28}{28},
  \protect\hyperlink{Page_215}{215}, \protect\hyperlink{Page_246}{246}
\item
  Letters, autograph, \protect\hyperlink{Page_179}{179}
\item
  Library binding, \protect\hyperlink{Page_27}{27},
  \protect\hyperlink{Page_173}{173}, \protect\hyperlink{Page_308}{308}
\item
  Light, effect of, on leather, \protect\hyperlink{Page_292}{292}
\item
  Lining up back, \protect\hyperlink{Page_152}{152}
\item
  Lithographic stone, \protect\hyperlink{Page_157}{157},
  \protect\hyperlink{Page_160}{160}
\item
  Loose covers, \protect\hyperlink{Page_304}{304}
\item
  \protect\hypertarget{Lying_press}{}{}Lying press,
  \protect\hyperlink{Page_128}{128}
\item
  ~
\item
  {Manuscripts}, binding of, \protect\hyperlink{Page_31}{31},
  \protect\hyperlink{Page_108}{108}, \protect\hyperlink{Page_113}{113},
  \protect\hyperlink{Page_125}{125}, \protect\hyperlink{Page_135}{135},
  \protect\hyperlink{Page_223}{223}
\item
  Manuscripts, collating, \protect\hyperlink{Page_46}{46}
\item
  Maps, throwing out, \protect\hyperlink{Page_60}{60}
\item
  Marbled paper, \protect\hyperlink{Page_83}{83}
\item
  Margins, inside, \protect\hyperlink{Page_253}{253}
\item
  Marking up, \protect\hyperlink{Page_98}{98}
\item
  Materials for sewing, \protect\hyperlink{Page_111}{111}
\item
  Mending, \protect\hyperlink{Page_76}{76}
\item
  Mending tooling, \protect\hyperlink{Page_208}{208}
\item
  Mending vellum, \protect\hyperlink{Page_79}{79}
\item
  Metal on bindings, \protect\hyperlink{Page_262}{262}
\item
  Millboards, \protect\hyperlink{Page_124}{124}
\item
  Millboard machine, \protect\hyperlink{Page_127}{127},
  \protect\hyperlink{Page_315}{315}
\item
  Millboard shears, \protect\hyperlink{Page_126}{126}
\item
  Mitring corners, \protect\hyperlink{Page_165}{165},
  \protect\hyperlink{Page_168}{168}
\item
  Morocco, \protect\hyperlink{Page_277}{277}
\item
  Morocco, ``Persian,'' \protect\hyperlink{Page_271}{271}
\item
  Mount-cutters' knife, \protect\hyperlink{Page_54}{54}
\item
  Mounting India-proofs, \protect\hyperlink{Page_63}{63}
\item
  Mounting very thin paper, \protect\hyperlink{Page_63}{63}
\item
  ~
\item
  {Net} for gilding edges, \protect\hyperlink{Page_96}{96}
\item
  Niger morocco, \protect\hyperlink{Page_278}{278}
\item
  Nipping press, \protect\hyperlink{Page_211}{211}
\item
  Nippers, band, \protect\hyperlink{Page_160}{160},
  \protect\hyperlink{Page_163}{163}
\item
  ~
\item
  {\protect\hypertarget{Page_341}{}{{[}341{]}}}{Oil}, cocoanut,
  \protect\hyperlink{Page_200}{200}
\item
  Opening newly-bound books, \protect\hyperlink{Page_257}{257}
\item
  Overcasting, \protect\hyperlink{Page_51}{51}
\item
  ``Overs,'' \protect\hyperlink{Page_35}{35}
\item
  Oxalic acid, use of, \protect\hyperlink{Page_173}{173}
\item
  ~
\item
  {Pad} for gold, \protect\hyperlink{Page_201}{201}
\item
  Paging, \protect\hyperlink{Page_44}{44}
\item
  Painted edges, \protect\hyperlink{Page_146}{146}
\item
  Painted end papers, \protect\hyperlink{Page_83}{83}
\item
  Pallets, \protect\hyperlink{Page_189}{189}
\item
  Paper, \protect\hyperlink{Page_280}{280}
\item
  Paper, art, \protect\hyperlink{Page_48}{48},
  \protect\hyperlink{Page_283}{283}
\item
  Paper, hand-made, \protect\hyperlink{Page_280}{280}
\item
  Paper, Japanese, \protect\hyperlink{Page_282}{282}
\item
  Paper, marbled, \protect\hyperlink{Page_83}{83}
\item
  Paper, sizes of, \protect\hyperlink{Page_36}{36},
  \protect\hyperlink{Page_283}{283}
\item
  Paper, sizing, \protect\hyperlink{Page_67}{67}
\item
  Paper, splitting, \protect\hyperlink{Page_63}{63}
\item
  Paper, washing, \protect\hyperlink{Page_71}{71}
\item
  Paraffin wax, \protect\hyperlink{Page_303}{303}
\item
  Paring leather, \protect\hyperlink{Page_154}{154}
\item
  Paring paper, \protect\hyperlink{Page_61}{61}
\item
  Paring stone, \protect\hyperlink{Page_157}{157},
  \protect\hyperlink{Page_160}{160}
\item
  Pastes, \protect\hyperlink{Page_286}{286}
\item
  Paste water, \protect\hyperlink{Page_198}{198}
\item
  Pasting down end papers, \protect\hyperlink{Page_254}{254}
\item
  Patterns, \protect\hyperlink{Page_232}{232}
\item
  ``Peel,'' \protect\hyperlink{Page_316}{316}
\item
  Permanent binding, \protect\hyperlink{Page_19}{19}
\item
  ``Persian'' morocco, \protect\hyperlink{Page_271}{271}
\item
  Pigskin, \protect\hyperlink{Page_278}{278}
\item
  Plates, detaching, \protect\hyperlink{Page_48}{48}
\item
  Plates, guarding, \protect\hyperlink{Page_56}{56}
\item
  Plates, hinging, \protect\hyperlink{Page_57}{57}
\item
  Plates, inlaying, \protect\hyperlink{Page_64}{64}
\item
  Plates, trimming, \protect\hyperlink{Page_40}{40}
\item
  Plough, \protect\hyperlink{Page_128}{128}
\item
  Plough knife, \protect\hyperlink{Page_129}{129},
  \protect\hyperlink{Page_139}{139}
\item
  Polishing, \protect\hyperlink{Page_191}{191}
\item
  Preserving old bindings, \protect\hyperlink{Page_302}{302}
\item
  Press, arming, \protect\hyperlink{Page_229}{229},
  \protect\hyperlink{Page_313}{313}
\item
  Press, blocking, \protect\hyperlink{Page_229}{229},
  \protect\hyperlink{Page_313}{313}
\item
  Press, cutting, \protect\hyperlink{Page_128}{128}
\item
  Press, finishing, \protect\hyperlink{Page_194}{194}
\item
  Press, lying, \protect\hyperlink{Page_128}{128}
\item
  Press, nipping, \protect\hyperlink{Page_211}{211}
\item
  Press pin, \protect\hyperlink{Page_316}{316}
\item
  Press, sewing (\emph{see} \protect\hyperlink{Sewing_frame}{Sewing
  frame})
\item
  Press, standing, \protect\hyperlink{Page_88}{88}
\item
  Pressing boards, \protect\hyperlink{Page_193}{193},
  \protect\hyperlink{Page_210}{210}
\item
  Pressing in boards, \protect\hyperlink{Page_138}{138}
\item
  Pressing plates, \protect\hyperlink{Page_192}{192},
  \protect\hyperlink{Page_316}{316}
\item
  Pressing sections, \protect\hyperlink{Page_87}{87}
\item
  ``Proof,'' \protect\hyperlink{Page_316}{316}
\item
  Publishers' binding, \protect\hyperlink{Page_20}{20}
\item
  Pulling to pieces, \protect\hyperlink{Page_46}{46}
\item
  ~
\item
  {Quarter} sections, \protect\hyperlink{Page_42}{42}
\item
  Quires, books in, \protect\hyperlink{Page_34}{34}
\item
  ~
\item
  {Rats} and mice, \protect\hyperlink{Page_299}{299}
\item
  Re-backing, \protect\hyperlink{Page_305}{305}
\item
  Re-binding, \protect\hyperlink{Page_18}{18},
  \protect\hyperlink{Page_306}{306}
\item
  Refolding, \protect\hyperlink{Page_51}{51}
\item
  Register of printing, \protect\hyperlink{Page_52}{52},
  \protect\hyperlink{Page_316}{316}
\item
  Representations of bindings,
  \protect\hyperlink{Page_321}{321}-\protect\hyperlink{Page_336}{336}
\item
  Roll, \protect\hyperlink{Page_190}{190}
\item
  Rounding, \protect\hyperlink{Page_117}{117}
\item
  ~
\item
  {Sawing} in, \protect\hyperlink{Page_20}{20},
  \protect\hyperlink{Page_25}{25}, \protect\hyperlink{Page_100}{100},
  \protect\hyperlink{Page_108}{108}
\item
  Scrap books, \protect\hyperlink{Page_178}{178}
\item
  Sealskin, \protect\hyperlink{Page_278}{278}
\item
  Sections, pressing, \protect\hyperlink{Page_87}{87}
\item
  Sewing, \protect\hyperlink{Page_100}{100}
\item
  Sewing cord, \protect\hyperlink{Page_111}{111}
\item
  \protect\hypertarget{Sewing_frame}{}{}Sewing frame,
  \protect\hyperlink{Page_100}{100}
\item
  Sewing keys, \protect\hyperlink{Page_101}{101}
\item
  Sewing on tapes, \protect\hyperlink{Page_26}{26},
  \protect\hyperlink{Page_111}{111}, \protect\hyperlink{Page_174}{174}
\item
  Sewing on vellum slips, \protect\hyperlink{Page_111}{111},
  \protect\hyperlink{Page_181}{181}
\item
  Sewing silk, \protect\hyperlink{Page_112}{112}
\item
  Sewing, tape for, \protect\hyperlink{Page_112}{112}
\item
  Sewing thread, \protect\hyperlink{Page_112}{112}
\item
  Sheepskin,
  \protect\hyperlink{Page_277}{277}-\protect\hyperlink{Page_308}{308}
\item
  {\protect\hypertarget{Page_342}{}{{[}342{]}}}Sheets, books in,
  \protect\hyperlink{Page_34}{34}
\item
  Sheets, waterproof, \protect\hyperlink{Page_161}{161}
\item
  Signatures, \protect\hyperlink{Page_34}{34},
  \protect\hyperlink{Page_43}{43}
\item
  Silk ends, \protect\hyperlink{Page_84}{84}
\item
  Silk sewing, \protect\hyperlink{Page_112}{112}
\item
  Sizes of paper, \protect\hyperlink{Page_36}{36},
  \protect\hyperlink{Page_283}{283}
\item
  Sizing, \protect\hyperlink{Page_67}{67}
\item
  Sizing edges,
  \protect\hyperlink{Page_95}{95}-\protect\hyperlink{Page_146}{146}
\item
  Sizing leather, \protect\hyperlink{Page_198}{198}
\item
  Sizing paper, \protect\hyperlink{Page_67}{67}
\item
  Slips, \protect\hyperlink{Page_317}{317}
\item
  Slips, fraying out, \protect\hyperlink{Page_114}{114}
\item
  Slips, lacing in, \protect\hyperlink{Page_132}{132}
\item
  Soaking off India proofs, \protect\hyperlink{Page_62}{62}
\item
  Society of Arts, Report of Committee on Leather for Bookbinding,
  \protect\hyperlink{Page_22}{22}, \protect\hyperlink{Page_264}{264}
\item
  Society of Arts, Report of Committee on Paper,
  \protect\hyperlink{Page_284}{284}
\item
  Specifications, \protect\hyperlink{Page_308}{308}
\item
  Split boards, \protect\hyperlink{Page_28}{28},
  \protect\hyperlink{Page_175}{175}, \protect\hyperlink{Page_311}{311}
\item
  Splitting paper, \protect\hyperlink{Page_63}{63}
\item
  Sprinkling leather, \protect\hyperlink{Page_27}{27},
  \protect\hyperlink{Page_279}{279}
\item
  Squares, \protect\hyperlink{Page_131}{131},
  \protect\hyperlink{Page_153}{153}, \protect\hyperlink{Page_317}{317}
\item
  Standing press, \protect\hyperlink{Page_88}{88}
\item
  Standing press, French, \protect\hyperlink{Page_89}{89},
  \protect\hyperlink{Page_91}{91}
\item
  Staples, wire, \protect\hyperlink{Page_49}{49}
\item
  ``Starred'' sheets, \protect\hyperlink{Page_43}{43}
\item
  Stove, finishing, \protect\hyperlink{Page_195}{195}
\item
  Stone, lithographic, \protect\hyperlink{Page_157}{157},
  \protect\hyperlink{Page_160}{160}
\item
  Striking dots, \protect\hyperlink{Page_205}{205}
\item
  Striking tools, \protect\hyperlink{Page_204}{204}
\item
  ~
\item
  {Tape}, sewing on, \protect\hyperlink{Page_26}{26},
  \protect\hyperlink{Page_112}{112}, \protect\hyperlink{Page_174}{174}
\item
  Temporary binding, \protect\hyperlink{Page_20}{20}
\item
  Testing leather, \protect\hyperlink{Page_274}{274}
\item
  Thin books, binding, \protect\hyperlink{Page_177}{177}
\item
  Thin paper, mounting, \protect\hyperlink{Page_63}{63}
\item
  Thread, sewing, \protect\hyperlink{Page_112}{112}
\item
  Throwing out maps, \protect\hyperlink{Page_60}{60}
\item
  Ties and clasps, \protect\hyperlink{Page_183}{183},
  \protect\hyperlink{Page_259}{259}
\item
  Tobacco smoke, effect of, on binding,
  \protect\hyperlink{Page_294}{294}
\item
  Tooling, blind, \protect\hyperlink{Page_188}{188},
  \protect\hyperlink{Page_222}{222}
\item
  Tooling, gold, \protect\hyperlink{Page_24}{24},
  \protect\hyperlink{Page_188}{188}, \protect\hyperlink{Page_191}{191}
\item
  Tooling on vellum, \protect\hyperlink{Page_212}{212}
\item
  Tools, designing, \protect\hyperlink{Page_188}{188},
  \protect\hyperlink{Page_230}{230}
\item
  Tools, finishing, \protect\hyperlink{Page_188}{188},
  \protect\hyperlink{Page_230}{230}
\item
  Training for bookbinding, \protect\hyperlink{Page_32}{32}
\item
  Trimming before sewing, \protect\hyperlink{Page_93}{93}
\item
  Trimming machine, \protect\hyperlink{Page_94}{94}
\item
  Trimming plates, \protect\hyperlink{Page_40}{40}
\item
  Tub, \protect\hyperlink{Page_317}{317}
\item
  Tying up, \protect\hyperlink{Page_167}{167}
\item
  ~
\item
  {Varnish}, \protect\hyperlink{Page_209}{209}
\item
  Vellum binders, \protect\hyperlink{Page_26}{26}
\item
  Vellum bindings, \protect\hyperlink{Page_180}{180}
\item
  Vellum ends, \protect\hyperlink{Page_84}{84}
\item
  Vellum, flattening, \protect\hyperlink{Page_65}{65}
\item
  Vellum, Japanese, \protect\hyperlink{Page_282}{282}
\item
  Vellum, mending, \protect\hyperlink{Page_79}{79}
\item
  Vellum slips, sewing on, \protect\hyperlink{Page_111}{111},
  \protect\hyperlink{Page_183}{183}
\item
  Vellum tooling on, \protect\hyperlink{Page_212}{212}
\item
  ~
\item
  {Walker}, Emery, \protect\hyperlink{Page_216}{216}
\item
  Washing, \protect\hyperlink{Page_71}{71}
\item
  Waterproof sheets, \protect\hyperlink{Page_161}{161}
\item
  Weaver's knot, \protect\hyperlink{Page_106}{106}
\item
  Wooden boards, \protect\hyperlink{Page_32}{32},
  \protect\hyperlink{Page_135}{135}, \protect\hyperlink{Page_223}{223},
  \protect\hyperlink{Page_330}{330}
\item
  Worm holes, \protect\hyperlink{Page_78}{78},
  \protect\hyperlink{Page_297}{297}
\end{itemize}

\hypertarget{the-artistic-crafts-series-of-technical-handbooks.343}{%
\subsection[THE ARTISTIC CRAFTS SERIES OF TECHNICAL
HANDBOOKS.]{\texorpdfstring{\protect\hypertarget{THE_ARTISTIC_CRAFTS_SERIES_OF}{}{}THE
ARTISTIC CRAFTS SERIES OF TECHNICAL
HANDBOOKS.{\protect\hypertarget{Page_343}{}{{[}343{]}}}}{THE ARTISTIC CRAFTS SERIES OF TECHNICAL HANDBOOKS.{[}343{]}}}\label{the-artistic-crafts-series-of-technical-handbooks.343}}

Edited by {W.~R. Lethaby}.

{The} series will appeal to handicraftsmen in the industrial and
mechanic arts. It consists of authoritative statements by experts in
every field for the exercise of ingenuity, taste, imagination---the
whole sphere of the so-called ``dependent arts.''

\begin{center}\rule{0.5\linewidth}{0.5pt}\end{center}

BOOKBINDING AND THE CARE OF BOOKS. A Handbook for Amateurs, Bookbinders,
and Librarians. By {Douglas Cockerell}. With 120 Illustrations and
Diagrams by Noel Rooke, and 8 collotype reproductions of binding. 12mo.
\$1.25 net.

SILVERWORK AND JEWELRY. A Text-Book for Students and Workers in Metal.
By {H. Wilson}. With 160 Diagrams and 16 full-page Illustrations, 12mo.
\$1.40 net.

WOOD CARVING: DESIGN AND WORKMANSHIP. By {George Jack}. With Drawings by
the Author and other Illustrations. \$1.40 net.

STAINED-GLASS WORK. A Text-Book for Students and Workers in Glass. By
{C.~W. Whall}. With Diagrams by two of his Apprentices, and other
Illustrations. \$1.50 net; postage, 14 cents additional.

\begin{center}\rule{0.5\linewidth}{0.5pt}\end{center}

D. APPLETON AND COMPANY, NEW YORK.

\hypertarget{transcribers-note}{%
\paragraph{Transcriber\textquotesingle s Note}\label{transcribers-note}}

Obvious typographical errors have been corrected. Spelling has been
normalized. For the detailed list please see the list below. If your
cursor turns into a hand while you hover it over an illustration, the
click on that illustration will open its larger version.

\begin{itemize}
\tightlist
\item
  page \protect\hyperlink{Page_14}{014}---typo fixed: changed
  \textquotesingle Making\textquotesingle{} to
  \textquotesingle Marking\textquotesingle{}
\item
  page \protect\hyperlink{Page_138}{138}---spelling normalized: changed
  \textquotesingle head-banding\textquotesingle{} to
  \textquotesingle headbanding\textquotesingle{}
\item
  page \protect\hyperlink{Page_159}{159}---typo fixed: changed
  \textquotesingle wook\textquotesingle{} to
  \textquotesingle wood\textquotesingle{}
\item
  page \protect\hyperlink{Page_173}{173}---typo fixed: changed
  \textquotesingle CHAPTER VIII\textquotesingle{} to
  \textquotesingle CHAPTER XIII\textquotesingle{}
\item
  page \protect\hyperlink{Page_198}{198}---typo fixed: changed
  \textquotesingle isinglas\textquotesingle{} to
  \textquotesingle isinglass\textquotesingle{}
\item
  page \protect\hyperlink{Page_249}{249}---spelling normalized: changed
  \textquotesingle tie downs\textquotesingle{} to
  \textquotesingle tie-downs\textquotesingle{}
\item
  page \protect\hyperlink{Page_253}{253}---spelling normalized: changed
  \textquotesingle headcap\textquotesingle{} to
  \textquotesingle head-cap\textquotesingle{}
\item
  page \protect\hyperlink{Page_298}{298}---spelling normalized: changed
  \textquotesingle millboard\textquotesingle{} to
  \textquotesingle mill-board\textquotesingle{}
\item
  page \protect\hyperlink{Page_303}{303}---spelling normalized: changed
  \textquotesingle re-binding\textquotesingle{} to
  \textquotesingle rebinding\textquotesingle{}
\item
  page \protect\hyperlink{Page_304}{304}---spelling normalized: changed
  \textquotesingle millboard\textquotesingle{} to
  \textquotesingle mill-board\textquotesingle{}
\item
  page \protect\hyperlink{Page_310}{310}---spelling normalized: changed
  \textquotesingle Goat-skin\textquotesingle{} to
  \textquotesingle Goatskin\textquotesingle{}
\item
  page \protect\hyperlink{Page_314}{314}---spelling normalized: changed
  \textquotesingle head-banding\textquotesingle{} to
  \textquotesingle headbanding\textquotesingle{}
\item
  page \protect\hyperlink{Page_315}{315}---spelling normalized: changed
  \textquotesingle millboards\textquotesingle{} to
  \textquotesingle mill-boards\textquotesingle{}
\item
  page \protect\hyperlink{Page_339}{339}---spelling normalized: changed
  \textquotesingle millboards\textquotesingle{} to
  \textquotesingle mill-boards\textquotesingle{}
\item
  page \protect\hyperlink{Page_341}{341}---spelling normalized: changed
  \textquotesingle Re-folding\textquotesingle{} to
  \textquotesingle Refolding\textquotesingle{}
\end{itemize}

\begin{verbatim}





End of the Project Gutenberg EBook of Bookbinding, and the Care of Books, by 
Douglas Cockerell

*** END OF THIS PROJECT GUTENBERG EBOOK BOOKBINDING, AND THE CARE OF BOOKS ***

***** This file should be named 26672-h.htm or 26672-h.zip *****
This and all associated files of various formats will be found in:
        http://www.gutenberg.org/2/6/6/7/26672/

Produced by Suzanne Shell, Irma Spehar and the Online
Distributed Proofreading Team at http://www.pgdp.net


Updated editions will replace the previous one--the old editions
will be renamed.

Creating the works from public domain print editions means that no
one owns a United States copyright in these works, so the Foundation
(and you!) can copy and distribute it in the United States without
permission and without paying copyright royalties.  Special rules,
set forth in the General Terms of Use part of this license, apply to
copying and distributing Project Gutenberg-tm electronic works to
protect the PROJECT GUTENBERG-tm concept and trademark.  Project
Gutenberg is a registered trademark, and may not be used if you
charge for the eBooks, unless you receive specific permission.  If you
do not charge anything for copies of this eBook, complying with the
rules is very easy.  You may use this eBook for nearly any purpose
such as creation of derivative works, reports, performances and
research.  They may be modified and printed and given away--you may do
practically ANYTHING with public domain eBooks.  Redistribution is
subject to the trademark license, especially commercial
redistribution.



*** START: FULL LICENSE ***

THE FULL PROJECT GUTENBERG LICENSE
PLEASE READ THIS BEFORE YOU DISTRIBUTE OR USE THIS WORK

To protect the Project Gutenberg-tm mission of promoting the free
distribution of electronic works, by using or distributing this work
(or any other work associated in any way with the phrase "Project
Gutenberg"), you agree to comply with all the terms of the Full Project
Gutenberg-tm License (available with this file or online at
http://gutenberg.org/license).


Section 1.  General Terms of Use and Redistributing Project Gutenberg-tm
electronic works

1.A.  By reading or using any part of this Project Gutenberg-tm
electronic work, you indicate that you have read, understand, agree to
and accept all the terms of this license and intellectual property
(trademark/copyright) agreement.  If you do not agree to abide by all
the terms of this agreement, you must cease using and return or destroy
all copies of Project Gutenberg-tm electronic works in your possession.
If you paid a fee for obtaining a copy of or access to a Project
Gutenberg-tm electronic work and you do not agree to be bound by the
terms of this agreement, you may obtain a refund from the person or
entity to whom you paid the fee as set forth in paragraph 1.E.8.

1.B.  "Project Gutenberg" is a registered trademark.  It may only be
used on or associated in any way with an electronic work by people who
agree to be bound by the terms of this agreement.  There are a few
things that you can do with most Project Gutenberg-tm electronic works
even without complying with the full terms of this agreement.  See
paragraph 1.C below.  There are a lot of things you can do with Project
Gutenberg-tm electronic works if you follow the terms of this agreement
and help preserve free future access to Project Gutenberg-tm electronic
works.  See paragraph 1.E below.

1.C.  The Project Gutenberg Literary Archive Foundation ("the Foundation"
or PGLAF), owns a compilation copyright in the collection of Project
Gutenberg-tm electronic works.  Nearly all the individual works in the
collection are in the public domain in the United States.  If an
individual work is in the public domain in the United States and you are
located in the United States, we do not claim a right to prevent you from
copying, distributing, performing, displaying or creating derivative
works based on the work as long as all references to Project Gutenberg
are removed.  Of course, we hope that you will support the Project
Gutenberg-tm mission of promoting free access to electronic works by
freely sharing Project Gutenberg-tm works in compliance with the terms of
this agreement for keeping the Project Gutenberg-tm name associated with
the work.  You can easily comply with the terms of this agreement by
keeping this work in the same format with its attached full Project
Gutenberg-tm License when you share it without charge with others.

1.D.  The copyright laws of the place where you are located also govern
what you can do with this work.  Copyright laws in most countries are in
a constant state of change.  If you are outside the United States, check
the laws of your country in addition to the terms of this agreement
before downloading, copying, displaying, performing, distributing or
creating derivative works based on this work or any other Project
Gutenberg-tm work.  The Foundation makes no representations concerning
the copyright status of any work in any country outside the United
States.

1.E.  Unless you have removed all references to Project Gutenberg:

1.E.1.  The following sentence, with active links to, or other immediate
access to, the full Project Gutenberg-tm License must appear prominently
whenever any copy of a Project Gutenberg-tm work (any work on which the
phrase "Project Gutenberg" appears, or with which the phrase "Project
Gutenberg" is associated) is accessed, displayed, performed, viewed,
copied or distributed:

This eBook is for the use of anyone anywhere at no cost and with
almost no restrictions whatsoever.  You may copy it, give it away or
re-use it under the terms of the Project Gutenberg License included
with this eBook or online at www.gutenberg.org

1.E.2.  If an individual Project Gutenberg-tm electronic work is derived
from the public domain (does not contain a notice indicating that it is
posted with permission of the copyright holder), the work can be copied
and distributed to anyone in the United States without paying any fees
or charges.  If you are redistributing or providing access to a work
with the phrase "Project Gutenberg" associated with or appearing on the
work, you must comply either with the requirements of paragraphs 1.E.1
through 1.E.7 or obtain permission for the use of the work and the
Project Gutenberg-tm trademark as set forth in paragraphs 1.E.8 or
1.E.9.

1.E.3.  If an individual Project Gutenberg-tm electronic work is posted
with the permission of the copyright holder, your use and distribution
must comply with both paragraphs 1.E.1 through 1.E.7 and any additional
terms imposed by the copyright holder.  Additional terms will be linked
to the Project Gutenberg-tm License for all works posted with the
permission of the copyright holder found at the beginning of this work.

1.E.4.  Do not unlink or detach or remove the full Project Gutenberg-tm
License terms from this work, or any files containing a part of this
work or any other work associated with Project Gutenberg-tm.

1.E.5.  Do not copy, display, perform, distribute or redistribute this
electronic work, or any part of this electronic work, without
prominently displaying the sentence set forth in paragraph 1.E.1 with
active links or immediate access to the full terms of the Project
Gutenberg-tm License.

1.E.6.  You may convert to and distribute this work in any binary,
compressed, marked up, nonproprietary or proprietary form, including any
word processing or hypertext form.  However, if you provide access to or
distribute copies of a Project Gutenberg-tm work in a format other than
"Plain Vanilla ASCII" or other format used in the official version
posted on the official Project Gutenberg-tm web site (www.gutenberg.org),
you must, at no additional cost, fee or expense to the user, provide a
copy, a means of exporting a copy, or a means of obtaining a copy upon
request, of the work in its original "Plain Vanilla ASCII" or other
form.  Any alternate format must include the full Project Gutenberg-tm
License as specified in paragraph 1.E.1.

1.E.7.  Do not charge a fee for access to, viewing, displaying,
performing, copying or distributing any Project Gutenberg-tm works
unless you comply with paragraph 1.E.8 or 1.E.9.

1.E.8.  You may charge a reasonable fee for copies of or providing
access to or distributing Project Gutenberg-tm electronic works provided
that

- You pay a royalty fee of 20% of the gross profits you derive from
     the use of Project Gutenberg-tm works calculated using the method
     you already use to calculate your applicable taxes.  The fee is
     owed to the owner of the Project Gutenberg-tm trademark, but he
     has agreed to donate royalties under this paragraph to the
     Project Gutenberg Literary Archive Foundation.  Royalty payments
     must be paid within 60 days following each date on which you
     prepare (or are legally required to prepare) your periodic tax
     returns.  Royalty payments should be clearly marked as such and
     sent to the Project Gutenberg Literary Archive Foundation at the
     address specified in Section 4, "Information about donations to
     the Project Gutenberg Literary Archive Foundation."

- You provide a full refund of any money paid by a user who notifies
     you in writing (or by e-mail) within 30 days of receipt that s/he
     does not agree to the terms of the full Project Gutenberg-tm
     License.  You must require such a user to return or
     destroy all copies of the works possessed in a physical medium
     and discontinue all use of and all access to other copies of
     Project Gutenberg-tm works.

- You provide, in accordance with paragraph 1.F.3, a full refund of any
     money paid for a work or a replacement copy, if a defect in the
     electronic work is discovered and reported to you within 90 days
     of receipt of the work.

- You comply with all other terms of this agreement for free
     distribution of Project Gutenberg-tm works.

1.E.9.  If you wish to charge a fee or distribute a Project Gutenberg-tm
electronic work or group of works on different terms than are set
forth in this agreement, you must obtain permission in writing from
both the Project Gutenberg Literary Archive Foundation and Michael
Hart, the owner of the Project Gutenberg-tm trademark.  Contact the
Foundation as set forth in Section 3 below.

1.F.

1.F.1.  Project Gutenberg volunteers and employees expend considerable
effort to identify, do copyright research on, transcribe and proofread
public domain works in creating the Project Gutenberg-tm
collection.  Despite these efforts, Project Gutenberg-tm electronic
works, and the medium on which they may be stored, may contain
"Defects," such as, but not limited to, incomplete, inaccurate or
corrupt data, transcription errors, a copyright or other intellectual
property infringement, a defective or damaged disk or other medium, a
computer virus, or computer codes that damage or cannot be read by
your equipment.

1.F.2.  LIMITED WARRANTY, DISCLAIMER OF DAMAGES - Except for the "Right
of Replacement or Refund" described in paragraph 1.F.3, the Project
Gutenberg Literary Archive Foundation, the owner of the Project
Gutenberg-tm trademark, and any other party distributing a Project
Gutenberg-tm electronic work under this agreement, disclaim all
liability to you for damages, costs and expenses, including legal
fees.  YOU AGREE THAT YOU HAVE NO REMEDIES FOR NEGLIGENCE, STRICT
LIABILITY, BREACH OF WARRANTY OR BREACH OF CONTRACT EXCEPT THOSE
PROVIDED IN PARAGRAPH F3.  YOU AGREE THAT THE FOUNDATION, THE
TRADEMARK OWNER, AND ANY DISTRIBUTOR UNDER THIS AGREEMENT WILL NOT BE
LIABLE TO YOU FOR ACTUAL, DIRECT, INDIRECT, CONSEQUENTIAL, PUNITIVE OR
INCIDENTAL DAMAGES EVEN IF YOU GIVE NOTICE OF THE POSSIBILITY OF SUCH
DAMAGE.

1.F.3.  LIMITED RIGHT OF REPLACEMENT OR REFUND - If you discover a
defect in this electronic work within 90 days of receiving it, you can
receive a refund of the money (if any) you paid for it by sending a
written explanation to the person you received the work from.  If you
received the work on a physical medium, you must return the medium with
your written explanation.  The person or entity that provided you with
the defective work may elect to provide a replacement copy in lieu of a
refund.  If you received the work electronically, the person or entity
providing it to you may choose to give you a second opportunity to
receive the work electronically in lieu of a refund.  If the second copy
is also defective, you may demand a refund in writing without further
opportunities to fix the problem.

1.F.4.  Except for the limited right of replacement or refund set forth
in paragraph 1.F.3, this work is provided to you 'AS-IS' WITH NO OTHER
WARRANTIES OF ANY KIND, EXPRESS OR IMPLIED, INCLUDING BUT NOT LIMITED TO
WARRANTIES OF MERCHANTIBILITY OR FITNESS FOR ANY PURPOSE.

1.F.5.  Some states do not allow disclaimers of certain implied
warranties or the exclusion or limitation of certain types of damages.
If any disclaimer or limitation set forth in this agreement violates the
law of the state applicable to this agreement, the agreement shall be
interpreted to make the maximum disclaimer or limitation permitted by
the applicable state law.  The invalidity or unenforceability of any
provision of this agreement shall not void the remaining provisions.

1.F.6.  INDEMNITY - You agree to indemnify and hold the Foundation, the
trademark owner, any agent or employee of the Foundation, anyone
providing copies of Project Gutenberg-tm electronic works in accordance
with this agreement, and any volunteers associated with the production,
promotion and distribution of Project Gutenberg-tm electronic works,
harmless from all liability, costs and expenses, including legal fees,
that arise directly or indirectly from any of the following which you do
or cause to occur: (a) distribution of this or any Project Gutenberg-tm
work, (b) alteration, modification, or additions or deletions to any
Project Gutenberg-tm work, and (c) any Defect you cause.


Section  2.  Information about the Mission of Project Gutenberg-tm

Project Gutenberg-tm is synonymous with the free distribution of
electronic works in formats readable by the widest variety of computers
including obsolete, old, middle-aged and new computers.  It exists
because of the efforts of hundreds of volunteers and donations from
people in all walks of life.

Volunteers and financial support to provide volunteers with the
assistance they need, is critical to reaching Project Gutenberg-tm's
goals and ensuring that the Project Gutenberg-tm collection will
remain freely available for generations to come.  In 2001, the Project
Gutenberg Literary Archive Foundation was created to provide a secure
and permanent future for Project Gutenberg-tm and future generations.
To learn more about the Project Gutenberg Literary Archive Foundation
and how your efforts and donations can help, see Sections 3 and 4
and the Foundation web page at http://www.pglaf.org.


Section 3.  Information about the Project Gutenberg Literary Archive
Foundation

The Project Gutenberg Literary Archive Foundation is a non profit
501(c)(3) educational corporation organized under the laws of the
state of Mississippi and granted tax exempt status by the Internal
Revenue Service.  The Foundation's EIN or federal tax identification
number is 64-6221541.  Its 501(c)(3) letter is posted at
http://pglaf.org/fundraising.  Contributions to the Project Gutenberg
Literary Archive Foundation are tax deductible to the full extent
permitted by U.S. federal laws and your state's laws.

The Foundation's principal office is located at 4557 Melan Dr. S.
Fairbanks, AK, 99712., but its volunteers and employees are scattered
throughout numerous locations.  Its business office is located at
809 North 1500 West, Salt Lake City, UT 84116, (801) 596-1887, email
business@pglaf.org.  Email contact links and up to date contact
information can be found at the Foundation's web site and official
page at http://pglaf.org

For additional contact information:
     Dr. Gregory B. Newby
     Chief Executive and Director
     gbnewby@pglaf.org


Section 4.  Information about Donations to the Project Gutenberg
Literary Archive Foundation

Project Gutenberg-tm depends upon and cannot survive without wide
spread public support and donations to carry out its mission of
increasing the number of public domain and licensed works that can be
freely distributed in machine readable form accessible by the widest
array of equipment including outdated equipment.  Many small donations
($1 to $5,000) are particularly important to maintaining tax exempt
status with the IRS.

The Foundation is committed to complying with the laws regulating
charities and charitable donations in all 50 states of the United
States.  Compliance requirements are not uniform and it takes a
considerable effort, much paperwork and many fees to meet and keep up
with these requirements.  We do not solicit donations in locations
where we have not received written confirmation of compliance.  To
SEND DONATIONS or determine the status of compliance for any
particular state visit http://pglaf.org

While we cannot and do not solicit contributions from states where we
have not met the solicitation requirements, we know of no prohibition
against accepting unsolicited donations from donors in such states who
approach us with offers to donate.

International donations are gratefully accepted, but we cannot make
any statements concerning tax treatment of donations received from
outside the United States.  U.S. laws alone swamp our small staff.

Please check the Project Gutenberg Web pages for current donation
methods and addresses.  Donations are accepted in a number of other
ways including checks, online payments and credit card donations.
To donate, please visit: http://pglaf.org/donate


Section 5.  General Information About Project Gutenberg-tm electronic
works.

Professor Michael S. Hart is the originator of the Project Gutenberg-tm
concept of a library of electronic works that could be freely shared
with anyone.  For thirty years, he produced and distributed Project
Gutenberg-tm eBooks with only a loose network of volunteer support.


Project Gutenberg-tm eBooks are often created from several printed
editions, all of which are confirmed as Public Domain in the U.S.
unless a copyright notice is included.  Thus, we do not necessarily
keep eBooks in compliance with any particular paper edition.


Most people start at our Web site which has the main PG search facility:

     http://www.gutenberg.org

This Web site includes information about Project Gutenberg-tm,
including how to make donations to the Project Gutenberg Literary
Archive Foundation, how to help produce our new eBooks, and how to
subscribe to our email newsletter to hear about new eBooks.

\end{verbatim}

\end{document}
